\chapter{Introduction}
  Microseconds after the big bang, the universe existed in a state known as
    the Quark Gluon Plasma (QGP).
  In the QGP, quarks and gluons are not in hadronic bondage, forced to 
    the confines of bound states such as protons and neutrons.
  The Large Hadron Collider (LHC) produces QGP in the lab in PbPb (lead-lead)
    collisions.
  The high energies and rates of the collisions at the LHC make it possible 
    to do detailed studies of the QGP. 
  The LHC is producing rare experimental probes such as suppressed jets and 
    heavy quarkonia at an unprecedented rate in heavy-ion collisions. 
  Physicists now have better constraints on the properties like temperature,
    viscosity, and energy density of the QGP.

  The detailed studies of PbPb collisions coming out of the LHC 
    experiments require an understanding of the initial state of the ions 
    before they collide.
  Without knowledge of the initial state, physicists cannot determine which
    experimental effects are due to the QGP and which effects are inherent to
    the nuclei themselves. 
  For example, suppression of heavy quarkonia is a signature of the QGP 
    but also appears to occur in deuterium-gold collisions where the QGP is not
    expected to arise \cite{dAuOniaPHENIX}. 
  Because it is not certain how much of the reduction of quarkonia production
    is due to the initial state of the nuclei, the reduction due to the QGP
    is unclear. 
  Without a clean probe of the initial state, physicists' knowledge of the 
    QGP is limited.
  Ultra-Peripheral Collisions (UPC) at the LHC fill this need for a clean 
    probe.

  \section{Experimental History}
    \subsection{History of HI physics}
      The study of relativistic heavy ion collisions, like PbPb collisions at 
        the LHC, began in mid 80's.
      Relativistic heavy ion collisions were first studied using the 
        Alternating Gradient Synchrotron (AGS) at Brookhaven National Lab (BNL) 
        in Upton, NY, and the Super Proton Synchrotron (SPS) at CERN near 
        Geneva, Switzerland. 
      From the numerous AGS and SPS experiment two main observables emerged,
        \JPsi{} suppression and strangeness enhancement. 
      Both were indications that QGP state had been produced. 

      The AGS and SPS experiments were fixed target experiments.
      At AGS the ion isotopes $^{16}$O, $^{28}$Si, and $^{197}$Au beams were 
        collied with fix targets. 
      At SPS the same fix target configurations were used but with the ion 
        isotopes $^{16}$O, $^{32}$S, and $^{208}$Pb.
      Because the these fixed target colliding configurations rather that beam
        colliders, the collision energies were much lower than today.
      The center of mass energies per nucleon pair for these experiments ranged
        from just below 5 GeV to 20 GeV. 
      At these energies the threshold for creating the QGP of an energy density 
        of 0.15 GeV/fm$^{3}$ and a temperature of 170 MeV was just barely 
        met.
      Though the strangeness enhancement and \JPsi{} suppression 
        signals indicated that there was likely a deconfined state of quarks 
        and gluons created, it likely did not last long enough to reach thermal
        equilibrium or order to be a true plasma. 

      Plans for a colliding beam machine dedicated to heavy ions was first 
        proposed 1983.  
      In the summer of 2000 RHIC began collisions and the four experiments,
        STAR, PHENIX, BRAHMS, and PHOBOS started taking data. 
      With collision energies of 200 GeV per colliding nucleon, the energies 
        at RHIC were a factor of 10 higher than at the previous fixed target 
        expriments. 
      Of the 15 years data taking interesting measurements of the 
        jet-suppression, azimuthal anisotropy, and double ridge in two particle
        correlations have confirmed the production of QGP.
      RHIC produced a thermalized state of quarks and gluons that could be 
        confirmed by experiment for the first time. 
      
      The LHC heavy ion program began collisions in 2010 and reignited the 
        BNL/CERN trans-atlantic competition. 
      The PbPb collisions at CERN increased the collision energy by another 
        factor of 10 to 2.76 TeV per nucleon pair. 
      At the LHC the experiments ALICE, ATLAS, and CMS have been producing 
        measurements for the last 4 years. 
      The higher energies of LHC and sophistication of the LHC experiments have
        embarked on an new era of precision heavy ion measurements and unlock
        new questions.

    \subsection{History of UPC physics}
      



  \section{History of heavy ion physics}
    \subsection{AGS and SPS}
    \subsection{RHIC and LHC}

  \section{History of UPC pysics}
    \subsection{RHIC}
    \subsection{LHC}
