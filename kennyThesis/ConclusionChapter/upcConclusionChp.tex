\chapter{Summary}
  As physicists' understanding of the QGP has deepened over the past 30 years
    of doing experimental heavy ion physics, the questions surrounding the
    QGP have shift from the confirmation of creation of a deconfined state
    to understanding the properties of the state that is created.
  It appears that the QGP is a hot nearly viscosity free fluid of strongly 
    coupled quarks and gluons. 
  The control measurements from dAu collisions at RHIC, and pPb collisions
    at the LHC have shown signs of collective behavior such as flow, which 
    made these results difficult to interpret.
  Additional knowledge of the initial state of the colliding nuclei is needed
    in order to fully understand the QGP signal seen in PbPb and AuAu 
    collisions.
  UPC events can provide this needed knowledge. 
  This thesis contributes to the understanding of the initial state through
    the measurement of the UPC \JPsi{} photoproduction cross section. 

  Ultra-peripheral collisions are clean probe of the initial state. 
  In UPC \JPsi{} photoproduction, the nuclei interact through the 
    electromagnetic force precluding the possibility of creating a collective
    medium.
  The theoretical models of coherent UPC \JPsi{} photoproduction model these 
    electromagnetic interactions by combining a semi-classical calculation 
    of the photon flux with a variety of phenomenological and QCD based 
    calculations of the nuclear gluon density. 
  The Wis\"{a}cker-Williams approximation \cite{WWFermi} is used to calculate 
    the flux of photons that surround the colliding nuclei. 
  The interaction of these photons with the nucleus is either calculated 
    through a nuclear modification of the proton gluon density 
    \cite{pQCD2013.02, lta2012.03}, or by modeling the nucleus as a collection
    of nucleons and scaling the nucleon photoproduction cross-sections from 
    e-p collisions \cite{vmd1999}, a Glauber model approach. 
  Photoproduction cross sections from from UPC events can determine at what 
    energy scale the Glauber based method breaks down.
  For the gluon density based calculations, there is a wide descrepancy 
    between the predictions, and photoproduction cross sections constrain which
    gluon density models are viable. 

  In this thesis, the CMS detector was used to measure the coherent UPC \JPsi{} 
    photoproduction cross section.
  The all of the three  major subsystems of CMS were used, the tracker, 
    the muon system, the calorimeter system, were used.
  The tracker records the position charge excitations in the silicon due to 
    particle hits, which are used to reconstruct the trajectory of charged 
    particles. 
  The muon system is comprised of the three gaseous detectors, the DTs, the 
    RPC, and the CSCs, which record charge deposits as high momentum particles
    ionize the gass within the detector.
  The muon system primary purpose is for triggering on and identifying muons.
  The calorimeter system measures the energy of deposited by particle induced
    showers as a means of reconstructing neutral particles and jets.
  
  The analysis in this thesis consists of three major components, development
    of a trigger, estimation of efficiency, and measurement of signal events.
  The trigger development involved designing a trigger based on rate estimates
    from past data that ensured a sample that could be used for both measuring
    the signal and estimate the efficiency of the trigger, reconstruction, and
    event selection.
  The number signal \JPsi{} candidates was measured by first applying a set 
    of event selection cuts that rejected background events such as hadronic
    collisions and beam gas collisions, then fitting the remaining events to
    templates from simulation to separate the three remaining physics processes,
    the coherent, incoherent, and photon-photon process.
  The efficiencies for each part of the trigger were measured from data. 
  The acceptance and reconstruction efficiency were estimated from MC.
  The cross section was calculated by combining the efficiency with the 
    measured luminosity and number of coherent \JPsi{}.
  The statistical uncertainties were taken from the template fit.
  The systematic uncertainties were estimated by varying the method used on 
    each component of the analysis. 

  The UPC \JPsi{} photoproduction cross section, $\frac{d\sigma^{J/\psi}_{co}}{dy}$,
    was found to be \textcolor{red}{368 $\pm$ 38 $\mu$b}. 
  When rescaled by a factor of 5.06 to account for the difference of break-up mode between 
    the measurement in this thesis and the ALICE result in Ref.\cite{alice2012.09}, the 
    result of \textcolor{red}{1.86 $\pm$ 19 mb} was found to be consistent with the 
    predictions in Ref.\cite{pQCD2013.02} of \textcolor{red}{1.8 mb}.  
  The calculation in Ref.\cite{pQCD2013.02} is also favored by the ALICE measurements. 
  Of the gluon distributions used in Ref.\cite{pQCD2013.02}, a gluon distribution with 
    moderately strong gluon shadowing, EPSO9, is consistent with both the results from this
    thesis and the previous ALICE results. 
  This indicates that at the scale of the mass of the \JPsi{} the nucleus gluon density is 
    significantly suppressed compared to the gluon densities of a nucleon.
  At this scale the nucleus can not be represented as a collection of nucleons as in the 
    Glauber like model described in Ref.\cite{vmd1999}.

  The measurement in this thesis confirms the ability to increase the knowledge of the 
    initial state though the exploration of UPC events. 
  This confirmation opens the door to additional measurements in this growing field of UPC
    research.
  A whole host of measurements will be possible with the data already recorded and will be
    recorded in the coming years by CMS and the other LHC experiments. 

  Ultra-peripheral heavy ion collisions provide an opportunity to study 
    the nature of the initial state. 
  Model calculations of the nuclear gluon density can be constrained by 
    studying quarkonia photoproduction at the LHC.
  In this thesis, the coherent \JPsi{} photoproduction cross section in 
    ultra-peripheral PbPb collisions at $sqrt(s_{NN})$ = 2.76 TeV.
  This was done by analysing data from the 2011 PbPb run recorded by the CMS 
    detector.
  A brief description of the detector apparatus was presented in this thesis.
  A special set of triggers was developed for the analysis in this thesis.
  These events are characterized by low \pt{} dimuons.
  Although the CMS detector was not designed to study this type of event, the
    analysis in this thesis demonstrates how versatile CMS can be at handling 
    both low \pt{} and low multiplicity events. 
  However, this analysis suffers from small statistics. 
  This is due to low acceptance for triggering on low \pt{} muons as well as 
    the difficulty to trigger these muons in coincidence with neutrons in the 
    ZDC.
  
    
  
