\chapter{Future Works}	
\section{High mass $\gamma-\gamma \rightarrow e^{+} e^{-}$  in PbPb 2011}
    A study of the di-electron production in UPC events is already possible 
      from the recorded 2011 data. 
    This measurement would make use of the electron triggers and combined the 
      current di-muon data with di-electron data from the triggers using the
      ECAL. 
    The electron triggered sample potentially offers a large increase in 
      statistics. 
    By adding the additional channel the statistics would already increase.
    However in addition to this, because of the smaller mass of the electron,
      di-electron production is slightly favor compared to di-muon 
      production.
    STARlight predicts that di-electron cross section is a factor of more than 
      2.5 higher in Xn break-up mode than the di-muons channel when looking 
      at masses above 4 GeV.
    The acceptance for electrons in potential higher as well. 
    The ECAL is position just beyond the tracker, whereas the muon system is 
      outermost sub-detector. 
    This elevates the main reduction of muon acceptance, which is the material
      budget. 
    There is simply a lot a detector in front the muon system.

    In order to perform the study several key additions would need to be made
      relative to the current di-muon analysis. 
    The original reconstruction of the data used in the current di-muon 
      analysis does not contain electron objects. 
    Either the analysis would have to migrated to reconstruction of the data
      done in a newer software version, or reconstruction of the electrons 
      would have to be added to the current analysis chain. 
    There are currently no electron UPC MC samples produced. 
    In order to study the acceptance and efficiency for electrons these samples
      would be need. 
    The ultimate limitation on this study is the 2 GeV threshold in $p_{T}$ in
      the ECAL trigger. 
    This limits the di-electron mass range to where the trigger is efficient. 

    The contribution of higher order diagrams can be explored by the 
      photoproduction of di-lepton pairs is to explore.
    With additional contributions to the physics communities understanding of 
      this process this study will help to determine necessity or 
      non-necessity of including higher order of corrections in simulations 
      such as STARlight.
    Having an additional channel to help constrain the current di-muon measure 
      of the of UPC $\gamma-\gamma$ interaction will also help to constrain 
      the $J/\psi$ measurement by adding a data driven check on the 
      normalization $\gamma-\gamma$ background to the $J/\psi$
    
  \section{UPC Hadronic Overlap and PbPb 2011}
    A detailed exploration of the exclusivity cuts could both explore 
      photoproductions contribution to the inclusive $J/\psi$ measurements in
      PbPb and the validity of rejecting hadronic interactions in UPC 
      photoproduction cross-section calculations. 
    Coherent quarkonia photoproduction has distinctive low $p_{T}$ structure
      that can be used to identify photoproduced candidates.
    In addition to investigating the overlap between photoproduction and 
      inelastic nuclear collisions, examination of the exclusivity cuts will 
      allow for search of candidates where one muons is in the tracker but 
      the other muon is in HF. 

  \section{pPb J/Psi}
  \section{UPC J/Psi 2015}
  \section{UPC Upsilon 2015}
