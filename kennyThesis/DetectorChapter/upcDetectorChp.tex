\chapter{The CMS Detector}	
CMS is housed at interaction point 5 of the LHC. 
The LHC is designed to pursue physics at the TeV scale. 
This is the scale where electroweak symmetry breaking is believed to occur
	\cite{CmsPTdrv2}.
While this means that the search for the standard model Higs is the central 
	driving design consideration, the wide range of possibilities for
	finding new physics signals requires a general purpose detector.
The expedient discovery of new physics through low cross section interactions 
	requires high luminosity.
This consideration leads inevitably to pile up, where multiple collisions 
	occurs at a single bunch crossing.
At peak luminosity the LHC is expected to produce on average 20 hard 
	proton-proton (pp) collisions per bunch crossing \cite{tCmsE}.
These particle physics considerations of high multiplicity due to pileup and the
	need for a general purpose detector make CMS serendipitously well suited
	for heavy ion physics.
\begin{figure}[h]
  \centering
    \includegraphics[width=.5\textwidth]{cms}
  \caption{The Compact Muon Solenoid from Reference~\cite{tCmsE}.}
  \label{cms}
\end{figure}

The general purpose design of CMS is dominated by the massive 4T 
	superconducting solenoid at its core.
The magnets is 13m long with a 6m diameter, and pushes the limits of power
	and compactness \cite{tCmsE}. 
These two conflicting limits are achieved through the novel design of 
	interweaving structural and conducting elements together in the coil of
	the solenoid.

Within the solenoid resides three different sub detectors.
The inner most is the world's largest silicon tracker \cite{tCmsE}.
The tracker is surrounded by a highly effective lead tungstate crystal 
	electromagnetic calorimeter (ECAL).
ECAL is encapsulated in a brass scintillating hadronic calorimeter (HCAL).
Outside the magnet, muon chambers are used to aid in the measurement and 
	triggering of muon events. 
Altogether CMS weighs 12,500 metric tons, has a diameter of 14.6m,
	and a length of 21.6m \cite{tCmsE}.

The Silicon Tracker is the innermost sub-detector of CMS, and has active
	elements as close as 4.4cm to the interaction point \cite{tCmsE}. 
The tracker has a length 5.8m, a diameter of 2.6m and
	covers a range in pseudorapidity of \(|\eta| <\) 2.5.
Pseudorapidity is defined as $\eta\equiv-\ln(\tan(\theta/2))$, where $\theta$ is 
	the polar angle, and $\phi$ is the azimuthal angle with respect to the 
	beam axis.
At the center of the tracker are three rings of silicon pixels around the beam 
	with two disks of silicon pixels to cap the rings.
The pixel portion of the silicon tracker is comprised of 66x10$^{6}$
	pixels.
The silicon pixels are surrounded by silicon strips.
The silicon strips are separated into 4 different sections: 
	the Tracker Inner Barrel, the Tracker Inner Disk, the Tracker Outer 
	Barrel, and the Tracker End Caps.
The silicon strip detectors as a whole are comprised of 9.3x10$^{6}$ silicon 
	strips.
The high number of pixels and strips allow for the ability to distinguish
	and collect enough distinct points to reconstruct the path of the 1000
	or so charge particles per bunch crossing expected at peak luminosity
	\cite{tCmsE}.  

The next detector beyond the tracker is ECAL.
ECAL is made of 61,200 lead tungstate (PbWO$_{4}$) crystals in the central
	barrel and 7,324 on each of the two endcaps \cite{tCmsE}.
The barrel (EB) covers a pseudorapidity range $|\eta| < 1.479$ and has an
	approximate $\eta-\phi$ segmentation of $0.0174\times0.0174$.
Lead tungstate is very dense, which is reflect in the high number of interaction
	lengths the short depth of one crystal provides.
The crystals of the barrel have a depth of 230 mm corresponding to 25.8 
	radiation lengths ($X_{0}$).
The radiation length is the mean distance a high energy particle travels before
	giving up one e-fold of kinetic energy through electromagnetic
	interactions.
For example, after one radiation length $E \rightarrow E/e$, where 
	$e = 2.71828183$. 
The endcaps (EE) cover the psuedorapitity region $1.479 < |\eta| < 3$.
In the endcap the crystals have an exposed area of 28.62 $\times$ 28.62 
	mm$^{2}$, and a depth of 220 mm corresponding to 24.7 $X_{0}$.
The energy resolution of the ECAL as measured by test beam data can be seen in
	Figure~\ref{ECALeRes}.
\begin{figure}[h]
  \centering
    \includegraphics[width=0.5\textwidth]{ECALeRes}
  \caption{The energy resolution of ECAL as a function of energy from 
	Reference~\cite{tCmsE}.}
  \label{ECALeRes}
\end{figure}

The HCAL like the ECAL has both a barrel (HB) and endcaps (HE).
The pseudorapidity region $|\eta|<1.3$ is covered by HB \cite{tCmsE}. 
HB has an $\eta-\phi$ segmentation of $0.0897\times0.0897$, and is 25 times more
	sparsely granulated than EB.
HE covers the pseudorapidity region $1.3<|\eta|<3$.
HE, like EE and the tracker endcaps, is aligned perpendicular to the beam axis
	resulting in granularity that changes with $\eta$.
In the region $1.3 <|\eta|< 1.6$ HE has an $\eta-\phi$ segmentation of 
	$0.0897\times0.0897$.
The $\eta-\phi$ segmentation roughly doubles to $0.17\times0.17$ in the region
	$1.6 <|\eta|< 3$.
The energy resolution of the barrel and endcaps can be seen in  
	Figure~\ref{HCALeRes}.
The thickness of the hadronic calorimeter is best described in interaction
	lengths, the mean distance for a particle to give up an e-fold of energy
	through nuclear interactions. 
At $\eta = 0$ the barrel has a thickness 5.82 interaction lengths 
	($\lambda_{I}$), and increases as the path length through the material 
	increases to 10.6 $\lambda_{I}$ at $|\eta| = 1.3$.
\begin{figure}[h]
  \centering
    \includegraphics[width=0.5\textwidth]{HCALeRes}
  \caption{The $E_{T}$ resolution of HCAL as a function of $|\eta|$ and $E_{T}$
	from Reference~\cite{tCmsE}.}
  \label{HCALeRes}
\end{figure}

In addition to HB and HE, HCAL has two additional calorimeters.
Because the space between ECAL and the magnet is restricted to 1.18 m, an
	outer hadronic calorimeter section (HO) is placed beyond the magnet
	in the region $|\eta|<1.3$ \cite{tCmsE}.
The main function of HO is to collect energy from the highest energy hadrons
	before they reach the muon system.
HO is not used in this analysis, but does contribute to the material budget. 
To increase the total calorimetric coverage, HCAL also has a quartz fiber 
	calorimeter (HF) in the forward region, $3 < |\eta| < 5$.
For the majority of HF's 13 $\eta$ rings the $\eta-\phi$ segmentation is 
	$0.175\times0.175$.
In the lowest $|\eta|$ ring the segmentation is $0.111\times0.175$ in 
	$\eta-\phi$.
In the highest two $|\eta|$ rings the segmentation in $\phi$ is 0.349, with an
	$\eta$ segmentation of 0.175 in the outer and 0.300 in the innermost 
	ring. 
The longitudinal direction is effectively segmented by using short fibers and
	long fibers.
The measure energy deposited deeper than 22 cm is measured in both the short
	and long fibers, where as the long fibers are present throughout.
This allows electromagnetic showers to be distinguished from purely hadronic 
	showers \cite{tCmsE}.
The energy resolution for HF can be seen in Figure~\ref{HCALeRes}.  

Beyond HF there are two more detectors in the forward region.
CASTOR covers the range 5.2 $< \eta <$ 6.6 on the positive side of the beam. 
The Zero Degree Calorimeters (ZDC) sit between the beam pipes on either side of
	the interaction point covering the area around $\theta = 0$, $|\eta| > 
	8.3$.
In heavy ion collisions the ZDC has the ability to measure neutral particles 
	that do not participate in the collision \cite{tCmsE}.
CASTOR extends the total coverage of the CMS as whole giving more access to 
	low-x physics \cite{tCmsE}.

The ZDC has a total of 18 channels.
Half of these 18 channels are on either side of the interaction point.
The 9 channels on the side of CMS that correspond to positive $\eta$
  are denoted ZDC$^{+}$, where as the 9 channels on the negative side are
  denoted ZDC$^{-}$.
The 9 channels on each side are further sub-divided into an electro-magnetic  
  (EM) section and a hadronic (HAD) section.
The EM section is positioned in front of the HAD section with respect to the 
  interaction point and is segmented transverse to the beam direction.
The 5 EM sections are positioned in front to absorb the energy from 
  electro-magnetically induced showers, which develop over a shorter distance 
  than hadronically induced showers.
The transverse segmentation allows for a measurement of the transverse shower
  width and the size of the beam spot at the ZDC.
The HAD section is segmented in the direction of the beam and consists of 4
  channels.
The longitudinal segmentation allows for absorption of the full extended 
  hadronic shower and the ability to measure the longitudinal shower shape.

Each the 18 channels contains a tungsten target and quartz fibers.
The dense tungsten target is used to initiate the shower.
The quartz fibers shine Cerenkov light as the high momentum charged particles
  from the shower pass through it. 
the light from the quartz fibers is channeled to photo-multiplier tubes, one 
  for each ZDC channel. 
Through a cascade of photon induced electrical discharges, the photo-multiplier
  converts the Cerenkov light to an electrical pulse. 

This electrical pulse travels $\sim$ 200 m down a coaxial cable from the LHC
  tunnel to the counting house in the CMS service cavern. 
There the electrical pulse is digitized by the Charge Integrator and Encoder 
  (QIE).
The QIE integrates the current each 25 nano seconds.
The charge is than mapped logarithmically to the 128 bits. 
This bit is sent across a small fiber optic cable to the HTR firmware card.
Here each 25 ns signal is stored in a 250 ns buffer, and the timing is sync
  with the rest of the detector to insure the ZDC signal arrives at the central
  data acquisition system at the same time as the other sub detectors from the 
  same collision. 

The muon system resides just outside of the superconducting magnet.
It consists of three complementary systems: drift tube (DT) chambers in the
	barrel, cathode strip chambers (CSC) in the endcaps, and resistive 
	plate chambers (RPC) in both the barrel and endcap regions \cite{tCmsE}.
Ultimately the muon system is most useful for triggering on muons \cite{tCmsE}.

The heavy ion community is making use of the capabilities of CMS in a myriad of
	ways.
The muon trigger has been used in the search for suppression of quarkonium 
	states. 
This is an important probe of the correlation length within the hot dense state
	known as the quark gluon plasma (QGP).
The tracker has been utilized for to study charged particle multiplicities, and
	and elliptical flow, two probes of the thermal expansion of the QGP.
HCAL has aided in measuring jet suppression, which probes the strength with 
	which the QGP interacts with strong interacting objects.
Through its general purpose design and its ability to handle the high
	multiplicities produce by the LHC, CMS proves to be an excellent 
	detector for investigating strongly interacting mater through heavy ion
	collisions. 
%  \section{CMS general}
%  \section{Muons}
%  \section{HCal}
%  \section{ZDC}
  \section{Trigger}
    The CMS trigger is two teired. 
    The L1 trigger is the lower level hardward based system. 
    The High Level Trigger (HLT) is software base and runs on a computer farm
      at point 5 where CMS is housed. 
