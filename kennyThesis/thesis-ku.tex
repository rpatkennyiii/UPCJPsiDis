%% LyX 2.0.3 created this file.  For more info, see http://www.lyx.org/.
%% Do not edit unless you really know what you are doing.
\documentclass[12pt,english]{kuthesis}
\usepackage{mathptmx}
\renewcommand{\sfdefault}{lmss}
\renewcommand{\ttdefault}{lmtt}
\usepackage[T1]{fontenc}
\usepackage[utf8]{inputenc}
\usepackage{listings}
\usepackage{geometry}
\geometry{verbose,tmargin=1in,bmargin=1in,lmargin=1in,rmargin=1in}
\setcounter{secnumdepth}{3}
\setcounter{tocdepth}{3}
\usepackage{url}
\usepackage{graphicx}
\usepackage{wrapfig}
\usepackage{setspace}
\usepackage{esint}
\usepackage{hyperref}
\usepackage{xcolor}
\usepackage{easylist}
\usepackage{multirow}
\usepackage{hhline}
%\usepackage{mathtools}
%\usepackage[authoryear]{natbib}

\graphicspath{{IntroductionChapter/figures/},{TheoryChapter/figures/}
  ,{DetectorChapter/figures/},{AnalysisChapter/figures/}
  ,{ResultsChapter/figures/},{ConclusionChapter/figures/}
  ,{FutureWorksChapter/figures/},{ZDCRecoChapter/figures/}
  ,{TriggerChapter/figures/},{SystematicsChapter/figures/}}

\newcommand{\JPsi}{J/$\psi$}
\newcommand{\pt}{$p_{T}$}

\doublespacing

\makeatletter

%%%%%%%%%%%%%%%%%%%%%%%%%%%%%% LyX specific LaTeX commands.
\providecommand{\LyX}{L\kern-.1667em\lower.25em\hbox{Y}\kern-.125emX\@}
%% Because html converters don't know tabularnewline
\providecommand{\tabularnewline}{\\}

%%%%%%%%%%%%%%%%%%%%%%%%%%%%%% User specified LaTeX commands.

%used to align decimals in tables according to APA style
\usepackage{dcolumn}
\usepackage{booktabs}

% Set the title and author info
\title{J/$\psi$ photoproduction in ultra-peripheral PbPb collisions at $\sqrt{s_{NN}}$ = 2.76 TeV with CMS}
\author{R. Patrick Kenny III}


\dept{the Department of Physics and Astronomy}
\degreetitle{Doctor of Philosophy}
\papertype{Dissertation} %capitalization is important here
\committee{Dr. Michael Murray}{Dr. Daniel Tapia Takaki}{Dr. Stephen Sanders}{Dr. Dave Besson, Dr. Kyoungchul Kong}{Dr. Ann Cudd}
%AT Most 5 members allowed, last here is blank on purpose to demonstrate
%flexibility

%% These settings are now in the kuthesis.cls file, but users are free
% to customize. listings has great documentation online
%% When listings are used, break lines
%\lstset{
 %    breaklines=true,  % sets automatic line breaking
 %    breakindent=2em,
 %    breakatwhitespace=true,  % sets if automatic breaks should
 %   breakautoindent=true
%}

%% The following is OPTIONAL. Remove all 3 of the next 3 lines
%% to leave dates blank. If dates are included, then both dates 
%% must be included.
\@printd@testrue
\datedefended{August 28, 2014}
\dateapproved{December 16, 2014}

\@ifundefined{showcaptionsetup}{}{%
 \PassOptionsToPackage{caption=false}{subfig}}
\usepackage{subfig}
\makeatother

\usepackage{babel}
\begin{document}
\begin{romanpages}

\maketitle
\begin{abstract}
  In ultra-peripheral heavy-ion collisions, photon-induced interactions between
    colliding nuclei occur at large impact parameters.
  In this thesis, a study of \JPsi{} photoproduction in ultra-peripheral PbPb 
    collisions at $\sqrt{s_{NN}}=2.76$ TeV is presented.
  This was carried out using data collected by the Compact Muon Solenoid in 
    2011. 
  An integrated luminosity of 143 $\mu$$b^{-1}$ was analyzed for this thesis
    using specifically designed triggers.
  The cross section for coherent \JPsi{} photoproduction in the single sided
    break-up mode, $\frac{d\sigma^{J/\psi}}{dy}(Xn0n)$, was measured in the 
    rapidity interval $2.0 < |y| < 2.2$ and was found to be 
    356 $\pm$ 71 (stat) $^{+43}_{-50}$ (syst) $\mu$b.
  The total cross section, after correcting for break-up mode, 
    is found to be 1.80 $\pm$ 0.37 (stat) $^{+0.22}_{-0.25}$ (syst) mb, in 
    agreement with theoretical predictions and previous measurements by the 
    ALICE collaboration.
  The cross section measurement is consistent with \JPsi{} photoproduction 
    models that contain significant nuclear gluon shadowing.
  In addition, the correlation between the \JPsi{}'s and neutrons 
    rapidities was measured and found to qualitatively agree with recent 
    theoretical calculations. 
  The measured rapidity correlation provides a way to distinguish between 
    coherent and incoherent \JPsi{} photoproduction. 

\begin{acknowledgementslong}
%%if you want a "quote" environment for acknowledgements,
%% use acknowledgements instead of acknowledgementslong
  It has been a long road.
  I find it shocking to think that it is really done. 
  It is pretty surreal to see Dr. Pat on things, but I can't say I mind.
  Actually, it feels pretty good.  

  My parents met at KU, so I've been a Jayhawk since the day I was born.
  I owe much of what I've accomplished to The University of Kansas.
  Both my undergraduate and graduate careers took place within the nurturing 
    environment of the KU physics department.
  I can't imagine a more supportive place to have grown from a student to a 
    physicist. 

  Dave Besson gave me my first opportunity to really ``do'' physics research 
    as an undergraduate.
  I am immensely grateful for the experience. 
  The things I learned in his lab have been invaluable to me in my academic 
    career.
  Thank you for giving me a chance.
  
  I began working with my advisor, Michael Murray, during my second year of 
    graduate school.
  His unique combination of patience and pushiness, his soft spoken voice, but 
    firm resolve, provided me with both the space and direction that I needed
    to become the researcher I am today. 
  In particular, I am especially thankful for the encouragement and support 
    Michael provided, which convinced me to take my family to CERN when I 
    didn't think it was possible. 
  My time at CERN was the most formative time in my development as researcher.
  Thanks you for your support. 

  At CERN, I met Magdalena Malek. 
  If not for her, the data for this thesis would not have been recorded nor 
    would the analysis have gotten off the ground.
  Her tenacity is awe inspiring. 
  It was her drive that taught me what it takes to do science at the highest
    level.
  Thanks you for pushing me. 

  When Daniel Tapia Takaki came to KU, he brought the energy and excitement of 
    CERN with him. 
  His enthusiasm is infectious.
  Daniel worked with me when I thought I could not go any further. 
  Without Daniel, this thesis never would have been completed. 
  Thank you for showing me how to close. 

  My family is the foundation for everything I do. 
  My parents always encouraged me to be thoughtful, creative, and idealistic. 
  This is what cultivated the mindset I needed to pursue physics.
  The love of my mom and dad, and the support of my step-mom, Jerry, and my 
    in-laws have sustained me when times were difficult.
  My mom's commitment to my education is something I only hope I can replicate
    for my own children. 
   
  My children are what give my life meaning. 
  Declan, Rowan: I love you.
  My wife has been my biggest supporter, my last bastion of sanity, and love of 
    my life.
  Thank you so very much for traveling on this long road with me. 

\end{acknowledgementslong}
\end{abstract}

\tableofcontents{}

\listoffigures

\listoftables

\end{romanpages}


\chapter{Introduction}
  \section{Overview}
  Microseconds after the Big Bang, the Universe existed in a state known as
    the Quark Gluon Plasma (QGP).
  In the QGP, quarks and gluons are not in hadronic bondage, forced to 
    the confines of bound states such as protons and neutrons.
  The Large Hadron Collider (LHC) produces QGP in the lab in lead-lead (PbPb)
    collisions.
  The high energies and rates of the collisions at the LHC make it possible 
    to do detailed studies of the QGP. 
  The LHC is producing rare experimental probes such as suppressed jets and 
    heavy quarkonia at an unprecedented rate in heavy ion collisions. 
  As a result of recent LHC studies, physicists now have better constraints on 
    the properties like temperature, viscosity, and energy density of the QGP.

  The detailed studies of PbPb collisions coming out of the LHC 
    experiments require an understanding of the initial state of the ions 
    before they collide.
  Without more knowledge of the initial state, physicists cannot determine 
    which experimental effects are due to the QGP and which effects are 
    inherent to the nuclei themselves. 
  For example, suppression of heavy quarkonia is a signature of the QGP 
    but also appears to occur in deuterium-gold collisions where the QGP is not
    expected to arise \cite{dAuOniaPHENIX}. 
  Another important example is measurement of the viscosity, which depends on 
    the relationship between the observed azimuthal anisotropy and the 
    initial eccentricity of the overlap of the two colliding nuclei. 
  A clean probe of the initial state is needed by physicists to comprehensively 
    understand the QGP.
  Ultra-Peripheral Collisions (UPC) at the LHC provide such a probe.

  The current understanding of heavy ion collisions evolved over the
    last 30 years.
  Relativistic heavy ion collisions were first studied using the 
    Alternating Gradient Synchrotron (AGS) at Brookhaven National Lab (BNL) 
    in Upton, NY, followed by the Super Proton Synchrotron (SPS) at CERN near 
    Geneva, Switzerland. 
  From the numerous AGS and SPS experiments two main observables emerged,
    namely, \JPsi{} suppression and strangeness enhancement \cite{sps}. 
  These results pioneered the search for the QGP. 

  The AGS and SPS experiments were fixed target experiments.
  At AGS the ion isotopes $^{16}$O, $^{28}$Si, and $^{197}$Au beams were 
    collied with fix targets. 
  At SPS the same fix target configuration was used, but the ion isotopes were 
    $^{16}$O, $^{32}$S, and $^{208}$Pb.
  The center of mass energies per nucleon pair for these experiments ranged
    from just below 5 GeV to 20 GeV. 
  The threshold for creating the QGP requires an energy density of 0.15 
    $\sim$ GeV/fm$^{3}$ and a temperature near 170 MeV \cite{qgpThresh}.
  AGS and SPS just barely reached this threshold.
  Though the strangeness enhancement and \JPsi{} suppression 
    signals indicated that there was likely a deconfined state of quarks 
    and gluons created, at the energies of the AGS and SPS this state perished 
      to quickly to study any of its properties. 

  Plans for a colliding beam machine dedicated to heavy ions was first 
    proposed 1983.
  The proposed machine was designed to reach energies of 200 GeV per nucleon.
  At these energies, the QGP would presist long enough, and a that signs of
    a gas of hot quarks and gluons would emerge.
  In the summer of 2000 RHIC began collisions and the four experiments,
    STAR, PHENIX, BRAHMS, and PHOBOS started taking data. 
  With collision energies of 200 GeV per colliding nucleon, the energies 
    at RHIC were a factor of 10 higher than was previously achieved. 
  RHIC experiments confirmed for the first time the presence of a thermalized 
    state of quarks and gluons.
  Contrary to expectations, the state found at RHIC was found to be 
    strongly coupled fluid with nearly no viscosity \cite{}.

  The LHC heavy ion program began collisions in 2010, colliding PbPb at 
    a center of mass energy of 2.76 TeV per nucleon pair. 
  This corresponds to an increase in the colliding energy by an order of 
    magnitude with respect to RHIC. 
  The LHC experiments, ALICE, ATLAS, and CMS have studying the heavy ion 
    collisions since then. 
  In 2013 LHCb joined the LHC heavy ion program. 
  Thanks to the LHC and RHIC physics programs, a new era of precision
    heavy ion measurements is underway. 

  The latest results from the LHC have come from the 2013 proton-lead (pPb)
    run.
  This period of data taking was originally designed to be a control 
    measurement.
  For example, the initial suppression signals observed in dAu collisions at 
    RHIC were believed to be due to non-QGP effects \cite{ }. 
  The azimuthal anisotropy of particles present in PbPb and AuAu at the LHC and 
    RHIC respectively were believed to be signals of flow from the QGP and 
    would not appear in the lower density pp and pPb collisions.
  However, CMS showed evidence of a flow signal in high multiplicity pp events
    in early 2011 \cite{ }. 
  More recently ALICE has shown a structure in two particle correlation 
    measurements, referred to as the double ridge \cite{ }, and CMS and ALICE have both 
    shown an elliptical flow signal present in the pPb data \cite{}.

  The latest data from the pPb and dAu measurements confirm the need to 
    understand the nature of the initial state. 
  UPC events fulfill this need by probing the nucleus through photon 
    interactions.
  By measuring UPC \JPsi{} events, theoretical models of the initial state can 
    be constrained.
  In this thesis, the CMS capability for measuring this process, the 
  description of the analysis, and the comparison between the measured 
    coherent \JPsi{} cross section to theoretical models are given. 

  \section{Confirmation and characterization of the QGP from HI measurements}
    Creation of the QPG can be confirmed by comparing the measured energy 
      densities with predicted critical temperature and energy densities
      from lattice QCD measurements. 
    The critical energy density, $\epsilon_{crit}$, and critical temperature, 
      $T_{crit}$, are calculated to be 1.5 GeV and 170 MeV from lattice QCD 
      estimates \cite{ }.
    The measurement of the $\frac{d\E_{T}}{d\eta}$ provides a means of 
      estimating the energy density of the hot state created in heavy ion
      collisions. 
    The temperature can be estimated from the transverse momentum, \pt{}, 
      spectrum of the direct photons, photons that come directly from the 
      QGP.
    At both RHIC and the LHC the energy density and temperature were found to 
      be well above the critical values. 
    The measurements from CMS and ALICE at the LHC and STAR and PHENIX at RHIC
      confirm that the critical values for energy density and temperature are
      exceeded.

    The $\frac{dE}{d\eta}$ measured by CMS \cite{cmsEt} and PHENIX is done 
      using the expirements' calorimeter systems.

    The direct photons were measured this way.

    The energy density and temperature were found to be blah. 
    It was this much higher at LHC than at RHIC

    Prior to the RHIC, the QGP was thought to be a gas of quarks and gluons.
    At RHIC the measurements at STAR showed that the medium appeared to obey 
      hydrodynamic equations and flows like a fluid.
    This same signal was also measured by CMS at the LHC. 

    This is how you measure elliptic flow is measured.

    These are the results. 

    The signal indicates that there is very little viscosity. 
    However, this depends on the eccentricity of initial overlap region, which
      depends on the description of the initial state of the colliding nuclei.

    \begin{figure}[!Hhbt]
      \centering
      \includegraphics{elipFlow}
      \caption{ $v^{2}$ elliptical flow measurements from SPC to the LHC.}
      \label{fig:elipFlow}
    \end{figure}

     \begin{figure}[!Hhbt]
      \centering
      \includegraphics{elipFlowSchem}
      \caption{ Elliptical flow schematic diagram.}
      \label{fig:elipFlowSchem}
    \end{figure}

    \begin{figure}[!Hhbt]
      \centering
      \includegraphics{hadRaaRhic}
      \caption{Hadron R$_{AA}$ from RHIC}
      \label{fig:hadRaaRhic}
    \end{figure}

    \begin{figure}[!Hhbt]
      \centering
      \includegraphics[width=.45\textwidth]{hadRaaLhc}
      \caption{Hadron R$_{AA}$ from the LHC.}
      \label{fig:hadRaaLhc}
    \end{figure}

    \begin{figure}[!Hhbt]
      \centering
      \includegraphics[width=.45\textwidth]{noRaaLhc}
      \caption{R$_{AA}$ for unsuppressed Z bosons and isolated photons.}
      \label{fig:noRaaLhc}
    \end{figure}

  \section{Recent results from HI control measurements}

    \subsection{The HI collision}
      The AGS and SPS created the first signs of a deconfined state, but the 
        nature of the state was uncertain.
      At RHIC the existence of the QGP was confirmed and it's nature found to 
        be hydrodynamic.
      The LHC and RHIC experiments are now looking deeper in the characteristics
        of QGP.
      More precise and sophisticated measurement techniques now require a 
        better understanding of the ions before they collide in order to 
        produce the proper theoretical modeling. 

      Over the course of the experimental evolution the following picture of 
        a heavy ion collision emerged. 
      First, highly contracted ions travel toward each other.
      Second, QGP forms and reaches thermal equilibrium.
      Third, this hot dense state expands hydrodynamically.
      Fifth, Fifth as the collection of quarks and gluons cool, a gas of hot
        hadrons forms and expands.
      Finally, all interactions freeze out and the produced particles stream 
        to the detector. 

\chapter{Theory}
  \section{Introduction}
    Microseconds after the big bang, the universe existed in a state known as
      the Quark Gluon Plasma (QGP).
    In the QGP, quarks and gluons are not in hadronic bondage, forced to 
      the confines of bound states such as protons and neutrons.
    The Large Hadron Collider (LHC) produces QGP in the lab in PbPb (lead-lead)
      collisions.
    The high energies and rates of the collisions at the LHC make it possible 
      to do detailed studies of the QGP. 
    The LHC is producing rare experimental probes such as suppressed jets and 
      heavy quarkonia at an unprecedented rate in heavy-ion collisions. 
    Physicists now have better constraints on the properties like temperature,
      viscosity, and energy density of the QGP. 
    
    The detailed studies of PbPb collisions coming out of the LHC 
      experiments require an understanding of the initial state of the ions 
      before they collide.
    Without knowledge of the initial state, physicists cannot determine which
      experimental effects are due to the QGP and which effects are inherent to
      the nuclei themselves. 
    For example, suppression of heavy quarkonia is a signature of the QGP 
      but also appears to occur in deuterium-gold collisions where the QGP is not
      expected to arise \cite{dAuOniaPHENIX}. 
    Because it is not certain how much of the reduction of quarkonia production
      is due to the initial state of the nuclei, the reduction due to the QGP
      is unclear. 
    Without a clean probe of the initial state, physicists' knowledge of the 
      QGP is limited.
    Ultra-Peripheral Collisions (UPC) at the LHC fill this need for a clean 
      probe. 

    The colliding nuclei interact electromagnetically in an UPC event, avoiding
      the complicated mixing of final state and initial state effects found 
      in nuclear collisions.
    In UPC events, no QGP state emerges, and the effects arising from the QGP 
      no longer obscure the initial state effects.
    Other initial state probes such as peripheral nuclear collisions and 
      proton-nucleus collisions have the potential to create the QGP obscuring 
      which effects come from the initial state.
    It is impossible to create the QGP in UPC events because the nucleons 
      within the nucleus do not collide. 
    UPC events provide clarity by enhancing physicists' 
      understanding of the initial state. 
    
    The interactions between the field of photons surrounding the colliding 
      nuclei and the gluons of nuclei can produce a $J/\Psi$ probing the 
      gluon density.
    The UPC $J/\Psi$ photoproduction cross section is therefore a probe of 
      the initial state of the nucleus. 
    The Weizsi\"{a}cker-Williams approximation provides a way to calculate the 
      density of probing photons that surrounds the nucleus. 
    The electron-proton scattering data gives a value for the proton 
      photoproduction cross section at lower energies.
    The perterbutive Quantum Chromo-dynamics (pQCD), Vector Messon Dominance 
      (VMD), and Leading Twist (LTA) methods all combined the nuclear photon 
      flux with the proton scattering data to calculate the nuclear 
      photoproduction cross section.
    Each of these methods handle the gluon density of the nucleus differently 
      producing a measurable difference in the value of the $J/\Psi$ 
      photoproduction cross section. 


  \section{QCD/QGP}


  \section{CGC/intial state}


  \section{Weizs\"{a}cker-Williams Approximation}
    The Weizsi\"{a}cker-Williams approximation relates the electric field of a 
      stationary point charge to the photon field that arises at ultra 
      relativistic velocities. 
    The approximation is semi-classical and combines both classical and quantum 
      elements.
    A Fourier transform of Maxwell's equations combine with Einstein's equation 
      for the energy of a photon in the Weizsi\"{a}cker-Williams approximation.
    \begin{wrapfigure}{r}{0.75\textwidth}
      \begin{center}
        \includegraphics{boost.png}
      \end{center}
      \caption{ \label{fig:boost} The electromagnetic field boosted and at rest. }
    \end{wrapfigure}
    The frequency modes of the electrostatic field are treated as photons. 
    The conversion of the electric field to a flux of photons simplifies the
      calculation of interaction cross sections. 
    The Weizsi\"{a}cker-Williams approximation makes the calculation of 
      electromagnetic interactions with the nucleus tractable. 

    The Wiezacker-Williams approximation begins with the equation for the 
      electric field of the projectile nucleus at rest. 
    The electromagnetic field only needs to be considered at the position of 
      the target nucleus. 
    From the projectile's point of view, the target is moving and contributes
     $-vt$ to Eq.~\ref{eq:staticEFromTargtmp}, the equation for the electric 
     field of the projectile nucleus at rest.
    \begin{equation} \label{eq:staticEFromTargtmp}
        x'=-vt'\qquad
        y'=b\qquad
        z'=0\qquad
	\vec{\mathbf{E'}}=\left(\frac{eZ}
         {4 \pi \epsilon_{0}\left(\left(-vt'\right)^{2}+b^{2}\right)^{3/2}}\right)
	 \left(-vt'{\mathbf{\hat{x'}}+b\mathbf{\hat{y'}}}\right)
    \end{equation}        
    In Eq.~\ref{eq:staticEFromTargtmp}, $b$ is the impact parameter, 
      the distance of separation at closest approach, $v$ is the velocity of the 
      projectile nucleus, $Z$ is the number of protons in the nucleus, and $e$ 
      is the charge of the electron.
    Two simplifications occur due to the coordinates of 
      Eq.~\ref{eq:staticEFromTargtmp}.
    The magnetic field is equal to zero, because the projectile is at rest, and
      the $z$ coordinate can be ignored, reducing the equation to two dimensions. 

    The Lorentz transformation converts the field equations in the 
      projectile's frame to equations in the target's frame.
    The result is a set of equations that relate the electric and magnetic field
      components in one frame to the components of the electric and magnetic 
      field in another frame moving at a different constant velocity.
    Eq.~\ref{eq:staticEFromTarg2tmp} gives the result of the transformation 
      from the projectile's primed frame to the target's rest frame for 
      the field components \cite{WWJackson}:
    \begin{eqnarray} \label{eq:staticEFromTarg2tmp}
        E_{x}'=E_{x}\qquad
        \gamma\left(E_{y}'/c+\beta B_{z}'\right)=E_{y}/c\qquad
        \gamma\left(E_{z}'/c=\beta B_{y}'\right)=E_{z}/c \nonumber \\
        B_{x}'=B_{x}\qquad
        \gamma\left(B_{y}'-\beta E_{z}'/c\right)=B_{y}\qquad
        \gamma\left(B_{z}'+\beta E_{y}'/c\right)=B_{z}
    \end{eqnarray}
    The transformation equations for the fields, 
      Eq.~\ref{eq:staticEFromTarg2tmp}, and the transformation of the 
      coordinates reduce to Eq.~\ref{eq:staticEFromTarg3tmp} \cite{WWJackson}:
    \begin{eqnarray} \label{eq:staticEFromTarg3tmp}
        E_{x}'=E_{x}\qquad
        \gamma E_{y}'=E_{y}\qquad
        \gamma \beta E_{y}'/c=B_{z}\nonumber \\
        ct'=\gamma ct \qquad
        x'=-\gamma \beta c t
    \end{eqnarray}
    The simplicity of Eq.~\ref{eq:staticEFromTargtmp} creates the simplicity of
      Eq.~\ref{eq:staticEFromTarg2tmp}.
    The Lorentz transformation reduces the six components of the 
      electromagnetic field in the target's frame to the three equations in 
      Eq.~\ref{eq:staticEFromTarg2tmp} by relating them to the fields of the 
      projectile's frame. 
    
    The combination of Eq.~\ref{eq:staticEFromTargtmp} and 
      Eq.~\ref{eq:staticEFromTarg2tmp} produce equations for the electric and 
      magnetic fields in the target's rest frame. 
    Eq.~\ref{eq:staticEFromTargtmp} gives the expression for the field 
      components as seen in the projectile frame. 
    \begin{eqnarray} 
	\vec{\mathbf{E}}=\left( \frac{\gamma e Z}
         { 4 \pi \epsilon_{0} \left( \left( \gamma v t \right)^{2} 
	 + b^{2}\right)^{3/2} }\right)
         \left(vt\mathbf{\hat{x}}+b\mathbf{\hat{y}}\right)\qquad\nonumber \\
	 \vec{\mathbf{B}}=\frac{\gamma\beta e Z b}
	 { 4 \pi c \epsilon_{0} \left( \left( \gamma v t \right)^{2} 
	 + b^{2}\right)^{3/2} }
         \mathbf{\hat{z}}=
	 \frac{\gamma\mu_{0}veZb}{4\pi\left(\left(\gamma v t \right)^{2}
	 +b^{2}\right)^{3/2}}\mathbf{\hat{z}}
    \end{eqnarray}
    If the impact parameter $b$ goes to zero, the target sits in the line of 
      the projectile particle's motion, and the denominator carries a factor of
      $\gamma$ squared. 
    If $vt$ goes to zero, the projectile particle is directly above or below in
      the $y$ direction, and the numerator carries a factor of $\gamma$. 
    This results in fields that are a factor of $\gamma^3$ higher in the 
      $y$ direction than in the $x$ direction (see Fig.~\ref{fig:boost}).  
    The boost compresses the electric field of the charge 
      in the direction of the boost and produces a magnetic field 
      resulting in a form similar to radiation.
    The point charge at ultra relativistic velocities produces a strong 
      electric field in the plane transverse to its motion resembling a plane 
      wave.
         
    Separating the even and odd functions of the electromagnetic field simplify
      the decomposition of the field equations into Fourier modes.
    The even functions decompose into cosine functions, odd functions 
      into sine functions. 
    The y-component of the electric field and the z-component of 
      the magnetic field are even functions in time, and the 
      x-component of the electric field is an odd function in time.
    Eq.~\ref{eq:fourier1tmp} gives the Fourier transformation integrals. 
    \begin{eqnarray} \label{eq:fourier1tmp}
        E_{x}(\omega)=\sqrt{\frac{2}{\pi}}\frac{eZ}{4\pi\epsilon_{0}b^{2}}
         \int^{\infty}_{0}\frac{\left(\gamma vt/b\right)\sin
	 \left(\omega t\right)}
	 {\left(\left(\gamma vt/b\right)^{2}+1\right)^{3/2}}dt\qquad
	E_{y}(\omega)=\sqrt{\frac{2}{\pi}}\frac{\gamma eZ}
	 {4\pi\epsilon_{0}b^{2}} \int^{\infty}_{0}\frac{\cos(\omega t)}
	 {\left(\left(\gamma vt/b\right)^{2}+1\right)^{3/2}}dt\nonumber \\
	B_{z}(\omega)=\frac{\beta E_{y}(\omega)}{c}\qquad
    \end{eqnarray}
    With the appropriate substitutions, tables provide 
      solutions to the integrals of Eq.~\ref{eq:fourier1tmp} as seen in 
      Ref. \cite{WWFermi}.
    \begin{eqnarray}  \label{eq:fourier2tmp}
        u=\frac{\gamma v t}{b}\qquad du\left(\frac{b}{\gamma v}\right)=dt\qquad
	 \omega'=\frac{\omega b}{\gamma v}\nonumber \\
	\int^{\infty}_{0}\frac{u \sin(\omega'u)}{\left(u^{2}+1\right)^{3/2}}du
	 =\omega'K_{0}(\omega')\qquad
	\int^{\infty}_{0}\frac{\cos(\omega'u)}{\left(u^{2}+1\right)^{3/2}}
	 =\omega'K_{1}(\omega')
    \end{eqnarray}
    The Fourier transformation replaces the time variable with a frequency 
      variable in the field equations. 
    The frequency relates to photon energy by the Einstein's photon energy  
      equation, $E=\hbar\omega$.
    The substitution of time with frequency allows for a flux of photons 
      to replace the classical electromagnetic field.

    The $\gamma$ dependence of the field components is different because of the
      different $t$ dependence of Eq.~\ref{eq:fourier2tmp}.
    The integrals in Eq.~\ref{eq:fourier2tmp} shift the $\gamma$ dependence of 
      the field component equations.
    Eq.~\ref{eq:fourier3tmp} gives the result of the integrals:
    \begin{equation} \label{eq:fourier3tmp}
        E_{x}(\omega)=\sqrt{\frac{2}{\pi}}\frac{eZ}{4\pi\epsilon_{0}b^{2}}
	 \frac{b}{\gamma v}\frac{\omega b}{\gamma v}K_{0}
	 \left(\frac{\omega b}{\gamma v}\right)\qquad
	E_{y}(\omega)=\sqrt{\frac{2}{\pi}}\frac{\gamma eZ}{4\pi\epsilon_{0}b^{2}}
	 \frac{b}{\gamma v}\frac{\omega b}{\gamma v}K_{1}
	 \left(\frac{\omega b}{\gamma v}\right)\qquad
    \end{equation}
    $\gamma$ is subsumed into the substitution from $t$ to $\omega$ in the 
      numerator of the x-component and becomes a part of the zeroth-order 
      modified Bessel function upon integration.
    The y-component does not have a factor of $t$ in the numerator, therefore 
      the factor of $\gamma$ remains outside of the integral, and it does not 
      get subsumed into the first-order modified Bessel function.
    \begin{wrapfigure}{r}{0.5\textwidth}
      \begin{center}
        \includegraphics{bess.png}
      \end{center}
      \caption{\label{fig:bess} The zero and first order modified Bessel functions. }
    \end{wrapfigure}
    In Eq.~\ref{eq:fourier3tmp}, $E_{y}$ carries an additional factor of 
      $\gamma$ in the numerator relative to the $E_{x}$.
    $E_{y}$ is $\gamma$ times larger then $E_{x}$.

    In the ultra-relativistic limit, the electric and magnetic fields have the 
      same configuration as electromagnetic plane wave radiation. 
    The electric and magnetic fields are perpendicular and related by a factor
      of $c$ in the ultra relativistic limit.
    When $v$ approaches $c$, $\beta \approx 1$, the y-component of the 
      electric field and the z-component of the magnetic field are related by 
      a factor of $c$, $E_{y}/c=B_{z}$.
    Because $K_{0}(x)$ is smaller than $K_{1}(x)$ for all x, when $\gamma >> 1$, 
      $E_{y}$ is approximately equally to $\gamma$ $E_{x}$. 
    The conditions imposed by the ultra-relativistic limit result in the 
      relationship of Eq.~\ref{eq:ultraRelAprox}.
    \begin{equation} \label{eq:ultraRelAprox}
      \gamma >> 1 \rightarrow \gamma E_{x}>>E_{x} \rightarrow E_{y} >> E_{x}
    \end{equation}
    The x-component of the electric field can therefore be ignored and 
      the magnetic and electric fields are left perpendicular to each other.
    The six field components reduced to one electric component and one 
      perpendicular magnetic field component, which have a configuration 
      identical to a plane wave. 

    As with plane waves, the energy per area per time transfered by 
      the electromagnetic field is given by the Poynting vector.
    The Poynting vector takes the simple form of a plane pulse propagating in 
     the x direction.
    \begin{equation} \label{eq:poyntingVectortmp}
        \vec{\mathbf{S}}\equiv
	\vec{\mathbf{E}}\times\vec{\mathbf{B}}/\mu_{0}=
	\left(E_{y}^{2}/c\mu_{0}\right)\mathbf{\hat{x}}=
	c\epsilon_{0}E_{y}^{2}\mathbf{\hat{x}}
    \end{equation}
    The Poynting vector relates to the fluence (energy per unit area) 
      \cite{WWBrau},
    \begin{equation} \label{eq:fluencytmp}
        I(b)=\mathbf{\hat{x}}\cdot\int^{\infty}_{0}\vec{\mathbf{S}}d\omega=
	 \int^{\infty}_{0}\left(c\epsilon_{0}E_{y}^{2}\right)d\omega=
	 \int^{\infty}_{0}\left(\frac{dI}{d\omega}\right)d\omega
    \end{equation}
      and the spectral fluence (energy per area per frequency).
    \begin{equation} \label{eq:specturalFluencytmp}
	\frac{dI}{d\omega}=c\epsilon_{0}E_{y}^{2}=
	 \frac{e^{2}Z^{2}c}{4\pi^{3}b^{2}v^{2}\epsilon_{0}}
	 \left(\frac{\omega b}{\gamma v}\right)^{2}
	 K_{1}^{2}\left(\frac{\omega b}{\gamma v}\right)=
	\alpha\hbar\left(\frac{Z}{b\beta\pi}\right)^{2}
	 \left(\frac{\omega b}{\gamma v}\right)^{2}
	 K_{1}^{2}\left(\frac{\omega b}{\gamma v}\right)
    \end{equation}
    Substituting Eq.~\ref{eq:fourier3tmp} into Eq.~\ref{eq:fluencytmp} gives 
      the Poynting vector as a function of frequency.
    Eq.~\ref{eq:specturalFluencytmp} paves the way for Einstein's 
      equation. 
    The spectral fluence given by Eq.~\ref{eq:specturalFluencytmp} 
      relates the frequency to energy, which are the same quantities 
      present in Einstein's equation. 

    Einstein's equation, $E=\hbar\omega$, gives the energy of a photon, which
      is related to the spectral fluence. 
    If the fluence is due to a photon number density, $N$, Einstein's equation
      relates $N$ to the fluence. 
    The relationship between the number of photons per unit area in an 
      infinitesimal energy range and the spectral fluence in an infinitesimal 
      frequency range is given by Eq.~\ref{eq:photonFluxtmp} \cite{WWJackson}.
    \begin{equation}  \label{eq:photonFluxtmp}
        \frac{dI}{d\omega}d\omega=\hbar\omega N(\omega)d(\hbar\omega)
         \rightarrow \frac{1}{\hbar^{2}\omega}\frac{dI}{d\omega}=N(\omega)
    \end{equation}
    Plugging Eq.~\ref{eq:specturalFluencytmp} into 
      Eq.~\ref{eq:photonFluxtmp} yields the semiclassical photon flux of an 
      ultra-relativistic nucleus.
    \begin{equation} \label{eq:photonFluxFinaltmp}
	N(\omega,b)=\frac{\alpha}{\hbar\omega}
	 \left(\frac{Z}{b\beta\pi}\right)^{2}
	 \left(\frac{\omega b}{\gamma v}\right)^{2}
	 K_{1}^{2}\left(\frac{\omega b}{\gamma v}\right)
    \end{equation}
    Eq.~\ref{eq:photonFluxFinaltmp} replaces the classical electric field of a 
      point charge with a semiclassical field of photons. 
    Physicists can calculate the electromagnetic interactions between nuclei 
      with the final result of the Weizs\"{a}cker-Williams approximation, 
      Eq.~\ref{eq:photonFluxFinaltmp}.
    The photon flux in Eq.~\ref{eq:photonFluxFinaltmp} provides the 
      electromagnetic input to the $J/\Psi$ photoproduction cross section 
      calculation. 


  \section{Vector Meson Dominance}
    The Vector Messon Dominace method for calculating the $J/\Psi$
      photoproduction cross section has three main components.
    VMD approach is constructed from the Weizsi\"{a}cker-Williams photon
      flux, the VMD fit to the proton-electron data, and the Glauber model for 
      calculating the nuclear cross sections from the proton-electron cross 
      sections.
    The Weizsi\"{a}cker-Williams photon flux provides the probe. 
    The proton-electron scattering data combine with the Glauber model  
      create a picture of the initial state of the nucleus. 
    Each of the different approaches to calculating the UPC $J/\Psi$ 
      photoproduction cross section use these same elements.
    However, the different models each use the last two elements differently 
      to produce different pictures of the nucleus and different cross 
      sections values. 

    The photon flux in the photoproduction cross section calculation must be 
      finite in order for the cross section to be meaningful.
    The Weizsi\"{a}cker-Williams approximation, Eq.~\ref{eq:photonFluxFinaltmp}, 
      produces a divergence at $b=0$.
    The probability of the nuclei interacting would exceed one if the photon 
      flux were infinite. 
    The divergence that arises at $b=0$ from $K_{1}$ results in an 
      unphysically infinite photon flux.
    Removing the divergence is necessary.
    Special treatment of impact parameter, $b$, where the colliding nuclei 
      overlap eliminates the divergence. 
    
    A modulation of the photon flux can subdue the divergence at $b=0$.
    A convolution of the photon flux with the nucleon number density functions 
      of the colliding nuclei produces the necessary modulation. 
    Eq.~\ref{eq:woodsSaxon} gives the nucleon density of a single nucleus,
    \begin{equation} 
      \rho_{A}(s)=\frac{\rho_{0}}{1+exp[(s-R_{WS})/d)]}
      \label{eq:woodsSaxon}
    \end{equation}
    In Eq. ~\ref{eq:woodsSaxon}, $s$ is the distance from the center of the 
      nucleus, $R_{WS}$ is the radius of the nucleus, and $d$ is the skin depth, 
      which determines how quickly the nucleon density falls off beyond the 
      nuclear radius. 
    In Eq.~\ref{eq:TaNuc} the depth of the nucleus is integrated out leaving
      just the transverse dimension in $T_{A}$.
    The average number of nucleons in the overlap region is given by a 
      convolution of $T_A$ from each of the two nuclei to produce $T_{AA}$.
    \begin{eqnarray} \label{eq:TaNuc}
      T_{A}(\vec{r})=\int{dz\rho_{A}(\sqrt{|\vec{r}|^{2}+z^{2}})} \nonumber \\ 
      T_{AA}(|\vec{b}|)=\int{d^{2}\vec{r}T_{A}(\vec{r})T_{A}(\vec{r}-\vec{b})}
    \end{eqnarray}
    $T_{AA}$ is the function that modulates the photon flux. 
    As input to the Poisson distribution, $T_{AA}$ reduces 
      Eq.~\ref{eq:photonFluxFinaltmp} at values of $b$ where the nuclei overlap 
      significantly and eliminates the divergence in the photon flux. 
    
    Modulating the photon flux by the probability that no nucleon-nucleon 
      collisions occur limits the photon flux at low $b$ in Eq.~\ref{eq:photonFluxFinaltmp}.
    The convolution of the photon flux with the $b$ dependent probability that 
      no nucleon-nucleon collisions occur removes the divergence in 
      Eq.~\ref{eq:photonFluxFinaltmp}. 
    Using the mean number of nucleons in the overlap region given by $T_{AA}$, 
      the Poisson distribution gives the probability that no collisions 
      occur at a given $b$:
    \begin{equation} \label{eg:poisNoCol}
      P_{0}(b)=exp[-T_{AA}(b)\sigma_{NN}]
    \end{equation}
    In Eq.~\ref{eg:poisNoCol}, $\sigma_{NN}$ is the cross section for a 
      nucleon-nucleon interaction, which gives the probability that a collision
      will occur given the average number of nucleons in the overlap region.
    The average photon flux over impact parameter, $b$, can be calculated 
      from the integration of the $b$-dependent photon flux, Eq.~\ref{eq:photonFluxFinaltmp}, 
      with the $b$-dependent probability of having no nucleon-nucleon 
      interactions, Eq.~\ref{eg:poisNoCol}. 
    \begin{equation} \label{eq:fluxMod}
      \frac{dN_{\gamma}(k)}{dk}=\int_{0}^{\infty}{2\pi bdbP_{0}(b)
         \int_{0}^{R}{\frac{rdr}{\pi R^{2}_{A}}\int_{0}^{2\pi}d\phi
         \frac{d^{3}N_{\gamma}(k,b+r\cos(\phi))}{dkd^{2}r}}}
    \end{equation}
    Eq.~\ref{eq:fluxMod} goes down to $b=0$ where the photon flux is infinite, but 
      because the probability of having a nucleon-nucleon collisions is high, 
      the divergence is eliminated.
    The result of Eq.~\ref{eq:fluxMod} does not diverge.  

    A power-law fit to the proton photoproduction data gives an analytic 
      expression for the energy dependence of the proton photoproduction 
      cross section.
    The fitting function is simple and only depends on the photon-proton center
      of mass energy, $W$. 
    Eq.~\ref{eq:proElFit} gives the parameterization of the forward 
      proton photoproduction cross section fit. 
    \begin{equation} \label{eq:proElFit}
      \frac{d\sigma(\gamma p\rightarrow Vp)}{dt}\Big|_{t=0}
        =b_{v}(XW^{\epsilon}+YW^{-\eta})
    \end{equation}
    $W$ is the center of mass energy of the proton-photon system in 
      Eq.~\ref{eq:proElFit}.
    The remaining variables in Eq.~\ref{eq:proElFit} are simple power-law fit
      parameters.  
    The $XW^{\epsilon}$ term characterizes pomeron mediated interactions, and
      the $YW^{\eta}$ term characterizes meson mediated interactions\cite{vmd1999}. 
    $J/\Psi$'s high mass relative to the $\pi$ and $\rho$ renders the second 
      term in Eq. ~\ref{eq:proElFit} negligible as the term falls rapidly with
      increasing $W$. 
    Eq.~\ref{eq:proElFit} allows for extrapolation and interpolation of the 
      measured forward proton photoproduction cross section. 
    The fit to the data provides estimates for energies that have not yet been 
      probed experimentally.

    The proton-electron scattering data is used differently in the VMD method 
      than in the other major methods.  
    The VMD method for calculating UPC photoproduction cross sections 
      relies more on electron-proton scattering data.
    The proton photoproduction cross sections from the electron-proton 
      scattering data is a direct input to the VMD model. 
    A power-law fit to the proton photoproduction data, as opposed to model 
      dependent gluon densities of other approaches, combines with the Glauber
      model to provide the nuclear model in the VMD method. 
    Because of the simplicity of the method, the VMD approach incorporates less
      modifications of the nuclear initial state relative to the proton initial
      state. 
    As a result, the VMD method produces a higher UPC J/$\Psi$ photoproduction 
      cross section relative to the other methods. 

    Vector meson dominance and the optical theorem allow for the calculation of 
      the total proton-meson scattering cross section from the fit given by 
      Eq.~\ref{eq:proElFit}. 
    The optical theorem relates a total cross section, $\sigma$, 
      to a corresponding forward scattering cross section, $d\sigma/dt|_{t=0}$.
    Vector meson dominance asserts that the colored part of the photon wave 
      function is dominated by vector mesons; therefore, the photon is 
      represented as a quark-antiquark pair in photoproduction calculations. 
    These two components combine to produce Eq.~\ref{eq:VMDforwardScat}.
    \begin{eqnarray} \label{eq:VMDforwardScat}
      \frac{d\sigma(\gamma p\rightarrow Vp)}{dt}\Big|_{t=0}=
        \frac{4\pi \alpha}{f^{2}_{v}(M_{V},\Gamma_{l^{+}l^{-}})}\frac{d\sigma(Vp\rightarrow Vp)}{dt}
        \Big|_{t=0} \nonumber \\
       \sigma(Vp)_{tot}^{2}=16\pi\frac{d\sigma(Vp\rightarrow Vp)}{dt}\Big|_{t=0}
    \end{eqnarray}
    In Eq.~\ref{eq:VMDforwardScat}, the photon-proton scattering is related to
      meson-proton scattering through the photon-meson coupling, which 
      depends on the vector meson's mass, $M_{V}$, and leptonic decay width, 
      $\Gamma_{l^{+}l^{-}}$. 
    The result of combining vector meson dominance and the optical theorem 
      in Eq.\ref{eq:VMDforwardScat} provides the cross section for a meson to 
      scatter off a proton.
    The total proton-meson scattering cross section, provides 
      the input to the Glauber model calculation of the nuclear photoproduction 
      cross section. 

    The nucleus-meson scattering cross section relates to 
      Eq.~\ref{eq:VMDforwardScat} through the Glauber model. 
    The Glauber model allows for Eq.~\ref{eq:VMDforwardScat}, the proton-meson
      scattering cross section, to be used to calculate a nucleus-meson 
      scattering cross section. 
    The Glauber model produces nuclear cross section calculations from 
      nucleon (proton or neutron) interaction cross sections by use of $T_{AA}$. 
    The combination of the mean number of nucleons in the overlapping region
      of a nucleus-nucleus collision, $T_{AA}$, the nucleon cross section, 
      $\sigma$, and the Poisson distribution make-up the core of the Glauber 
      model. 
    For the total nucleus-meson scattering cross section, the equation has the 
      following form:
    \begin{equation} \label{eq:glauber}
      \sigma_{tot}(VA)=\int d^{2}\vec{r}(1-e^{-\sigma_{tot}(Vp)T_{AA}(\vec{r})})
    \end{equation}
    In Eq.~\ref{eq:glauber}, the term $e^{\sigma_{tot}(Vp)T_{AA}}$ gives the
      probability of having no meson-nucleon scatterings from the Poisson 
      distribution. 
    The probability of having at least one scattering is given by subtracting 
      one from the term  $e^{\sigma_{tot}(Vp)T_{AA}}$ in Eq.~\ref{eq:glauber}.
    As seen in Eq.~\ref{eq:glauber}, the Glabuer model leverages scientific 
      knowledge of the proton to understand of the nucleus. 
    The Glauber model is the tool that combines the proton photoproduction data 
      with nucleon distributions in the nucleus to produce a nuclear vector 
      meson photoproduction cross section in the VMD approach. 

    Reversing the process used for the proton, Eq.~\ref{eq:glauber}, the meson 
      nucleus scattering cross section, relates to forward nuclear 
      photoproduction cross section through the optical theorem. 
    The nuclear photoproduction cross section is the input to the calculation 
      of the final result, the nuclear vector meson photoproduction cross 
      section in UPC events.  
    Eq.~\ref{eq:optTheA} uses the optical theorem to produce the nuclear 
      photoproduction cross section from the nucleus-meson scattering cross 
      section:
    \begin{eqnarray} \label{eq:optTheA}
      \frac{d\sigma(\gamma A\rightarrow VA)}{dt}\Big|_{t=0}=
      \frac{\alpha\sigma_{tot}^{2}(VA)}{4\pi f_{v}^{2}}\nonumber \\
      \sigma(\gamma A\rightarrow VA)=\frac{d\sigma(\gamma A\rightarrow VA)}{dt}
        \Big|_{t=0}\int_{t_{min}}^{\infty}dt|F(t)|^{2}
    \end{eqnarray}
    $F$ in equation Eq.~\ref{eq:optTheA} is the Fourier transform of the 
      nuclear density function, $\rho_{A}$.
    To produce the formula for calculating the UPC vector meson 
      photoproduction cross section, Eq.~\ref{eq:optTheA} is combined with the 
      photon flux incident on the nucleus, Eq.~\ref{eq:fluxMod}. 
    \begin{equation} \label{eq:finalVMDResult}
      \sigma(AA\rightarrow AAV)=2\int{dk\frac{dN_{\gamma}}{dk}
                    \sigma(\gamma A\rightarrow VA)}
    \end{equation}
    The factor of 2 in Eq.~\ref{eq:finalVMDResult} comes from the fact that both 
      of the two colliding nuclei contribute. 
    Combining the three elements of VMD, Eq.~\ref{eq:finalVMDResult} is the 
      final result of the VMD UPC photoproduction cross section calculation. 
    Vector meson production rates in UPC collisions are predicted 
      by Eq.~\ref{eq:finalVMDResult}, which can be confirmed or denied by 
      experiment. 


  \section{Leading Twist Approach Derivation}
    The LTA method for calculating UPC photoproduction cross sections
      combines elements of the Glauber model with direct use of gluon 
      densities. 
    The proton gluon density is modified by a nuclear modification 
      function in the LTA method to produce the nuclear gluon density. 
    The nuclear modification function converts the proton photoproduction
      cross section to a nuclear photoproduction cross section in the LTA 
      method.
    The LTA method is different from the other methods in its direct use of the
       nuclear modification factor and how the nuclear modification factor
       calculation incorporates multiple scattering.
    The direct use of the nuclear modification factor produces the most gluon 
      shadowing out of the three major methods, and results in the lowest
      cross sections.
    The LTA method is the easiest to constrain experimentally for this reason. 

    The LTA method uses the Weizsi\"{a}cker-Williams approximation to calculate 
      the photon flux created by the colliding nuclei. 
    As in the VMD method, the probability of having no hadronic collisions 
      modulates the flux.
    The photon flux for the LTA method has the following form \cite{lta2011.09}:
    \begin{eqnarray} \label{eq:ltaPhotonFlux}
      n_{\gamma/A}^{i}(\omega_{\gamma})=\frac{2\alpha Z^{2}}{\pi}\int_{b_{min}}^{\infty}
        db\frac{x^{2}}{b}\Big[K_{1}^{2}(x)+\frac{K_{0}^{2}(x)}{\gamma_{L}^{2}}\Big]
        P_{0}(b)P_{C}^{i}(b) \\
      x=\frac{\omega b}{\gamma_{L} v} \nonumber 
    \end{eqnarray}
    The $K_{0}^{2}(x)$ term contributes a photon flux in the 
      transverse direction.
    $P_{C}^{i}(b)$ is an additional modulation factor that requires various 
      additional interactions. 
    These interactions result in additional emissions of neutrons from the 
      receding nuclei as the nuclei relax from excited states. 
    The LTA flux reproduces the VMD result when the $K_{0}$ term becomes 
      negligible as $\gamma_{L}$ approaches $\infty$ and $P_{C}^{i}=1$ when all
      emissions are allowed.
    The terms $P_{C}^{i}$ and $K_{0}$ create additional ways to distinguish UPC
      events from nuclear collisions experimentally but leave the underlying 
      interaction mechanism the same. 
    For example, the additional terms in the LTA formulation of the photon flux
      produce calculations of asymmetric neutron emission, which separate UPC 
      events from nuclear collisions.  

    The LTA method calculates the nucleon photoproduction cross section 
      from the nucleon gluon density. 
    Ref. \cite{lta2011.09} derives the nucleon cross section from derivations
     of the nucleon gluon densities from electron-proton scattering data and
     leading order perturbative quantum field theory calculations.
    The forward photoproduction cross section of the nucleon has the following
     form \cite{lta2011.09}:
   \begin{equation} \label{eq:ltaFowardPhotoXSec}
     \frac{d\sigma_{\gamma N\rightarrow J/\Psi N}(t=0)}{dt}=\frac{16\Gamma_{l^{+}l^{-}}\pi^{3}}
     {3\alpha M_{J/\Psi}^{5}}[\alpha_{s}\mu^{2}xG_{N}(x,\mu^{2})]^{2}
   \end{equation}
   Here $G_{N}$ is the gluon density of the nucleon, $x$ is the fraction of
     the nucleon's momentum the gluon carries, and $\mu$ is related
     to momentum at which the nucleon is being probed, which is equal to 
     $M_{J/\Psi}/2$ for $J/\Psi$ photoproduction.
   In Eq.~\ref{eq:ltaFowardPhotoXSec} the nucleon cross section is explicitly 
     connected to the gluon density.
   By connecting the gluon density to the cross section, Eq.~\ref{eq:ltaFowardPhotoXSec}
     allows for the gluon density to be experimentally probed. 

   Ref. \cite{lta2011.09} exploits the optical theorem to relate the forward 
     photoproduction cross section of the nucleon to the nuclear cross section. 
   Eq.~\ref{eq:ltaOptTheWNucMo} gives the relation:
   \begin{eqnarray} \label{eq:ltaOptTheWNucMo}
     \sigma_{\gamma A\rightarrow J/\Psi A}(\omega)=
     \frac{d\sigma_{\gamma N\rightarrow J/\Psi N}}{dt}(\omega,t_{min})
     R_{g}^{2}\int_{t_{min}}^{\infty}dt|F(t)|^{2} \\
     R_{g}=\frac{G_{A}(x,\mu^{2})}{AG_{N}(x,\mu^{2})} \nonumber 
   \end{eqnarray} 
   $R_{g}$, the nuclear modification function, is the ratio between the gluon 
     density of the nucleon, $G_{N}$, to the gluon density of the nucleus, 
     $G_{A}$.
   As with the VMD method, the optical theorem relates the forward cross 
     section, $\frac{d\sigma_{\gamma N\rightarrow J/\Psi N}}{dt}(\omega,t_{min})$,
     to the total cross section, $\sigma_{\gamma A\rightarrow J/\Psi A}$. 
   The LTA method relates the measurable UPC photoproduction cross section 
     to the gluon density of the nucleus.  
   Eq.~\ref{eq:ltaOptTheWNucMo} further connects the gluon density of the 
     nucleon to the relative reduction of the gluon density in the nucleus 
     through $R_{g}$.

   From Eq.~\ref{eq:ltaOptTheWNucMo}, the LTA method can predict the angular 
     distribution of photoproduced $J/\Psi$ with respect to the beam axis. 
   In Ref. \cite{lta2012.03} the angular distribution is expressed in the form 
     of the rapidity dependency of the UPC photoproduction cross section. 
   \begin{eqnarray} \label{eq:ltaRapDist}
     \frac{d\sigma_{A_{1}A_{2}\rightarrow A_{1}A_{2}J/\Psi}}{dy}=
       n_{\gamma/A_{1}}(y)\sigma_{\gamma A_{2}\rightarrow J/\Psi A_{2}}(y)
       +n_{\gamma/A_{2}}(-y)\sigma_{\gamma A_{1}\rightarrow J/\Psi A_{1}}(-y) \\
       y=ln\Big(\frac{2\omega}{M_{J/\Psi}}\Big) \nonumber
   \end{eqnarray}
   Eq.~\ref{eq:ltaRapDist} is comprised of two terms, one for photons from the
     forward going nucleus interacting with the backward going nucleus, and 
     a second for the reverse situation. 
   The integration of Eq.~\ref{eq:ltaRapDist} over $y$ produces the factor of 2 
     that is present in Eq.~\ref{eq:finalVMDResult}.
   The rapidity distribution of the photoproduction cross section given in 
     Eq.~\ref{eq:ltaRapDist} provides a more detailed prediction and allows for
     more direct experimental comparison.
   Eq.~\ref{eq:ltaRapDist} allows for comparison to rapidity regions that are 
     covered by experiments.
  
   The LTA method is distinct from the pQCD method and VMD method through the 
     use $R_{g}$, the nuclear gluon modification factor. 
   As opposed to using $R_{g}$, the pQCD method uses the nuclear gluon 
     density, and VMD model uses proton photoproduction cross sections 
     directly. 
   In the LTA method, $R_{g}$ is calculated through a combination of $J/\Psi$ 
     photoproduction data from proton-electron scattering and DGLAP evolution 
     equations, which incorporates nuclear multiple scattering effects 
     \cite{lta2011.09}. 
   The DGLAP evolution equations give the depends of nuclear gluon densities on
     the momentum scale at which the nucleus is probed, $\mu$ in 
     Eq.~\ref{eq:ltaFowardPhotoXSec}. 
   The unique way the LTA method calculates $R_{g}$ results in lower cross 
     sections than the other major methods and allows for experimental
     sensitivity. 
   Experimental measurements of the UPC $J/\Psi$ photoprodcution cross section 
     with CMS have the opportunity to distinguish whether $R_{g}$ as calculated 
     in the LTA method accurately predicts the gluon density of the nucleus. 
  
 
  \section{Perturbative Quantum Chromo-dynamics}
    To calculate the UPC $J/\Psi$ photoproduction cross section, the pQCD 
      method uses the nuclear gluon density to characterize the nucleus and 
      the Weizs\"{a}cker-Williams approximation for the probing photon flux. 
    The pQCD method combines these components such that the nuclear gluon 
      density is a direct variable.
    The nuclear gluon density term in the pQCD formulation allows for 
      the use of a variety of nuclear gluon density models.
    A range of nuclear gluon densities are present in the available models
      resulting in a wide range of cross section values. 
    The UPC $J/\Psi$ photoproduction cross section is correlated with the gluon
      density of the nucleus rising with higher densities and shrinking with 
      lower densities. 
    In the pQCD approach, the calculation of the UPC $J/\Psi$ photoproduction 
      cross section allows experiments to constrain many different nuclear 
      gluon density models. 
    
    In the pQCD method, the photon interacts with the nucleus by fluctuating to 
      a quark-anitquark pair.
    For $J/\Psi$, the photon fluctuates to a $c\bar{c}$ pair. 
    The probability for the photon to fluctuate to a $c\bar{c}$ pair
      depends on the $M_{J/\Psi}$, the mass of $J/\Psi$, $\Gamma_{l^{+}l^{-}}$, 
      the $J/\Psi$ leptonic decay width, and $\alpha$, the electromagnetic 
      coupling constant.
    These three variables connect the c quark to the electromagnetic force 
      mediator, the photon. 
    Recast as a $c\bar{c}$ pair, the photon couples to the nuclear gluon 
      density.
    Ref. \cite{pQCD2011.08} uses the fluctuation of the photon to a $c\bar{c}$ 
      pair as the foundation for calculating the forward $J/\Psi$ 
      photoproduction cross section. 

    The $c\bar{c}$ pair arising from the photon fluctuation scatters off the 
      gluons of the nucleus. 
    The density of gluons in the nucleus determines how likely and therefore 
      how large the cross section is for the quarks to scatter and form a 
      $J/\Psi$.
    The forward scattering cross section is the portion of those scattering 
      events which transfer the minimum amount of momentum between the 
      photon and the nucleus. 
    The forward cross section for $J/\Psi$ photoproduction in the nucleus has 
      the following form \cite{pQCD2011.08}:
    \begin{equation} \label{eq:pQCDfowardScat}
      \frac{d\sigma_{\gamma A\rightarrow J/\Psi A}}{dt}\Big|_{t=0}=\xi_{J/\Psi}
        \Big(\frac{16\pi^{3}\alpha_{s}^{2}\Gamma_{l^{+}l^{-}}}{3\alpha 
        M_{J/\Psi}^{5}}\Big)[xG_{A}(x,\mu^{2})]^{2}
    \end{equation}
    In Eq.~\ref{eq:pQCDfowardScat}, $\xi_{J/\Psi}$ is an experimentally derived 
      correction factor, $\alpha_{s}$ is the strong coupling constant, $x$ is 
      the momentum faction of the nucleus the scattering gluons carry, and 
      $G_{A}$ is the gluon density of the nucleus. 
    Both the $c$ and $\bar{c}$ couple to the gluon density, and the double 
      coupling results in the squared dependence of the cross section on the 
      gluon density in Eq.~\ref{eq:pQCDfowardScat}. 
    Fitting Eq.~\ref{eq:pQCDfowardScat} to proton-electron scattering data 
      sets $\xi_{J/\Psi}$ \cite{pQCD2011.08}.
    The forward scattering cross section given by Eq.~\ref{eq:pQCDfowardScat} 
      connects the photon flux to the gluon density and provides the input to 
      calculate the total cross section by the optical theorem. 
    Eq.~\ref{eq:pQCDfowardScat} is the crux of how UPC measurements provide
      insight into the gluon content of the nucleus.
    
    The optical theorem relates the forward cross section in 
      Eq.~\ref{eq:pQCDfowardScat} to the total photoproduction cross section. 
    The total cross section calculated by use of the optical theorem gives 
      the probability that a photon incident on the nucleus will produce a 
      $J/\Psi$ regardless of the momentum transfered in the interaction. 
    Ref. \cite{pQCD2011.08} gives the form of the total cross section equation:
    \begin{equation} \label{eq:pQCDtotXSec}
      \sigma_{\gamma A\rightarrow J/\Psi A}(k)=
      \frac{d\sigma_{\gamma A\rightarrow J/\Psi A}}{dt}\Big|_{t=0}
      \int_{t_{min}(k)}^{\infty}dt|F(t)|^{2}
    \end{equation} 
    Here $t_{min}=(M_{J/\Psi}^{2}/4k\gamma_{L})^2$, which is the minimum amount
      of momentum transfer required to produce a $J/\Psi$ given the photon wave
      number $k$.
    The $k$ dependence of $t_{min}$ produces the rapidity, $y$, dependence of
      the total cross section.
    The total cross section for photoproduction, Eq.~\ref{eq:pQCDtotXSec}, 
      provides the input to Eq.~\ref{eq:ltaRapDist}, 
      which gives the rapidity dependence of the UPC photoproduction cross 
      section. 
    Eq.~\ref{eq:pQCDtotXSec} as input to Eq.~\ref{eq:ltaRapDist} allows for 
      experimental comparison of the pQCD method to measurements of UPC 
      photoproduction cross sections. 
    With the pQCD method's direct use of the nuclear gluon density in 
      Eq.~\ref{eq:pQCDfowardScat}, the pQCD method allows for experimental 
      exploration of any gluon density model. 


  \section{Incoherent Photoproduction}


  \section{Photon Induced Nuclear Break-up}


  \section{Theoretical Results}
  The UPC photoproduction cross section calculations depend significantly on 
    how the nucleus is represented in the calculation. 
  The results from the VMD, LTA, and pQCD methods vary from a relatively 
    large cross section in the VMD model, ranging through a variety of values
    in the pQCD method, to a relatively small cross section in the LTA method. 
  Each of these methods utilizes the same probe of the nucleus, the equivalent 
    photon flux that is calculated using the Weizsi\"{a}cker-Williams approximation. 
  The three methods deviate in how they calculate the forward photoproduction
    scattering cross section.
  The differences in the UPC photoproduction cross sections predicted by the 
    different models demonstrates the amount of experimental sensitivity there 
    is to distinguishing between the models. 
  The dependence of the cross section on rapidity shows where in phase space 
    a measurement of the cross section is most sensitive.
   
  The predicted value for the UPC $J/\Psi$ photoproduction cross section in 
    PbPb collisions at the LHC differ widely depending on which of the three 
    main methods is used. 
  The cross section value calculated by Eq.~\ref{eq:finalVMDResult} in the 
    VMD, LTA, and the various gluon density models in pQCD method vary 
    significantly.
  Table~\ref{tab:allXsec} gives the predicted values for the three main methods
    taken from Ref~\cite{pQCD2013.02}, Ref~\cite{lta2011.09}, and Ref~\cite{vmd1999}.
  \begin{table} 
   \centering
   \begin{tabular}{|l|l|} 
     \hline
     Model & $\sigma_{AA\rightarrow AAJ/\Psi} (mb)$ \\ \hline \hline
     VMD/STARlight MC & 23 \\ \hline
     LTA & 9 \\ \hline
     pQCD-MSTW08 & 34 \\ \hline
     pQCD-EPS08 & 7  \\ \hline
     pQCD-EPS09 & 14 \\ \hline
     pQCD-HKN07 & 23 \\ \hline
     \hline
   \end{tabular}
   \caption{$\sigma_{AA\rightarrow AAJ/\Psi} (mb)$
    the LTA, VMD, pQCD methods. Four different gluon density models are used 
    in the pQCD method. STARlight is a simulation software package that utilizes 
    the VMD model.}
   \label{tab:allXsec}
  \end{table}
  The cross sections in Table~\ref{tab:allXsec} differ by a factor of $\approx4$ 
    from the smallest to largest and create an experimental opportunity. 
  The clear discrepancy between the models in Table~\ref{tab:allXsec} 
    demonstrates the high amount of experimental sensitivity there is for 
    distinguishing between the models. 

  The rapidity dependence of the cross sections determine which values of 
    rapidity will be most sensitive to differences in the models. 
  The rapidity dependence calculated by Eq.~\ref{eq:ltaRapDist} overlap between 
    the models at certain values of $y$ leaving the models indistinguishable at
    that rapidity.
  Fig.~\ref{fig:rapDepAll} \cite{alice2012.09} shows the rapidity dependency of 
    the UPC $J/\Psi$ photoproduction cross section for the three main models 
    including several different gluon density models using the pQCD method.
  \begin{figure}[h] 
    \begin{center}
      \includegraphics[width=0.5\textwidth,keepaspectratio]{aliceDsigDy.png}
    \end{center}
    \caption{ \label{fig:rapDepAll} AB is the pQCD method, RSZ-LTA is the LTA method, and STARlight
      is the VMD model.}
  \end{figure}
  In Fig.~\ref{fig:rapDepAll} at higher rapidities, in particular $|y|>3$, the 
    various models give similar values for $d\sigma/dy$. 
  At $y=0$ the models vary the most. 
  Fig.~\ref{fig:rapDepAll} shows that experiments that can measure $J/\Psi$
    at $y=0$ have the best opportunity to distinguish between the models.
  The high sensitivity at $y=0$ creates an advantage for experiments that can 
    measure particles with small rapidity and low momentum.  
 
  The UPC photoproduction models each have different shapes to their rapidity 
    dependence. 
  The slope of $d\sigma/dy$ in Fig.~\ref{fig:rapDepAll} depends on the model. 
  Through the rapidity region $1<|y|<3$, each of the models has a progressively
    steeper slope. 
  The LTA method and the pQCD method utilizing the EPS08 gluon density model 
    are relatively flat where as the VMD and other gluon density models using
    the pQCD method have a noticeable slope.
  The differing slopes provide an additional experimental observable. 
  The shape of the rapidity distributions provide experimental sensitivity at 
    rapidities away from $y=0$ and creates an opportunity for experiments that 
    can not measure $J/\Psi$ at $y=0$.

  The nuclear suppression factor, S, demonstrates the difference between how 
    the models represent the nucleus. 
  S, which is a ratio between the nuclear photoproduction cross section and the     
    free nucleon photoproduction cross seciton, is a measure of how the nuclear 
    gluon densities evolve in each of the models. 
  Fig.~\ref{fig:ltaAndPqcdNucSub} from Ref.\cite{lta2013.05} shows the nuclear 
    suppression, which is equivalent to $R_g$ in Eq.~\ref{eq:ltaOptTheWNucMo}, 
    for the LTA and pQCD method.
  \begin{figure}[h] 
    \begin{center}
      \includegraphics[width=0.5\textwidth,keepaspectratio]{ltaAndPqcdNucSub.png}
    \end{center}
    \caption{ \label{fig:ltaAndPqcdNucSub} Nuclear supression factor, $S$, in the pQCD and LTA methods.}
  \end{figure}
  \begin{figure}
    \begin{center}
      \includegraphics[width=0.5\textwidth,keepaspectratio]{ltaAndPqcdNucSubVMD.png}
    \end{center}
    \caption{ \label{fig:ltaAndPqcdNucSubVMD} Nuclear supression factor, $S$, in VMD method.}
  \end{figure}
  Fig.~\ref{fig:ltaAndPqcdNucSubVMD} shows the nuclear suppression for the VMD 
    method \cite{lta2013.05}. 
  Fig.~\ref{fig:ltaAndPqcdNucSubVMD} and Fig.~\ref{fig:ltaAndPqcdNucSub} show that 
    as the momentum of the probing photon goes up, increasing $W_{\gamma p}$, 
    and momentum of the probed gluon goes down, decreasing $x$, 
    the nuclear gluon density decreases relative to the free nucleon. 
  The nuclear suppression factor, S, allows for the different models' 
    representations of the gluon content of the nucleus to be directly compared
    to each other and to data. 
  S can be measured from data by assuming a Weizsi\"{a}cker-Williams photon flux and 
    provides insight into nuclear gluon densities. 

\chapter{The CMS Detector}	
CMS is housed at interaction point 5 of the LHC. 
The LHC is designed to pursue physics at the TeV scale. 
This is the scale where electroweak symmetry breaking is believed to occur
	\cite{CmsPTdrv2}.
While this means that the search for the standard model Higs is the central 
	driving design consideration, the wide range of possibilities for
	finding new physics signals requires a general purpose detector.
The expedient discovery of new physics through low cross section interactions 
	requires high luminosity.
This consideration leads inevitably to pile up, where multiple collisions 
	occurs at a single bunch crossing.
At peak luminosity the LHC is expected to produce on average 20 hard 
	proton-proton (pp) collisions per bunch crossing \cite{tCmsE}.
These particle physics considerations of high multiplicity due to pileup and the
	need for a general purpose detector make CMS serendipitously well suited
	for heavy ion physics.
\begin{figure}[h]
  \centering
    \includegraphics[width=.5\textwidth]{cms}
  \caption{The Compact Muon Solenoid from Reference~\cite{tCmsE}.}
  \label{cms}
\end{figure}

The general purpose design of CMS is dominated by the massive 4T 
	superconducting solenoid at its core.
The magnets is 13m long with a 6m diameter, and pushes the limits of power
	and compactness \cite{tCmsE}. 
These two conflicting limits are achieved through the novel design of 
	interweaving structural and conducting elements together in the coil of
	the solenoid.

Within the solenoid resides three different sub detectors.
The inner most is the world's largest silicon tracker \cite{tCmsE}.
The tracker is surrounded by a highly effective lead tungstate crystal 
	electromagnetic calorimeter (ECAL).
ECAL is encapsulated in a brass scintillating hadronic calorimeter (HCAL).
Outside the magnet, muon chambers are used to aid in the measurement and 
	triggering of muon events. 
Altogether CMS weighs 12,500 metric tons, has a diameter of 14.6m,
	and a length of 21.6m \cite{tCmsE}.

The Silicon Tracker is the innermost sub-detector of CMS, and has active
	elements as close as 4.4cm to the interaction point \cite{tCmsE}. 
The tracker has a length 5.8m, a diameter of 2.6m and
	covers a range in pseudorapidity of \(|\eta| <\) 2.5.
Pseudorapidity is defined as $\eta\equiv-\ln(\tan(\theta/2))$, where $\theta$ is 
	the polar angle, and $\phi$ is the azimuthal angle with respect to the 
	beam axis.
At the center of the tracker are three rings of silicon pixels around the beam 
	with two disks of silicon pixels to cap the rings.
The pixel portion of the silicon tracker is comprised of 66x10$^{6}$
	pixels.
The silicon pixels are surrounded by silicon strips.
The silicon strips are separated into 4 different sections: 
	the Tracker Inner Barrel, the Tracker Inner Disk, the Tracker Outer 
	Barrel, and the Tracker End Caps.
The silicon strip detectors as a whole are comprised of 9.3x10$^{6}$ silicon 
	strips.
The high number of pixels and strips allow for the ability to distinguish
	and collect enough distinct points to reconstruct the path of the 1000
	or so charge particles per bunch crossing expected at peak luminosity
	\cite{tCmsE}.  

The next detector beyond the tracker is ECAL.
ECAL is made of 61,200 lead tungstate (PbWO$_{4}$) crystals in the central
	barrel and 7,324 on each of the two endcaps \cite{tCmsE}.
The barrel (EB) covers a pseudorapidity range $|\eta| < 1.479$ and has an
	approximate $\eta-\phi$ segmentation of $0.0174\times0.0174$.
Lead tungstate is very dense, which is reflect in the high number of interaction
	lengths the short depth of one crystal provides.
The crystals of the barrel have a depth of 230 mm corresponding to 25.8 
	radiation lengths ($X_{0}$).
The radiation length is the mean distance a high energy particle travels before
	giving up one e-fold of kinetic energy through electromagnetic
	interactions.
For example, after one radiation length $E \rightarrow E/e$, where 
	$e = 2.71828183$. 
The endcaps (EE) cover the psuedorapitity region $1.479 < |\eta| < 3$.
In the endcap the crystals have an exposed area of 28.62 $\times$ 28.62 
	mm$^{2}$, and a depth of 220 mm corresponding to 24.7 $X_{0}$.
The energy resolution of the ECAL as measured by test beam data can be seen in
	Figure~\ref{ECALeRes}.
\begin{figure}[h]
  \centering
    \includegraphics[width=0.5\textwidth]{ECALeRes}
  \caption{The energy resolution of ECAL as a function of energy from 
	Reference~\cite{tCmsE}.}
  \label{ECALeRes}
\end{figure}

The HCAL like the ECAL has both a barrel (HB) and endcaps (HE).
The pseudorapidity region $|\eta|<1.3$ is covered by HB \cite{tCmsE}. 
HB has an $\eta-\phi$ segmentation of $0.0897\times0.0897$, and is 25 times more
	sparsely granulated than EB.
HE covers the pseudorapidity region $1.3<|\eta|<3$.
HE, like EE and the tracker endcaps, is aligned perpendicular to the beam axis
	resulting in granularity that changes with $\eta$.
In the region $1.3 <|\eta|< 1.6$ HE has an $\eta-\phi$ segmentation of 
	$0.0897\times0.0897$.
The $\eta-\phi$ segmentation roughly doubles to $0.17\times0.17$ in the region
	$1.6 <|\eta|< 3$.
The energy resolution of the barrel and endcaps can be seen in  
	Figure~\ref{HCALeRes}.
The thickness of the hadronic calorimeter is best described in interaction
	lengths, the mean distance for a particle to give up an e-fold of energy
	through nuclear interactions. 
At $\eta = 0$ the barrel has a thickness 5.82 interaction lengths 
	($\lambda_{I}$), and increases as the path length through the material 
	increases to 10.6 $\lambda_{I}$ at $|\eta| = 1.3$.
\begin{figure}[h]
  \centering
    \includegraphics[width=0.5\textwidth]{HCALeRes}
  \caption{The $E_{T}$ resolution of HCAL as a function of $|\eta|$ and $E_{T}$
	from Reference~\cite{tCmsE}.}
  \label{HCALeRes}
\end{figure}

In addition to HB and HE, HCAL has two additional calorimeters.
Because the space between ECAL and the magnet is restricted to 1.18 m, an
	outer hadronic calorimeter section (HO) is placed beyond the magnet
	in the region $|\eta|<1.3$ \cite{tCmsE}.
The main function of HO is to collect energy from the highest energy hadrons
	before they reach the muon system.
HO is not used in this analysis, but does contribute to the material budget. 
To increase the total calorimetric coverage, HCAL also has a quartz fiber 
	calorimeter (HF) in the forward region, $3 < |\eta| < 5$.
For the majority of HF's 13 $\eta$ rings the $\eta-\phi$ segmentation is 
	$0.175\times0.175$.
In the lowest $|\eta|$ ring the segmentation is $0.111\times0.175$ in 
	$\eta-\phi$.
In the highest two $|\eta|$ rings the segmentation in $\phi$ is 0.349, with an
	$\eta$ segmentation of 0.175 in the outer and 0.300 in the innermost 
	ring. 
The longitudinal direction is effectively segmented by using short fibers and
	long fibers.
The measure energy deposited deeper than 22 cm is measured in both the short
	and long fibers, where as the long fibers are present throughout.
This allows electromagnetic showers to be distinguished from purely hadronic 
	showers \cite{tCmsE}.
The energy resolution for HF can be seen in Figure~\ref{HCALeRes}.  

Beyond HF there are two more detectors in the forward region.
CASTOR covers the range 5.2 $< \eta <$ 6.6 on the positive side of the beam. 
The Zero Degree Calorimeters (ZDC) sit between the beam pipes on either side of
	the interaction point covering the area around $\theta = 0$, $|\eta| > 
	8.3$.
In heavy ion collisions the ZDC has the ability to measure neutral particles 
	that do not participate in the collision \cite{tCmsE}.
CASTOR extends the total coverage of the CMS as whole giving more access to 
	low-x physics \cite{tCmsE}.

The ZDC has a total of 18 channels.
Half of these 18 channels are on either side of the interaction point.
The 9 channels on the side of CMS that correspond to positive $\eta$
  are denoted ZDC$^{+}$, where as the 9 channels on the negative side are
  denoted ZDC$^{-}$.
The 9 channels on each side are further sub-divided into an electro-magnetic  
  (EM) section and a hadronic (HAD) section.
The EM section is positioned in front of the HAD section with respect to the 
  interaction point and is segmented transverse to the beam direction.
The 5 EM sections are positioned in front to absorb the energy from 
  electro-magnetically induced showers, which develop over a shorter distance 
  than hadronically induced showers.
The transverse segmentation allows for a measurement of the transverse shower
  width and the size of the beam spot at the ZDC.
The HAD section is segmented in the direction of the beam and consists of 4
  channels.
The longitudinal segmentation allows for absorption of the full extended 
  hadronic shower and the ability to measure the longitudinal shower shape.

Each the 18 channels contains a tungsten target and quartz fibers.
The dense tungsten target is used to initiate the shower.
The quartz fibers shine Cerenkov light as the high momentum charged particles
  from the shower pass through it. 
the light from the quartz fibers is channeled to photo-multiplier tubes, one 
  for each ZDC channel. 
Through a cascade of photon induced electrical discharges, the photo-multiplier
  converts the Cerenkov light to an electrical pulse. 

This electrical pulse travels $\sim$ 200 m down a coaxial cable from the LHC
  tunnel to the counting house in the CMS service cavern. 
There the electrical pulse is digitized by the Charge Integrator and Encoder 
  (QIE).
The QIE integrates the current each 25 nano seconds.
The charge is than mapped logarithmically to the 128 bits. 
This bit is sent across a small fiber optic cable to the HTR firmware card.
Here each 25 ns signal is stored in a 250 ns buffer, and the timing is sync
  with the rest of the detector to insure the ZDC signal arrives at the central
  data acquisition system at the same time as the other sub detectors from the 
  same collision. 

The muon system resides just outside of the superconducting magnet.
It consists of three complementary systems: drift tube (DT) chambers in the
	barrel, cathode strip chambers (CSC) in the endcaps, and resistive 
	plate chambers (RPC) in both the barrel and endcap regions \cite{tCmsE}.
Ultimately the muon system is most useful for triggering on muons \cite{tCmsE}.

The heavy ion community is making use of the capabilities of CMS in a myriad of
	ways.
The muon trigger has been used in the search for suppression of quarkonium 
	states. 
This is an important probe of the correlation length within the hot dense state
	known as the quark gluon plasma (QGP).
The tracker has been utilized for to study charged particle multiplicities, and
	and elliptical flow, two probes of the thermal expansion of the QGP.
HCAL has aided in measuring jet suppression, which probes the strength with 
	which the QGP interacts with strong interacting objects.
Through its general purpose design and its ability to handle the high
	multiplicities produce by the LHC, CMS proves to be an excellent 
	detector for investigating strongly interacting mater through heavy ion
	collisions. 
%  \section{CMS general}
%  \section{Muons}
%  \section{HCal}
%  \section{ZDC}
  \section{Trigger}
    The CMS trigger is two teired. 
    The L1 trigger is the lower level hardward based system. 
    The High Level Trigger (HLT) is software base and runs on a computer farm
      at point 5 where CMS is housed. 

\chapter{\label{ch:zdcReco}ZDC reconstruction}
  \section{\label{sec:breakUpDet} Nuclear break-up determination}
    As described in Section~\ref{sec:ltaTheory}, UPC \JPsi{} photoproduction 
      can be accompanied by the emission of neutrons from either of the two 
      colliding nuclei.
    The various neutron emission scenarios, or break-up modes, can 
      be distinguished by the two ZDCs.
    By separating events where the ZDC signal is consistent with 1 neutron 
      versus several neutrons, or where neutrons are present on only one or 
      both sides, the fraction of events which corresponds to a given 
      break-up mode can be measured and compared to theory. 

    In order to maximize the ability to explore the one neutron peak, which 
      sits at the bottom of the ZDCs dynamic range, a new ZDC reconstruction 
      method was devised. 
    This new reconstruction method was then used to establish 
      thresholds for one and more than one neutron in each ZDC. 
    This section describes the ZDC signal reconstruction and how the neutron 
      thresholds on this signal were set.
    
    \subsection{ZDC signal reconstruction}
      The signal from each ZDC is built up from the pulse shapes for each of 
        the 18 individual ZDC channels. 
      The pulse shape is recorded in 250 ns second chunks and is divided into
        10 time slices of 25 ns (see Fig~\ref{fig:zdcPulseShape}).
      Counting from 0, the 4th time slice is synced with the timing of the rest
        of the detector and corresponds to when the products of the recorded 
        collision reached the ZDC.
      The signal is therefore taken from the 4th time slice.
      \begin{figure}[h]
        \centering
        \includegraphics[width=\textwidth]{zdcPulseShape}
        \caption{Average ZDC pluse shape is plotted as the charge as a function
          of time slice for the first hadronic from ZDC$^{-}$ (left) and 
          ZDC$^{+}$ (right).}
        \label{fig:zdcPulseShape}
      \end{figure}

      The ZDC signal sits on top of a low frequency noise pedestal with a 
        period of about 2$\mu$ seconds. 
      Over the time scale of 250 ns, this low frequency noise signal appears
        as a constant that shifts randomly from event to event.
      The contribution from this noise is therefore measured event by event
        in order to subtract it.
      Time slice 5 is used for this purpose.
      Time slices 1 and 2 could also be used to estimate the low frequency 
        noise.
      However because the noise fluctuates to negative values of charge that 
        cannot be measured, these time slices can only provide a 
        measurement of the noise half the time. 
      By using time slice 5 which contains the falling tail of the signal, 
        the noise can be measured any time the signal raises significantly 
        above the noise.
      If the fraction of signal in time slice 4 and 5 are constant and
        the noise contributes the same value to both time slices, the 
        following formula is applicable:
      \begin{equation}
        Ts4 \propto (Ts4 + C) - ( Ts5 + C ) = Ts4 - R_{Ts5/Ts4}Ts4 
        = Ts4(1-R_{Ts5/Ts4}),
        \label{eq:ts4ish}
      \end{equation}
      where $Ts4$ is the signal contribution in time slice 4, $Ts5$ is the 
        signal contribution to time slice 5, $C$ is a random noise constant
        from the low frequency noise, and $R_{Ts5/Ts4}$ is the ratio between
        the signal contribution from time slice 5 over time slice 4.
      Figure~\ref{fig:zdcTs4OvTs5VTs5} demonstrates the consistency of the 
        fraction and validates the unconventional method of using the falling 
        tail of the signal to estimate the low frequency noise. 
      By using time slice 5, the chances of measuring the noise are maximized. 
      Separating the signal from the noise is especially important because
        the ZDC signal for the one neutron peak sits near the noise at the 
        bottom of the ZDC dynamic range.
      \begin{figure}[!Hhbt]
        \centering
        $ \begin{array}{cc}
          \includegraphics[width=.4\textwidth]{negTs5overTs4vts5} &
          \includegraphics[width=.4\textwidth]{posTs5overTs4vts5}
        \end{array} $  
        \caption{ The fraction of signal in time slice 5 over time slice 4 
          as a function of the signal in time slice 5 in ZDC$^{-}$ (left) and 
          ZDC$^{+}$ (right).}
        \label{fig:zdcTs4OvTs5VTs5}
      \end{figure}
      
      When summing the 9 channels in each ZDC, only channels with signals above 
        zero in time slices 4 and 5 were included. 
      The EM, electromagnetic, section of the calorimeter is more densely 
        packed with quartz fibers and therefore has a higher gain relative to 
        the HAD, hadronic, section. 
      To account for this, the EM channels were weighted with
        a factor of 0.1 to match the HAD channel gains.

    \subsection{Determination of the one neutron thresholds}
      The ZDC thresholds used to establish the various break-up modes were 
        measured from zero bias data.
      Figure~\ref{fig:zdcM2Fit} shows the weighted sum of the EM and 
        HAD sections for  ZDC$^{-}$ and  ZDC$^{+}$ for the zero bias 
        dataset.
      The neutron spectrum for this dataset is not biased since the 
        trigger only required that both beams were present in CMS. 
      This dataset does, however, include a significant electronic noise contribution due
        to events where no neutrons are emitted in the direction of the ZDC.
      It is clear from Fig.~\ref{fig:zdcM2Fit} that the gain of  
        ZDC$^{+}$ is lower than that of ZDC$^{-}$. 
      This is because of a damaged phototube on the first HAD section 
        of ZDC$^{+}$.

      To determine the thresholds for one and multiple neutrons, the ZDC$^{+}$ 
        and ZDC$^{-}$ spectra were 
        each fit to the sum of four Gaussians. 
      The electronic pedestal noise was fit to a Gaussian around zero.
      The one, two, and three neutron peaks are fit to Gaussians that are 
        successively broader.
      The mean of each peak was initially set to multiples of the mean of the 
        one neutron peak. 
      \begin{figure}[!Hh]
        \centering
        $
          \begin{array}{cc}
            \includegraphics[width=0.45\textwidth]{zdcFit45Neg.pdf} &
            \includegraphics[width=0.45\textwidth]{zdcFit45Pos.pdf}
          \end{array} 
        $
        \caption{Fit to the signal spectra for ZDC$^{-}$ (left) and ZDC$^{+}$ 
          (right)}
        \label{fig:zdcM2Fit}
      \end{figure}
      The threshold for a neutron in the ZDC was taken from the fits in 
        Fig.~\ref{fig:zdcM2Fit}.
      Any signal greater 2$\sigma$ below the mean of the one neutron peak was 
        considered signal.
      Any signal greater than 2$\sigma$ above was considered multiple 
        neutrons.
      The single neutron break up modes were separated from the multiple 
        neutron modes by use of these definitions.

      Several of the break-up mode calculations that have been done involve
        single sided configurations where neutrons are present on one side
        of the interaction point and not the other.
      These modes can be hard to identify because the single neutron peak in 
        ZDC$^{+}$ overlaps with the noise peak at zero.
      To identify events where the ZDCs only measured noise, the noise
        spectra were measured directly.
      Placing an additional criteria based on the ZDCs noise distributions for
        when the ZDCs are devoid of signal provides assurance that the events 
        tagged as single sided events are truly single sided.

      
      The noise distributions for the EM sections and the HAD sections were
        measured separately from out of time time slices.
      In Fig.~\ref{fig:zdcPulseShape} higher than average signal can be seen
        in the 0th time slice, which precedes the main signal time slice 
        time slice 4 by 200 ns. 
      This is due to events where activity was present in the ZDC for 
        two consecutive collisions.
      Time slices 1 and 2, however, occurred between collisions.
      These time slices, which occur out of time, were used to measure the 
        noise spectrum.
      \begin{figure}[!Hhbt]
        \centering
        $ \begin{array}{cc}
          \includegraphics[width=.45\textwidth]{zdcNegEMNoiseFromZBNoCor} & 
          \includegraphics[width=.45\textwidth]{zdcPosEMNoiseFromZBNoCor} \\
          \includegraphics[width=.45\textwidth]{zdcNegHDNoiseFromZB} &
          \includegraphics[width=.45\textwidth]{zdcPosHDNoiseFromZB}
        \end{array} $
        \caption{ZDC noise spectra from ZDC$^{-}$ EM section (upper left), 
          ZDC$^{+}$ EM section (upper right), ZDC$^{-}$ HAD section (lower left), 
          and ZDC$^{+}$ HAD section (lower right) from out of time time slices.}
        \label{fig:zdcNoiseSpectra}
      \end{figure}

      As with the signal measurements, the low frequency noise pedestal is 
        subtracted event by event by subtracting time slice 2 from time slice
        1 leaving only the high frequency noise.
      The noise distributions do not depend on the amount of quartz fibers, but
        because the signal does, the noise distributions for EM and HAD sections
        are measured separately.
      Figure~\ref{fig:zdcNoiseSpectra} shows the noise spectrum for each of the 
        EM and HAD sections for the two ZDCs.
      If the HAD or EM signals measured from time slices which match the 
        timing for a collision, time slices 4 and 5, are less than 2$\sigma$ 
        above the mean of the noise distribution or lower, these sections are 
        considered consistent with noise.
      A ZDC is considered consistent with noise if both the HAD section and EM 
        section from that ZDC have signal measurements consistent with noise.

    \subsection{\label{sec:zdcCompare}ZDC reconstruction method comparison}
      In this section the nominal ZDC reconstruction method designed for this
        thesis is compared to an alternative method.
      This additional method, used in previous ZDC measurements, differs 
        in the way the signal time slices are used to calculate the signal from
        each channel.
      In the additional alternative method, the signal is taken from the sum of 
        time slices 4, 5, and 6.
      To estimate the event by event noise pedestal the sum of time slice 
        1 and 2 are used. 
      The signal for an individual ZDC channel is then calculated as the 
        sum of the signal time slices minus the sum of the noise time slices
        weighted by a factor of 3/2 to account for the differing number of 
        noise versus signal time slices.
      As in the nominal method described in Section~\ref{sec:breakUpDet}, 
        the alternative method combines the channels to create a signal 
        measurement from the whole of each side of the ZDC, one
        measurement for ZDC$^{+}$, and one for ZDC$^{-}$.
      The noise subtracted signal from each of the HAD channels are added 
        together.
      Then the EM section channels are summed. 
      The EM section is weighted by a factor of 0.1 as in the nominal method. 
      After the weighting the EM and HAD channels are added to each to create
        one measurement for ZDC$^{+}$ and another measurement for ZDC$^{-}$.
      Figure~\ref{fig:zdcM1Fit} shows the spectra for ZDC$^{+}$ and ZDC${-}$ 
        using the alternative method. 
      The same fit used for the nominal method is applied to the alternative 
        method. 
      As in the nominal method, the single neutron threshold is set to 2$\sigma$
        below the mean from the fit to the one neutron peak.
      The multi-neutron threshold was set to 2$\sigma$ above the one neutron
        peak.
      \begin{figure}[!Hhtb]
        \centering
        $ \begin{array}{cc}
          \includegraphics[width=0.45\textwidth]{zdcMinusZBFitTimeCut} &
          \includegraphics[width=0.45\textwidth]{zdcPlusZBFitTimeCut}
        \end{array} $
        \caption{Fit to charge spectrum from ZDC$^{-}$ (left) and ZDC$^{+}$ 
          (right) using the alternative reconstruction method}
        \label{fig:zdcM1Fit}
      \end{figure}

      The advantage of the alternative method is that by using multiple signal
        and noise time slices the signal and noise are effectively averaged
        reducing time slice to time slice fluctuations.
      However, by using time slices 1 and 2 for measuring the noise, the noise
        can only be measured half the time due to unmeasurable negative 
        fluctuations of the dominant low frequency component of the noise.
      The nominal method relative to the alternative method separates low signal
        from the noise more effectively  for both sides of the ZDC.
      This is particularly important for ZDC$^{+}$ where the 1st HAD section
        had a lower gain than the other sections. 
      The ZDC$^{+}$ and ZDC$^{-}$ signals near the one neutron peak using the
        alternative and nominal reconstruction methods were plotted for comparison in 
        Fig.~\ref{fig:zdcSpec2v1}.
      \begin{figure*}[!Hhbt]
        \centering
        \includegraphics[width=.45\textwidth]{zdcSpec2v1Minus}
        \includegraphics[width=.45\textwidth]{zdcSpec2v1Plus}
        \caption{Comparison of the nominal (red) ZDC reconstruction 
          method and the alternative (blue) method for ZDC$^{-}$ (left) and 
          ZDC$^{+}$ (right).}
        \label{fig:zdcSpec2v1}
      \end{figure*}
      In Fig.~\ref{fig:zdcSpec2v1}, the shrinking of width of the noise peak 
        around zero in the nominal method versus the old method is apparent for
        both ZDC$^{+}$ and ZDC$^{-}$.
      For the alternative method no single neutron peak is resolved in ZDC$^{+}$,
        whereas the single neutron peak is resolved using the nominal method. 

      Timing cuts were applied to enhance the signal relative to the background
        in order to resolve the one neutron peak in ZDC$^{+}$ using the 
        alternative method. 
      Because the products of the collision are synced with time slice 4, noise
        can be rejected by selecting channels where the maximum signal falls 
        into time slice 4.
      The noise will have no preferred time slice (see Fig.~\ref{fig:zdcPulseShape}). 
      Using this fact, neutron events can be selected by requiring that the
        hadronic channels of the ZDC have a peak signal in the fourth time 
        slice.
      Through these timing cuts the single neutron peak was recovered using the
       alternative reconstruction for ZDC$^{+}$.

      To examine the effectiveness of the timing cuts, event by event noise 
        subtraction was removed from the alternative reconstruction.
      The signal from each channel was taken from time slices 4,5, and 6 with
        out subtracting 1 and 2.
      The signal spectrum from ZDC$^{-}$ was then plotted with the result
        shown in Fig.~\ref{fig:zdcTimingCuts}.
      \begin{figure}[!Hhbt]
        \centering
        \includegraphics[width=0.6\textwidth]{zdcMinusSingleNuNoInc}
        \caption{Effects of requiring in-time signal in successively more 
          ZDC$^{-}$ hadronic channels, no timing, at least one (red), at least two (green),
            at least three (blue), and all four (yelow) HAD channels have a maximum signal
            in the fourth time slice.}
        \label{fig:zdcTimingCuts}
      \end{figure}
      As each additional hadronic channel is required to have a maximum signal
        in the fourth time slice, the single neutron peak emerges. 
      Figure~\ref{fig:zdcTimingCuts} demonstrates that the single neutron peak 
        can be recovered from the noise using timing cuts alone. 

      Using the alternative noise subtraction method, the same signal that emerges
        from the timing cuts alone appear without timing cuts.
       \begin{figure}[h]
        \centering
        \includegraphics[width=0.6\textwidth]{zdcMinusSingleNuNoSub}
        \caption{Effect of ZDC signal timing requirements after noise 
          subtraction.}
        \label{fig:zdcTimingAfterNoiseSub}
      \end{figure}
      Figure~\ref{fig:zdcTimingAfterNoiseSub} confirms that both noise 
        subtraction and the timing requirement produce the same signal.
      This gives confidence that the signal is not an artifact of either cut, 
        but the true neutron signal.

      Figure~\ref{fig:zdcTimingAfterNoiseSub} and Fig.~\ref{fig:zdcTimingCuts} 
        demonstrate the consistency of using timing cuts and noise 
        subtraction to enhance the signal neutron peak. 
      Figure~\ref{fig:zdcTimingAfterNoiseSub} confirms the legitimacy of the 
        timing requirement method in ZDC$^{-}$ by showing that the same
        signal emerges from the noise subtraction method as the timing method.
      Fig.~\ref{fig:zdcSpec2v1} demonstrates the correspondence between
        the nominal noise subtraction method and the alternative method in 
        ZDC$^{-}$ where the signal is better separated from the electronic noise. 
      This provides confidence that the signal seen in ZDC$^{+}$ using 
        the nominal method is the one neutron peak.

\chapter{\label{ch:trigg}UPC trigger development for CMS}
  At the collision energy of the 2011 LHC PbPb run, the cross section of UPC 
    J/$\psi$ production was about 10$^{5}$ times smaller than that of ordinary 
    nucleus-nucleus collisions.
  To recorded these events dedicated UPC triggers were needed. 
  Unlike most heavy-ion triggers, UPC triggers are optimized for low-\pt{} 
    and low multiplicity events. 
  For this reason trigger development specific to UPC events was required to
    carry out the analysis in this thesis.

  The increase in collision rate of the LHC PbPb beams from 2010 to 2011 was
    nearly a factor of 15. 
  To accommodate this increase in rate, the 2011 trigger scheme needed to be 
    more selective than in 2010 where CMS could take any event which 
    appeared to have a collision.
  The available bandwidth was allocated equally amongst the various heavy ion
    analysis groups to pursue as wide a physics program as possible.
  From this consideration, bandwidth limits were placed on the trigger rates
    for each analysis group's trigger package. 
  
  \section{\label{sec:l1Trigger}L1 trigger}
    The UPC L1 triggers were designed to study UPC \JPsi{} production via the 
      dimuon and dielectron channels (see Section~\ref{sec:detTrg}).
    To achieve this, the loosest muon and electron triggers where combined with
      a trigger on energy in the ZDCs and no activity in BSCs (BSC veto).
    Additional triggers were commissioned in case radiation damage during the 
      run reduced the sensitivity of the BSCs.
    This required no activity in HF (HF veto). 
    These triggers are summarized in Table~\ref{tab:l1Triggers2011}.
    The ECAL2 and ECAL5 triggers in Table~\ref{tab:l1Triggers2011}
      indicate a 2 and 5 GeV threshold on $E_{T}$ measured in the ECAL.
    The MuonOpen trigger indicates that the trigger only 
      requires a muon candidate in one of the three muon sub-systems and that
      there is no momentum threshold.
    ZDC in the trigger names indicate energy constant with at least one neutron.
    The sign on the ZDC label indicates which of the two ZDCs is required. 

    \begin{table}[h]
      \centering
      \begin{tabular}{|l|l|l|l|l|}
        \hline L1 trigger name & Rate (Hz) & Prescale & Id & Type \\ \hline \hline
        MuonOpen and (ZDC$^{+}$~or~ZDC$^{-}$) and BSC veto & 2.1 & 1 & 1 & \multirow{3}{*}{Physics} \\  \hhline{----~}
        ECAL2 and (ZDC$^{+}$~or~ZDC$^{-}$) and BSC veto & 1.8 & 2 & 2 & \\  \hhline{----~}
        ECAL5 and (ZDC$^{+}$~or~ZDC$^{-}$) and BSC veto & 0.3 & 1 & 3 & \\  \hline
        (ZDC$^{+}$~or~ZDC$^{-}$) & 35 & 1500 & 4 & Monitor \\  \hline
        MuonOpen and (ZDC$^{+}$~or~ZDC$^{-}$) and HF veto & 0 & off & 5 & \multirow{3}{*}{Backup} \\ \hhline{----~}
        ECAL2 and (ZDC$^{+}$~or~ZDC$^{-}$) and HF veto & 0 & off & 6 & \\  \hhline{----~}
        ECAL5 and (ZDC$^{+}$~or~ZDC$^{-}$) and HF veto & 0 & off & 7 & \\  \hline
      \end{tabular}
      \caption{List of 2011 L1 seeds.}
      \label{tab:l1Triggers2011}
    \end{table}
    The cumulative L1 trigger rate for all the UPC L1 trigger seeds was
      required to be no greater than 200 Hz.
    This requirement came from the need to keep the tracker read-out rate
      low. 
    The trackers baseline voltage can fluctuate due to the high tracker hit 
      multiplicities in PbPb collisions.
    In order to monitor the zero suppression of the tracker, the zero 
      suppression algorithm was executed using the HLT computing farm 
	      rather than in the tracker firmware.

    The (ZDC$^{+}$ or ZDC$^{-}$) was used to estimate the efficiency of the UPC electron and muon triggers. 
    The factor that the rate is reduced by is called the prescale.
    A prescale of 2 for example means that half the triggers were 
      accepted.
    If the prescale is set to 1, then whole trigger rate is accepted. 
    The prescale factors for the triggers were set to balance the competing objectives 
      of rate reduction and increasing the overlap between the monitoring and
      signal triggers.

  \section{\label{sec:hltTrigger}HLT trigger}
    An event must pass the selection criteria of an HLT path in order to be
      recorded. 
    As opposed to the L1 trigger, which has access only to information from
      calorimeters and muon chambers, the HLT has access to all of the CMS 
      sub-detectors including the tracker. 
    Reconstruction of a track in the pixel detector is used by the UPC 
      trigger paths.
    The use of the pixel detector only, as opposed to using the whole tracker 
      including the silicon strip detector, allows for quick track 
      reconstruction saving computing cycles.
    The UPC triggers were required to have at lease one reconstructed pixel 
      track in order to reject backgrounds where no particles were 
      reconstructed by the tracker.
    For the muon trigger in Table~\ref{tab:hltTriggers2011} the rate was 
      reduced by nearly a factor or 4 compared to its L1 seed rate in 
      Table~\ref{tab:l1Triggers2011}.
    \begin{table}[h]
      \centering
      \begin{tabular}{|l|l|l|l|l|l|}
        \hline HLT trigger  & Rate (Hz) & L1 prescale & HLT prescale & L1 seed & Type \\ \hline \hline
        L1UPCMuon and Pixel Track & 0.52 & 1 & 1 & 1 & \multirow{3}{*}{Physics} \\ \hhline{-----~} 
        L1UPCECAL2 and Pixel Track & 1.65 & 2 & 1 & 2 & \\ \hhline{-----~}
        L1UPCECAL5 and Pixel Track & 0.26 & 1 & 1 & 3 & \\ \hline
        L1ZDCOr & 3.6 & 1500 & 11 & 4 & \multirow{2}{*}{Monitor}  \\ \hhline{-----~}
        L1ZDCOr and Pixel Track & 2.8 & 1500 & 1 & 4 & \\ \hline
        L1UPCMuonHFVeto and Pixel Track & 0 & off & off & 5 & \multirow{3}{*}{Backup}   \\ \hhline{-----~}
        L1UPCECAL2HFVeto and Pixel Track & 0 & off & off & 6 & \\ \hhline{-----~}
        L1UPCECAL5HFVeto and Pixel Track & 0 & off & off & 7 & \\ \hline 
      \end{tabular}
      \caption{List of 2011 HLT trigger.}
      \label{tab:hltTriggers2011}
    \end{table}

    The total HLT output for the UPC trigger package was limited to 20 Hz. 
    The limiting factor for the HLT rate was the amount of disk space given 
      to this analysis. 
    To meet the bandwidth requirements and collect a significant sample
      of data for estimating efficiencies, the prescales were balanced with 
      the goal of achieving at least 5\% statistical precision on the 
      efficiency measurements. 
    As an example of the balancing of the prescales, the HLT  ZDC trigger that 
      did not require a pixel track was given a additional prescale factor 
      of 11 on the HLT.
    The ZDC path that also required a pixel track on the HLT, which used 
      the same L1 seed, was only prescaled at the L1.
    The prescale of 11 was set to ensure that at least 1000 of the pixel track 
      ZDC triggers overlapped with the ZDC L1 only triggers so that efficiency
      of the pixel track requirement in the trigger could be estimated from 
      the tracks lost.

  \section{Studies of 2011 PbPb data}
    The UPC triggers for the 2011 PbPb run make several studies possible.
    Three such studies are discussed below: $\gamma\gamma \rightarrow e^{+} 
      e^{-}$, UPC interactions in peripheral nuclear collisions, and forward
      UPC \JPsi{} using HF. 

    \subsection{High mass $\gamma\gamma \rightarrow e^{+} e^{-}$  in PbPb 2011}
      This measurement would make use of the electron triggers and combine the 
        current dimuon data with dielectron data from the ECAL triggers.
      Because of the smaller mass of the electron,
        dielectron production is slightly favored compared to dimuon 
        production.
      STARlight predicts that dielectron cross section is a factor of 
        2.5 higher in Xn break-up mode than for the dimuons channel when looking 
        at masses above 4 GeV.
      The ECAL is positioned just beyond the tracker, whereas the muon system is 
        the outermost sub-detector.
      Electrons would therefore be less effected by the material budget, which
        is the main factor in reducing yields of reconstructed dimuon 
        candidates.

      The contribution from higher order diagrams can be explored by studying 
        photoproduction of dilepton pairs.
      Because the Pb nucleus has a charge 82 times higher than that of the 
        proton, the electromagnetic coupling is stronger. 
      Thus higher order terms in the perturbative expansion could potentially 
        be more important for the Pb nucleus compared to the proton.
      Recent results by ALICE favor very small contributions for higher order
        terms \cite{Abbas:2013oua}.
      In addition, this analysis provides a useful cross check to the UPC 
        quarkonia analyses such as \JPsi{} by verifying the cross section 
        normalization.

    \subsection{UPC hadronic overlap}
      In the model calculations for UPC quarkonia 
        photoproduction all hadronic interactions are rejected.
      However, inclusive \pt{} spectra of \JPsi{} measured by ALICE in 
        peripheral PbPb collisions show a low momentum peak consistent with 
        coherent photoproduction ~\cite{aliceIclJpsi}.
      \begin{figure}[h]
        \centering
        \includegraphics[width=0.5\textwidth]{2012-Sep-24-excess7090.pdf}
        \caption{Coherent excess in inclusive $J/\psi$ $p_{T}$ spectrum.}
        \label{fig:alicePtSpecLowPt}
      \end{figure}
      The ALICE spectra provide hints that UPC processes might also be present 
        in peripheral nucleus-nucleus collisions at the LHC.
      
      To study the overlap between photoproduction and hadronic production of 
        quarkonium events, the inelastic sample and the UPC sample could both be
        used. 
      The looseness of the rejection criteria to reject hadronic interactions,
        which uses the BSC detectors, leaves a significant overlap with 
        peripheral hadronic collisions. 
      The inclusive quarkonia sample from typical hadronic collisions can also 
        be utilized. 
      Coherent quarkonia photoproduction has a distinctive low $p_{T}$ structure
        that can be used to identify photoproduced candidates in a sample that 
        contains photoproduction combined with hadronic interactions.
      This measurement would open up the door to exploring the boundary between
        photoproduction and hadronic production.

    \subsection{UPC \JPsi{} with electrons in HF}
      As higher rapidities are explored both lower and higher momentum partons
        of the nucleus are probed. 
      Because these two contributions to the UPC photoproduction cross section 
        can be separated using neutron tagging in incoherent events, exploring
        higher dilepton rapidities becomes attractive.
      HF extends to 5 in $\eta$, which is 2.6 units beyond the edge of the 
        tracker.
      By combining hits in HF with tracks in the tracker,  
        \JPsi{} from higher rapidities can be measured. 
      When combined with neutron tagging of incoherently produced quarkonia,
        the current study can be extended to probe lower-$x$ nuclear partons 
        by identifying leptons in HF. 

 \section{\label{sec:pPbTrigDev}Trigger development for the LHC pPb Run}
   Specific UPC triggers were also developed for the pPb run in 2013. 
    For this period of running a much higher total trigger rate was read out 
      relative to 2011.
    The total rate allocated for UPC triggers at the L1 in 2013 was 5 kHz and 
      50 Hz at the HLT.
    This factor of 5 increase in HLT and factor of 25 in L1 bandwidth,
      allowed for a change in emphasis from the L1 to the HLT. 

    The basic strategy in 2013 was the same as in 2012, use the loosest 
      available ECAL and muon L1 triggers to capture the lowest \pt{}
      electrons and muons possible and reject hadronic interactions.
    Because of the L1 bandwidth restrictions in 2011, both the ZDCs and the 
      BCSs were used on the L1 to reduce rates.
    In 2013 only the muon and ECAL triggers were used on the L1 allowing for 
      rejection of hadronic interactions through cuts on track multiplicity. 
    In addition, a more sophisticated trigger using full dimuon reconstructed 
      was developed to increase purity.
    The main advantage in this shift in strategy was a higher purity due to 
      the increased sophistication of the reconstruction on the HLT.
    In addition, the cross section for photoproduciton without regard to 
      neutron emission is higher.
    Therefore, relaxing the requirement of activity in the ZDCs further 
      increases measurable yields. 

    The HLT triggers in 2013 rejected hadronic interactions through counting
      tracks. 
    For the five UPC trigger paths included in the HLT menu, 
      three levels of reconstruction were done at the HLT.

    \begin{itemize}
      \item Pixel tracks were reconstructed from the inner pixel section of the 
        silicon tracker alone, tracks were reconstructed using the full
        tracker using the strips as well, and full dimuon reconstruction was 
        done using the tracker and muon detector. 
      \item The least restrictive pixel track paths required at least 
        one track reconstructed from the pixel detector and less than 10 pixel 
        tracks in the event.
      \item Full tracking paths were added on top of the pixel track paths and included
        an additional requirement of one full track and less than 7 reconstructed
        tracks.
      \item The most restrictive path added to the pixel and full tracking paths and 
        required reconstruction of dimuons with a mass between 2 and 12 GeV.
    \end{itemize}

    \subsection{$J/\psi$ photoproduction in ultra-peripheral pPb collisions}
      The CMS UPC triggers commissioned for the 2013 LHC pPb run will allow for
        further analysis of \JPsi{} photoproduciton.
      In ultra-peripheral pPb collisions, \JPsi{} photoproduciton is dominated 
        $\gamma-p$ interactions~\cite{Frankfurt:2006tp,Guzey:2013taa}.
      The measurement would primarily probe the proton gluon densities.
      In Eq.~\ref{eq:photonFluxFinaltmp} the photon flux depends on the square
        of the number of protons in the parent nucleus, $Z^{2}$. 
      However, the cross section of the target increase as the total 
        number of nucleons to the two thirds power, $A^{2/3}$.
      The much higher photon flux from the Pb-ion compensates for 
        the decreased size of the proton.

      A pPb UPC \JPsi{} measurement will complement the measurements done at 
        HERA~\cite{Chekanov:2002xi,Aktas:2005xu,Alexa:2013xxa}, and measurements done by ALICE~\cite{TheALICE:2014dwa}.
      CMS will contribute by adding additional kinematic coverage and cover a 
        unique range of $\gamma$p center of mass energies, $W_{\gamma p}$. 
      The difference in beam energies and species at LHC versus HERA result in
        access to different $W_{\gamma p}$. 
      ALICE and CMS have different acceptance in \JPsi{} rapidity, which also 
        translates to coverage of different $W_{\gamma p}$.
      In addition, an excess in the UPC cross section compared to HERA 
        measurements would indicate a non-exclusive contribution to the pPb UPC 
        \JPsi{} cross section. 
      This measurement will enhance the current understanding of 
        the \JPsi{} photoproduction cross section as a function of the 
        photon-proton center of mass energy, $W_{\gamma p}$.

\chapter{Analysis}	
  \section{MC Simulation}
    Monte Carlo (MC) simulations are used to understand how the detector 
      effects the measurement.


    Two main classes of Monte Carlo (MC) simulation samples were used. 
    The first class uses STARlight to generate events. 
    The Second class uses PYTHIA6 to create and decay $J/\psi$s with a given
      input $p_{T}$ and rapidity distribution. 
    The 

  \section{\label{sec:TrigDev} Trigger Development} 
    Prior to the 2011 LHC PbPb run, UPC events had not been directly studied in 
      PbPb collisions using CMS. 
    Design of the UPC triggers required studies of the 2010 data to estimate 
      rates and assure that the bandwidth used by these trigger would be
      sufficiently low. 
    All the different physics analyses must share the limited readout rate of 
      the detector.
    For this reason, conservation of bandwidth was a major design consideration.

    To estimate the 2011 rates prior to the run, the 2010 rates were used to 
      extrapolate to the interaction rate of the 2011 run. 
    The unique UPC triggers were estimated by combining existing triggers from
      the 2010 run. 
    By calculating the ratio between the UPC trigger rates and the minimum bias
      trigger rate, the UPC trigger rates were scaled up to the 2011 
      interaction rates using the 2010 data. 
    The extrapolated rates allowed for a package triggers to be created, which 
      fit within the bandwidth requirement of CMS Heavy Ions. 
    
    The trigger package for 2011 contained ZDC based efficiency monitoring 
      triggers, muon and electron based triggers for measuring $J/\psi$, and 
      backup triggers in case there was a problem with the original muon and 
      electron triggers.
    In order to recorded the trigger efficiency monitoring data, the ZDC 
      triggers had to be prescaled to a lower rate. 
    The scaling down of the monitoring triggers were setup to assure overlap
      with the signal triggers.
    By balancing the competing objectives of rate reduction and increasing 
      the overlap between the monitoring and signal triggers, 
      the prescales for the trigger were as seen in Table .%~\ref{triggerTabel2011}.

    \subsection{L1 Trigger}
      The goal of the L1 triggers was to record enough data to measure dimuons
        and dielectrons in UPC events.
      To achieve this, the loosest muon trigger and lowest threshold ECAL 
        triggers where paired with a trigger on energy in the ZDC and a veto on
	energy in the BSC or HF.
      The L1 package that was constructed for the analysis of UPC $J/\psi$ 
        is presented in Table~\ref{tab:l1Triggers2011}.

	\begin{table}[h]
		\centering
		\begin{tabular}{|l|l|}
		  L1 Trigger Seed  & Type \\ \hline \hline
		  L1\_MuOpen\_ZdcCalo\_NotBscMinBiasThresh2\_BptxAND & Physics \\  \hline
		  L1\_EG2\_ZdcCalo\_NotBscMinBiasThresh2\_BptxAND & Physics \\  \hline
		  L1\_EG5\_ZdcCalo\_NotBscMinBiasThresh2\_BptxAND & Physics \\ \hline
		  L1\_ZdcCaloMinus\_BptxAND & Monitor \\  \hline
		  L1\_ZdcCaloMinus\_BptxAND & Monitor \\  \hline
		  L1\_MuOpen\_ZdcCalo\_NotHcalHfCoincidencePm\_BptxAND & Backup \\ \hline
		  L1\_EG2\_ZdcCalo\_NotHcalHfCoincidencePm\_BptxAND & Backup \\ \hline
		  L1\_EG5\_ZdcCalo\_NotHcalHfCoincidencePm\_BptxAND & Backup \\ \hline \hline
		\end{tabular}
		\caption{List of 2011 L1 seeds.}
		\label{tab:l1Triggers2011}
	\end{table}
       
       The cumulative L1 trigger rate for all the UPC L1 trigger seeds was
         required to be 200 Hz.
       This requirement stemmed from the need to keep the tracker read-out rate
         low. 
       The trackers baseline voltage can fluctuate due to the high tracker hit 
         multiplicities in PbPb collisions.
       In order to monitor the zero suppression of the tracker, the zero 
         suppression algorithm was executed using the HLT computing farm 
	 rather than the in the tracker firmware. 
       The additional computing cycles needed to run the zero suppression 
         set the limit for L1 bandwidth. 

    \subsection{HLT Trigger}
      As opposed to the L1 trigger, which reads out the tracker, the HLT has 
        access to the tracker information. 
      Reconstruction of a track in the pixel detector is used by the UPC paths.
      The use of the pixel detector only, as opposed to using the whole tracker 
        including the silicon strip detector, allows for quick track 
	reconstruction saving computing cycles.
      The requirement of at least on reconstructed pixel track for the HLT 
        triggers was designed to reject backgrounds where no particles are 
	reconstructed by the tracker. 
  \begin{table}[h]
		\centering
		\begin{tabular}{|l|l|}
		  \hline HLT Trigger  \\ \hline \hline
		  HLT\_HIUPCNeuMuPixel\_SingleTrack & Physics   \\ \hline
		  HLT\_HIUPCNeuEG2Pixel\_SingleTrack & Physics   \\ \hline
		  HLT\_HIUPCNeuEG5Pixel\_SingleTrack & Physics   \\ \hline
		  HLT\_HIMinBiasZDC\_Calo\_PlusOrMinus\_v1  & Monitor  \\ \hline
		  HLT\_HIMinBiasZDC\_PlusOrMinusPixel\_SingleTrack\_v1   & Monitor \\ \hline
		  HLT\_HIUPCNeuHcalHfMuPixel\_SingleTrack & Backup   \\ \hline
		  HLT\_HIUPCNeuHcalHfEG2Pixel\_SingleTrack & Backup   \\ \hline
		  HLT\_HIUPCNeuHcalHfEG5Pixel\_SingleTrack & Backup   \\ \hline \hline
		\end{tabular}
		\caption{List of 2011 HLT trigger.}
		\label{tab:hltTriggers2011}
	\end{table}

	The total HLT output for the UPC trigger paths was 20 Hz. 
	The limiting factor for the HLT rate was the amount of disk space 
	  available to store the data. 

    \section{\label{sec:DataSetEvSel} Data Sets and Event Selection}
  \subsection{Data Set}
    In order to investigate novel physics processes like UPC $J/\psi$ 
     production, the LHC has delivered unprecedented amounts of data.
    The data for this analysis was recorded during the 2011 LHC PbPb run. 
    During this period, 157 $\mu$$b^{-1}$ where recorded by the CMS detector,
      corresponding to over a billion PbPb collisions. 
    Of this, 143 $\mu$$b^{-1}$ are used in this analysis.

    Three specially selected samples are used (see Table~\ref{tab:sampleLumiNevt}).
    These samples were recorded using subsets of the triggers found in 
      Section~\ref{sec:TrigDev}.
    The $J/\psi$ events discussed in this thesis were obtained analyzing the 
    sample labeled in Table~\ref{tab:sampleLumiNevt} as physics.
    A minimum bias sample was recorded for the sake of estimating efficiencies.
    Last, a zero bias sample was recored for investigating the ZDC and the 
      noise distributions of HF.
    By recording this hierarchy of samples, interesting events are selected 
      with a much higher purity in the physics sample, while the zero bias and 
      minimum bias samples allow for the investigation of the selection 
      criteria as will be discussed in the next section. 

    To record the physics sample containing the $J/\psi$ signal, a muon trigger
      was paired with a veto on energy in the BSCs and a requirement that there 
      be energy in at least one of two sides the ZDC. 
    This trigger utilizes the unlikely chance of having overlapping noise
      in the ZDC and muon detector.
    Because of the characteristically low momentum of UPC $J/\psi$ as compared
      to $J/\psi$ created by other hadronic process, the loosest muon 
      trigger was used.
    The standard muon trigger for hadronicly produced $J/\psi$s contains a 3 
      GeV $p_{\mathrm{T}}$ threshold, which would reject nearly all UPC 
      $J/\psi$s.
    The ZDC was included as part of this physics trigger because it was thought
      to reduce possible noise contributions when used in coincidence with 
      the muon chambers. 
    Contributions from hadronic interactions are reduced by the veto on the 
      BSC.
    In this way the balance between reducing the rate and maximizing the 
      efficiency was struck, allowing for the data to be recorded within 
      the bandwidth limitations set by the physics program of CMS Heavy Ions.
    
    In order to investigate the muon trigger and the other parts of the events 
      selection, a minimum bias sample was recorded using the ZDC. 
    For ZDC triggered sample, any event which had energy consistent with at 
      least one neutron in either of the two sides of the ZDC was recorded.
    This process is much more common than the UPC $J/\psi$ production.
    For this reason, the rates of this trigger are much higher than the physics
      trigger, and only a small sub set of these events are recorded.
    From this trigger the pixel track efficiency was estimated (see 
      Section. 

    In addition to the minimum bias and physics sample, a zero bias sample was 
      recorded to examine the ZDC trigger and the HF noise distributions. 
    This sample records every event when both beams passed through CMS. 
    Only 4 events out every million triggered were recorded for this sample. 
    This sample allowed for an unbiased measurement of the ZDC trigger 
      efficiency. 
    Because the zero bias trigger does not require any activity any of the
      CMS sub detectors, the sample contains very few hadronic collisions. 
    This allowed for a measurement of the electronic noise distributions in
      the HF.

    The integrated luminosity for the each of the three samples is calculated
      by recording activity in HF. 
    The cross section for HF activity is measured from a van der Meer scan and
      the cross section was found to be \textcolor{red}{X}.
    In this way the amount of integrated luminosity for any period running is
      related to the activity in HF. 
    \begin{table}
	    \centering
	    \begin{tabular}{|l|l|l|}
	      \hline Sample & Events & $L_{int}$ \\ \hline \hline
	      Physics & \textcolor{red}{300K} & \textcolor{red}{143.3 
	        $\mu$$b$} \\ \hline
	      Minimum Bias & \textcolor{red}{100K} & \textcolor{red}{X} \\ \hline
	      Zero Bias & \textcolor{red}{5M} & \textcolor{red}{580 b} \\ \hline \hline
	    \end{tabular}
	    \caption{Integrated luminosities and number of events for the three
	      samples used in this analysis.}
	    \label{tab:sampleLumiNevt}
    \end{table}
    To check the values that come from the van der Meer scan technique, the 
      number of events that fire the L1 minimum bias trigger in the zero bias
      sample were counted and used to estimate the integrated luminosity for 
      that sample using the inelastic PbPb cross section. 
    From and integrated luminosity of 595$\pm 60$ barns is estimated, which  
      agrees with in uncertainty with the van der Meer scan method. 

  \subsection{Event Selection}
    

    Two sets of event selection cuts are applied to reject background events. 
    The first set rejects background from the beam.
    The second reject events where hadronic collisions have occurred.
    
    Three cuts are used to eliminate beam backgrounds.
    For this purpose, the following cuts are applied:
    \begin{itemize}
	    \item the reconstructed vertex must be within X cm in the in 
		    transverse dimension and X cm in the longitudinal
	    \item muon beam halo rejection
	    \item pixel cluster shape compatibility
    \end{itemize}
    The vertex cut assures that reconstructed particles are come from 
      interactions between the two beams rather than then event where one of
      the two beams interact with gas particles near the interaction point. 
    The pixel cluster shape compatibility cut requires that the clustering
      of energy deposits in the silicon tracker point back to the reconstructed
      primary vertex. 
    This cut rejects events where remnants of the beam scrap the innermost 
      most layer of the tracker. 
    By using timing on either side of the interaction point, the beam halo cut
      rejects events where muons surrounding stream through the detector. 

    Three cuts are applied to reject hadronic interactions. 
    \begin{itemize}
	    \item no more than 2 reconstructed tracks in the event
	    \item maximum RecHit energy in HF consistent with electronic noise 
	    \item Energy in the ZDCs consistent with neutrons on only one side 
		    of the interaction point
    \end{itemize}
    Each of these cuts are designed to reject topologies that are produced when
      the two colliding nuclei overlap and interact through the strong force. 
    The track requirement rejects events that produce many charge particles.
    Nearly all hadronic interactions (~98\%) produce particles in the 
      range $3<|\eta|<5$ covered by the HF detector.
    By requiring that the energy deposits in HF resemble noise, nearly all
      elastic hadronic collisions are expected to be rejected.
    In elastic hadronic events the both nuclei break-up. 
    By requiring that ZDC only reconstruct neutrons on one side of the 
      interaction point, hadronic interactions that produce neutrons on both 
      sides are reject.

  \section{break up determination}
    \subsection{ZDC Signal Reconstruction}
      There are 18 zdc channel. 
      There are 4 hadronic channels and 8 electromagnetic channels on each side
        of the CMS. 
      To measure neutrons in the ZDC the charge in each channel is first 
        converted to a signal. 

	\begin{figure}[h]
		\centering
		\includegraphics[width=\textwidth]{zdcPulseShape}
		\caption{ZDC pluse shape.}
		\label{fig:zdcPulseShape}
	\end{figure}

      Channel Signal definition:
      The signal in the zdc calculated two ways. 
      \begin{enumerate}
	\item Method 1
	\begin{enumerate}
	  \item signal: 4,5,6 
          \item background: 1,2
        \end{enumerate}
	\item Method 2
	\begin{enumerate}
	  \item signal: 4
	  \item background: 5
        \end{enumerate}
      \end{enumerate}

      As seen in Fig~\ref{fig:zdcPulseShape}, method 1 uses all the dominant 
        signal time slices and uses the maximum number of “clean” non signal 
        time slices to estimate the noise pedestal event by event. 
      This minimizes the effect of random noise time slice by time slice by 
        averaging over the maximum number of time slices per event. 

      Method 2 uses only the two time slice that are most likely to be above 
        zero. 
      Because all signal that is less than zero is not measured, method 1 
        only allows for a noise pedestal measurement half the time.
    
      From these two methods the ZDC+ and ZDC- energy spectra near the 
        one neutron peak are plotted in Fig.~\ref{fig:zdcSpec2v1}.
      While method 1 in blue and method 2 in reddo not differ much in ZDC-, 
        the clear separation of the one neutron peak signal from the noise 
        peak about zero is evident. 
      \begin{figure}[h]
        \centering
        \includegraphics[width=\textwidth]{zdcSpec2v1}
        \caption{Comparison of ZDC signal reconstruction methods.}
        \label{fig:zdcSpec2v1}
      \end{figure}

    \subsection{Determination of the one neutron thresholds}
      The ZDC thresholds used to establish the break-up mode were measured from
        zero bias data.
      By using this dataset, the spectrum does not contain a trigger bias. 
      The trigger requriment in the event is that both beams were present in 
        CMS.
      This does however include a significant electronic noise contribution due
        to events where no neutrons are emitted in the direction of the zdc.
      To sperate the signal from the electronic noise additional cuts are 
        applied in the zero bias data.

      In Fig.~\ref{fig:zdcPulseShape} the pulse shape peaks in the peaks in the
        fourth time slice, for electronic noise however, any of the ten time 
        slices are equally likely to have a peak value.
      Using this fact, signal can be preferably selected by requiring that the
        hadronic channels of the ZDC have a peak signal in the fourth time 
        slice.
      \begin{figure}[h]
        \centering
        \includegraphics[width=\textwidth]{zdcMinusSingleNuNoInc}
        \caption{Effects of requiring in time signal in ZDC hadronic 
          channels.}
        \label{fig:zdcTimingCuts}
      \end{figure}
      In Fig.~\ref{fig:zdcTimingCuts} no noise subtraction is used. 
      As each additional hadronic channel is required to have a maximum signal
        in the fourth time slice, the single neutron peak emerges. 
      Using the noise subtraction method described by Method 1, same signal 
        emerges.
      \begin{figure}[h]
        \centering
        \includegraphics[width=\textwidth]{zdcMinusSingleNuNoSub}
        \caption{Effect of ZDC signal timing requirements after noise 
          subtraction.}
        \label{fig:zdcTimingAfterNoiseSub}
      \end{figure}
      Fig.~\ref{fig:zdcTimingAfterNoiseSub} confirms that both noise 
        subtraction and the timing require produce the same signal.
      This gives confidence that the signal is not an artifact of either cut, 
        but the true neutron signal. 

  \section{Signal Extraction}
  \section{\label{sec:effDet} Efficiency determination}

\chapter{\label{sec:sysCheck}Systematic uncertainties}
  
  Table~\ref{tab:sumsyst} shows the systematic errors that were estimated.
  The method used to separate the coherent from the photon-photon process 
   is the most dominant error.
  The ZDC reconstruction method used to estimate the neutron thresholds 
    is the next most dominant, followed by the method used to estimate
    the HF noise threshold. 
  \begin{table}[!Hhtb]
    \begin{center}
      \begin{tabular}{|c|c|c|}
        \hline
        systematic & uncertainty in \%  \\ \hline
        Template fit normalization & +9.5\% -12.0\%    \\ \hline
        ZDC trigger efficiency & 3.5\%    \\ \hline
        ZDC reconstruction  & 2.9\%  \\ \hline
        HF noise threshold & +1.3\% -3.4\%    \\ \hline 
        MC acceptance & 1.1\%    \\ \hline
        \hline \hline
        Total systematic & +10\% -13\%   \\ \hline
      \end{tabular}
      \caption{Summary of systematic uncertainties}
      \label{tab:sumsyst}
    \end{center}
  \end{table}

  \section{Template fit normalization}
    \begin{figure}[!Hhtb]
      \centering
      \includegraphics[width=.6\textwidth]{ptOnly}
      \caption{Coherent, incoherent, and photon-photon process \pt{} template fit to data.}
      \label{fig:ptTempFit}
    \end{figure}
   
    The \pt{} template fit depends on the functions chosen for fitting
      the mass distribution.
    As described in Section~\ref{sec:sigEx}, the similarity of the 
      \pt{} distribution for the coherent and photon-photon process makes
      the contributions from the two process difficult to separate from the 
      \pt{} distribution alone.
    The mass distribution was used to distinguish between these two processes.

    The systematic uncertainty due to the choose of functions used to fit
      the mass distribution was estimated by varying the signal and 
      background functions.
    The contribution to the background from the mass fit was used to fix the
      contribution from the photon-photon process in the \pt{} template
      fit.
    Two functions were used to describe the signal, a Gaussian, and a Crystal
      ball function. 
    The background was fit to a linear function, a 2nd order polynomial, and
      a 2nd order Cheby-Chev polynomial. 
    The resulting variation on the coherent contribution was used to as an
      estimate of this systematic effect. 

    \begin{figure}[!Hhbt]
      \centering
      $ \begin{array}{ccc}
        \includegraphics[width=.3\textwidth]{cbPolyBkgEst} &
        \includegraphics[width=.3\textwidth]{gausLinBkgEst} &
        \includegraphics[width=.3\textwidth]{gausCCBkgEst} \\
        \includegraphics[width=.3\textwidth]{cbPoly} &
        \includegraphics[width=.3\textwidth]{gausLin} &
        \includegraphics[width=.3\textwidth]{gausCC}
      \end{array} $
      \caption{Various mass distribution fits and the corresponding \pt{}
        template fit.}
      \label{fig:massPtFitsForSyst}
    \end{figure}

    Moving from left to right in Fig~\ref{fig:massPtFitsForSyst}, the 
      contribution from the photon-photon process increases.
    The $\chi^{2}$ / degree of freedom is similar between the three 
      fits indicating a similar goodness of fit.
    On this basis, neither fit is preferred. 
    The left most fit uses a Crystal Ball function to account for the 
      radiative decay of the final state daughters of the \JPsi{}.
    The low mass exponential portion however picks up background events 
      and overestimates the \JPsi{} contribution. 
    The right most plot fits the background to a 2nd order Cheby-Chev 
      polynomial.
    Because the Cheby-Chev peaks just below the \JPsi{} peak, this fit 
      overestimates the background and in turn underestimates the signal 
      contribution.
    The Gaussian fit with a linear background however does a reasonable job
      of fitting both the background and the signal. 

    From these three fits an upper and lower bound of the systematics due
      the choice of fit functions was estimated. 
    The difference between the Gaussian-Linear fit and the 
      Crystal Ball-polynomial fit was taken as an upper bound. 
    The difference between the Gaussian-Linear fit and the 
        Gaussian-Cheby-Chev fit was taken as a lower bound. 
    The overall systematic uncertainty due to the choose of normalization for
      the photon-photon template is found to be $^{+9.5\%}_{-12\%}$.

  \section{ZDC trigger efficiency}
    The ZDC trigger efficiency measurement is sensitive to the underlying 
      neutron distribution.
    The more neutrons that hit the ZDC, the higher the trigger efficiency 
      will be.
    To estimate the effect the input sample has on the efficiency, the ZDC 
      trigger efficiency was measured from five different samples.
    Table~\ref{tab:zdcEfficiencySys} shows the results from the 
      three samples, which require a reconstructed pixel track in the event to 
      reduce noise. 
    Both the nominal and comparison ZDC reconstruction methods are shown.
    \begin{table}
      \centering
      \begin{tabular}{|c|c|c|c|c|}
        \hline ZDC Side & Reco Method & N$_{events}$ & N$_{trig}$ & $\varepsilon_{ZDC}$ \\ \hline
         \multicolumn{5}{|c|}{(ZDC$^{+}$ or ZDC$^{-}$) and 1 pixel track} \\ \hline 
         ZDC$^{-}$ & comparison & 72946  & 71688 & 0.982 $\pm$ 0.005 \\ \hline
         ZDC$^{-}$ & nominal & 73028  & 71706  & 0.982  $\pm$ 0.005  \\ \hline
         ZDC$^{+}$ & comparison & 76137  & 71786  & 0.943  $\pm$ 0.005  \\ \hline
         ZDC$^{+}$ & nominal & 76132  & 71859  & 0.944  $\pm$ 0.005  \\ \hline
         \multicolumn{5}{|c|}{(ZDC$^{-}$ or ZDC$^{+}$), 1 pixel track, and L1 EG trigger } \\ \hline 
         ZDC$^{-}$ & comparison & 613758  & 602123  & 0.9810 $\pm$ 0.0018 \\ \hline
         ZDC$^{-}$ & nominal & 614014  & 601863  & 0.9802 $\pm$ 0.0018 \\ \hline
         ZDC$^{+}$ & comparison & 643905  & 602671  & 0.9360  $\pm$ 0.0017 \\ \hline
         ZDC$^{+}$ & nominal & 647888  & 603089  & 0.9309  $\pm$ 0.0017 \\ \hline
         \multicolumn{5}{|c|}{(ZDC$^{-}$ or ZDC$^{+}$), 1 pixel track, and L1 Muon trigger} \\ \hline 
         ZDC$^{-}$ & comparison & 65466  & 63376  & 0.968 $\pm$ 0.005  \\ \hline
         ZDC$^{-}$ & nominal & 65543  & 63358  & 0.967 $\pm$ 0.005 \\ \hline
         ZDC$^{+}$ & comparison & 71929  & 63512  & 0.883  $\pm$ 0.005 \\ \hline
         ZDC$^{+}$ & nominal & 72932  & 63582  & 0.872  $\pm$ 0.005 \\ \hline
       \end{tabular}
      \caption{ZDC trigger efficiencies using  the nominal and comparison 
        ZDC reconstructions for trigger sample that require a pixel track.}
      \label{tab:zdcEfficiencySys}
    \end{table}

    The systematic uncertainty in the ZDC trigger efficiency due to the 
      uncertainty in the underlying distribution was estimated by calculating 
      the standard deviation of efficiency measurements in Table~\ref{tab:zdcEfficiencySys}.
    The uncertainty in the ZDC trigger for ZDC$^{-}$ was found to be less than 
      1\% and taken to be negligible. 
    For ZDC$^{+}$, the systematic uncertainty was taken to be 3.5\%.

  \section{ZDC reconstruction}
    In order to estimated the systematic uncertainty in the ZDC reconstruction
      method, the comparison ZDC reconstruction method described in 
      Section~\ref{sec:zdcCompare} is used to measure the number of candidates
      in the Xn0n mode.
    The systematic uncertainty due to the ZDC reconstruction method is
      estimated from the difference between the UPC \JPsi{} candidate yields 
      when using the comparison versus the nominal method.
    The yields for the nominal and comparison ZDC reconstruction method in the 
      Xn0n break up were found to be 298 and 315 respectively. 
    Half the difference between the two methods was used as an estimate of 
      the systematic uncertainty, giving 2.9\%.

  \section{HF noise threshold}
    The way in which the HF noise distribution is measured effects the event 
      selection and therefore the final candidate yield.
    This cut plays a significant role in rejecting hadronic events.
    In Table~\ref{tab:evSelCutNumbers} the importance of cutting on HF noise
      is evident. 
    The HF noise cut rejects nearly 1/5 of the remaining events. 
    The systematic uncertainties on the HF noise requirement is important for
      this reason.
   
    The most basic information from the HF detectors are contained in RecHits. 
    There is one RecHit per phototube on HF. 
    The RecHit signal is calibrated in GeV, and no noise subtraction is done. 
    The CaloTowers are formed from geometrical groups of RecHits. 
    They are the first stage of the CMS jet trigger and perform some noise 
      suppression.

    The default HF noise cut required that the maximum RecHit energy
      from both HF+ and HF- be less than 3.85 GeV. 
    This cut was designed to accept 99\% of the noise events, 
      see Fig.~\ref{fig:hfNoiseDist}. 
    The stability of this cut was tested by
    \begin{enumerate}
      \item Summing CaloTowers instead of RecHits
      \item Making separate cuts on HF- and HF+
      \item Tightening the threshold so that only 98\% or 97\% noise events 
        passed the cut.
    \end{enumerate}
    \begin{table}[!Hhbt]
      \centering
      \begin{tabular}{|c|c|c|c|}
        \hline
        Object type & HF (GeV) & HF$^{-}$ (GeV) & HF$^{+}$ (GeV) \\ \hline
        RecHits & 3.85 & 3.25 & 3.45 \\ \hline
        CaloTowers & 4.25 & 3.25 & 3.75 \\ \hline
      \end{tabular}
      \caption{Thresholds from combined HF noise distributions and 
        two-side noise distributions.}
      \label{tab:hfNoiseThreshAsym}
    \end{table}

    \begin{table}[!Hhbt]
      \centering
        \begin{tabular}{|c|c|c|} \hline
          \% &  $E_{RecHit}$ (GeV) & $E_{CaloTower}$ (GeV)\\ 
          \hline
          99 & 3.85& 4.25 \\ \hline
          98 & 3.25& 3.75 \\ \hline
          97 & 2.95& 3.25 \\  \hline
         \end{tabular}
        \caption{HF noise thresholds for keeping various fractions of the noise
          sample for RecHit and CaloTower.}
        \label{tab:hfAdjustedThresholds}
    \end{table}

    Table~\ref{tab:hfNoiseThreshAsym} shows the noise thresholds for RecHits 
      and CaloTowers for both the combined HF+ and HF- calorimeters and the 
      two-sided individual calorimeters when 99\% of noise events are accepted.
    Table~\ref{tab:hfAdjustedThresholds} compares the threshold for the cases 
      when 99\%, 98\% and 97\% of noise events are accepted.
    The number of \JPsi{} events remaining after these cuts is shown in 
      Table~\ref{tab:hfAdjThreshYields}. 
    The efficiency corrected numbers are also shown in order to compare between
      different noise thresholds. 
    The fractional systematic error is then estimated by finding the maximum 
      and minimum deviation from the default method. 
    The nominal number of candidates come from the 99\% combined RecHit 
      threshold.
    From Table~\ref{tab:hfAdjThreshYields}, the maximum number of corrected 
      candidates comes from the 99\% threshold for combined CaloTower objects,
      and the minimum from 97\% threshold for the same objects. 
    The fractional increase from the maximum of 305 corrected candidates and 
      the nominal value of 301 is taken as the systematic upper bound; the 
      bound is estimated from the minimum of 289, giving a systematic 
      uncertainty of $^{+0.3\%}_{-3.4\%}$.
    \begin{table}[!Hhbt]
      \centering
        \begin{tabular}{|c|c|c|c|c|c|} \hline
          \% &  RecHit cut & RecHit corrected & CaloTower cut & CaloTower corrected & Threshold type \\ \hline
          99 & 298 & 301 & 302 & 305 & \multirow{3}{*}{Combined} \\ \hhline{-----~}
          98 & 287 & 293 & 294 & 300 & \\ \hhline{-----~}
          97 & 284 & 292 & 280 & 289 & \\ \hline \hline
          99 & 290 & 293 & 288 & 291 & Two-sided \\ \hline
        \end{tabular}
      \caption{Number of upc dimuon candidates with  p$_{T} <$ 1 GeV when changing HF calorimeter cuts for RecHit and CaloTower.}
      \label{tab:hfAdjThreshYields}
    \end{table}

  \section{MC acceptance}
    The MC derived acceptance correction factors depend on the input physics
      generator. 
    The underlaying \pt{} distribution was assumed to be correctly 
      described by STARlight for the coherent cross section measurement.
    To estimate the effect of changing the underlaying \pt{} distribution 
      on the acceptance measured from the MC, the incoherent sample was used 
      to correct the coherent yield.
    Half the difference was used as the estimate and was found to be 1.1\%.

  \section{\label{sec:extraSys}Additional systematic checks}
    \subsection{Mass fit}
      \begin{figure}[!Hhtb]
        \centering
        $ \begin{array}{cc}
          \includegraphics[width=.45\textwidth]{massFitSimple} &
          \includegraphics[width=.45\textwidth]{massFitCBPoly2}
        \end{array} $
        \caption{Mass fit to \JPsi{} using Gaussian (Left) and Crystal Ball (Right) for the 
          signal and a polynomial for the background}
        \label{fig:massFitSys}
      \end{figure}
      Fig.~\ref{fig:massFitSys} demonstrates the small dependence the raw 
        \JPsi{} yield has on the fitting function. 
      Both fit functions agree well, with reduced $\chi^{2}$ values below one.
      The Crystal ball fit gives an upper estimate for the \JPsi{} yield.
      The Gaussian fit gives a lower estimate. 
      The main difference comes from the lower mass tails.
      In the Crystal ball fit the lower tail is considered to be signal due to 
        shifting of the mass spectrum to lower mass due to radiation from the 
        final state muons. 
      In the Gaussian fit the lower mass tail is considered to be background and 
        the signal is sharper.
  
      As check on the simultaneous \pt{} and mass fit, the mass fit is done
        using mass templates from STARlight.
      The coherent fraction, $f_{co}$, using the mass template for the 
        simultaneous fit gave 0.60 $\pm$ 0.09, which is consistent with the 
        nominal method described in Section~\ref{sec:sigEx}.
      \begin{figure}[!Hhbt]
        \centering
        \includegraphics[width=0.6\textwidth]{ptMassSimTemp}
        \caption{Simultaneous fit to the mass and \pt{} using mass templates
          for the mass fit. }
        \label{fig:simFitTemp}
      \end{figure}
  
    \subsection{Tag and probe from counting compared to fitting}
      The main purpose for fitting the mass spectra to estimate the efficiency
        is to separate the background from true signal. 
      The background may not have the same efficiency as the signal, so 
        separating the two is important if this is the case.
      In the tag and probe fit the signal peak from the \JPsi{} resonance
        is fit to the probes, passing probes, and failing probes alike (see
        Fig.~\ref{fig:tnpFitPlot}). 
      The signal shape, if from the same physical signal, will be 
        identical in each of the three distributions. 
      The background for the passing and failing probes is fit using 
        different parameters for the background because the background
        may come from different physical processes than the signal, or from 
        non-physical sources like combinatorial backgrounds or misidentified
        fake particles.
      When the background comes from sources other than the physical signal,
        the background may give an efficiency estimate that is lower than
        the signal. 
  
      The trigger efficiency measured by the tag and probe method depends on
        the fitting functions use to estimate the background and signal 
        contributions. 
      Depending on what functions is used to fit the spectra, the amount of
        background can be over or underestimated and effect the efficiency 
        measurement.
      To estimate this effect, the tag and probe efficiencies were additionally
        measured by counting probes in the \JPsi{} mass window. 
      The whole mass window is used to estimate the efficiency including all 
        the events from the mass side bands.
      In this way, a worst case scenario estimate is given where all background
        events are included as signal. 
      \begin{figure}[!Hhbt]
        \centering
        $ \begin{array}{cc}
          \includegraphics[width=.45\textwidth]{tNp/tnpCounting} &
          \includegraphics[width=.45\textwidth]{tNp/tnpFromFit}
        \end{array} $ 
        \caption{Tag and probe trigger efficiencies from counting (left) 
          compared to fitting (right).}
        \label{fig:tnpCntVFit}
      \end{figure}
  
      From Fig.~\ref{fig:tnpCntVFit} it is apparent that the choice of fit 
        function, and, therefore, the amount of background from the mass side 
        bands is included in the signal measurement has very little effect on 
        the tag and probe efficiency measurement.
      The small effect of including the side bands are due to the side bands 
        being comprised mostly of photon-photon events.
      Because this background is neither decays from other particles like pions,
        nor is it from non-physical background like combinatorics, 
        the efficiency for muons from the side bands are nearly identical to
        \JPsi{} signal.
      The photon-photon process directly produces two muons just like the 
        \JPsi{}, therefore efficiency estimated from the side bands has 
        little effect on the measurement because of this similarity.
      The counting and fitting trigger efficiency measurements agree within 
        statistical uncertainties, so this uncertainty was taken to be 
        negligibly.
  
    \subsection{The effect of noise on ZDC trigger efficiency estimates}
      The amount of electronic noise in the sample effects the ZDC energy 
        distribution and, therefore, the trigger efficiency measurement.
      The more noise that sits below the one neutron peak, the lower the 
        efficiency estimate will be. 
      In Table~\ref{tab:zdcEfficiencySysNoiseSample}, the zero bias sample 
        with the timing cuts desribed in Section~\ref{sec:zdcCompare} gives a 
        significantly hifher 
        estimated efficiency compared the zero bias sample with out timing cuts
        in Table~\ref{tab:zdcEfficiencySysNoiseSample}.
      The same increase is seen when comparing the ZDC triggered sample with 
        the ZDC triggered sample that also requires a pixel track shown in
        Table~\ref{tab:zdcEfficiencySys}. 
      The effect of the electronic noise is also present in the difference seen
        in using the two methods.
      As seen in Fig.~\ref{fig:zdcSpec2v1}, the new reconstruction method 
        shows better separation of the one neutron peak from the electronic 
        noise, in particular in ZDC$^{+}$ where the signal gain is lower.
      For this reason, the zero bias data, which contains the largest 
        contribution from electronic noise, shows the most separation between 
        the two methods and gives the lowest estimate for the ZDC trigger 
        efficiency.
      \begin{table}
        \centering
        \begin{tabular}{|c|c|c|c|c|}
          \hline ZDC Side & Reco Method & N$_{events}$ & N$_{trig}$ & $\varepsilon_{ZDC}$ \\ \hline
          \multicolumn{5}{|c|}{ Zero bias with ZDC timing cuts} \\ \hline 
           ZDC$^{-}$ & comparison & 88676  & 84429  & 0.9521 $\pm$ 0.0046 \\ \hline
           ZDC$^{-}$ & nominal & 88480  & 84202  & 0.9517 $\pm$ 0.0046 \\ \hline
           ZDC$^{+}$ & comparison & 59878  & 54728  & 0.9140  $\pm$ 0.0054 \\ \hline
           ZDC$^{+}$ & nominal & 60467  & 54733  & 0.9052  $\pm$ 0.0053 \\ \hline
           \multicolumn{5}{|c|}{(ZDC$^{-}$ or ZDC$^{+}$)} \\ \hline 
           ZDC$^{-}$ & comparison & 30986 & 30333 & 0.9789 $\pm$ 0.0079 \\ \hline
           ZDC$^{-}$ & nominal & 31029 & 30339 & 0.9778 $\pm$ 0.0079 \\ \hline
           ZDC$^{+}$ & comparison & 39178 & 30164 & 0.7699 $\pm$ 0.0059 \\ \hline
           ZDC$^{+}$ & nominal & 35703 & 30443 & 0.8527 $\pm$ 0.0067 \\ \hline
           \multicolumn{5}{|c|}{ Zero bias} \\ \hline 
           ZDC$^{-}$ & comparison & 109967  & 101598  & 0.9239 $\pm$ 0.0040 \\ \hline
           ZDC$^{-}$ & nominal & 110230  & 101561  & 0.9214 $\pm$ 0.0040 \\ \hline
           ZDC$^{+}$ & comparison & 253241  & 86660  & 0.3422 $\pm$ 0.0013 \\ \hline
           ZDC$^{+}$ & nominal & 156336  & 87401  & 0.5591 $\pm$ 0.0024 \\ \hline
         \end{tabular}
        \caption{ZDC trigger efficiencies using  the nominal and comparison 
        ZDC reconstructions for trigger sample that do no require a pixel track.}
        \label{tab:zdcEfficiencySysNoiseSample}
      \end{table}

\chapter{Results and summary}
  In the previous two chapters the analysis steps to measure the coherent
    \JPsi{} photoproduction were given.
  In this chapter, the measured cross section for this process is given
    in Section~\ref{sec:jpCoRes}.
  Comparisons to the theoretical models are discussed. 
  In addition, the rapidity correlation between UPC \JPsi{} and forward
    neutrons are presented and compared to model calculations in 
    Section~\ref{sec:upcCor}. 
  Finally, a summary of the complete thesis is given.  
  
  \section{Coherent \JPsi{} cross section\label{sec:jpCoRes}}
%
 The coherent \JPsi{} cross section is calculated using the following formula:
%
\begin{equation}
\frac{d\sigma^{J/\psi}_{coh}}{dy} (Xn0n)= \frac{N^{J/\psi}_{coh}}{\mathcal{L} \cdot (2 \cdot \Delta y )\cdot BR}~\textrm{,}
\label{eq:expXSecCo}
\end{equation}
%
where $N^{J/\psi}_{coh}$ is the corrected number of coherent $J/\psi$ candidates,  $\mathcal{L}$ is the integrated luminosity used for
this analysis, $\Delta y$ is the width of the rapidity interval, and $BR$ is the branching ratio for \JPsi{} to $\mu^{+}\mu^{-}$. To obtain $N^{J/\psi}_{coh}$ the following
formula was used  
%
\begin{equation}
N^{J/\psi}_{coh}(Xn0n) = \frac{N^{J/\psi}_{yield}}{(A\times \varepsilon) \cdot \varepsilon^{ZDC}_{trig} \cdot \varepsilon^{J/\psi}_{trig}}~\textrm{,}
\end{equation}
%
where $N^{J/\psi}_{yield}$ is the $J/\psi$ yield with a $p_{T} <$ 0.15 GeV, $A\times \varepsilon$ is the combined acceptance and efficiency, $\varepsilon^{ZDC}_{trig}$ is the ZDC triggering efficiency for detecting at least one neutron emitted in the forward region, and $\varepsilon^{J/\psi}_{trig}$ is the $J/\psi$ trigger efficiency measured by the ``tag and probe" method. Each of these quantities can be found in Table~\ref{tab:nJpCoh}.
%
\begin{table}
  \centering
  \begin{tabular}{|c|c|} \hline 
    $N^{J/\psi}_{yield}$ & \textcolor{red}{X} \\ \hline  
    $A\times \varepsilon$ & \textcolor{red}{X} \\ \hline
    $\varepsilon^{ZDC}_{trig}$ & \textcolor{red}{X} \\ \hline
    $\varepsilon^{J/\psi}_{trig}$ & \textcolor{red}{X} \\ \hline \hline 
    $N^{J/\psi{}}_{coh} (Xn0n)$ & \textcolor{red}{X} \\ \hline 
  \end{tabular}
  \caption{\label{tab:nJpCoh}}
\end{table}  

\begin{table}
  \centering
  \begin{tabular}{|c|c|} \hline 
    $N^{J/\psi}_{coh}$ & \textcolor{red}{X} \\ \hline  
    $\mathcal{L}$ & \textcolor{red}{X} \\ \hline
    $BR$ & \textcolor{red}{X} \\ \hline
    $\Delta y$ & 0.2 \\ \hline \hline 
    $\frac{d\sigma^{J/\psi}_{coh}}{dy} (Xn0n)$ & \textcolor{red}{X} \\ \hline 
  \end{tabular}
  \caption{\label{tab:nJpCoh}}
\end{table}  

The coherent J/$\psi$ cross section $\frac{d\sigma^{J/\psi}_{coh}}{dy} (Xn0n)$ includes the emission of at least one neutron emitted, Xn0n. Previous measurements carried 
by the ALICE Collaboration measured the cross section without any break-up requirement. In order to compare to their results, the cross section is scaled by the 
factor 5.06 to account for the increase in cross section. This value was obtained from STARlight. The resulting cross section is \textcolor{blue}{X $\pm$ Y}.

Figure~X shows the measured cross section compared to the ALICE data points, as well as STARligh, AB and LTA models (see Chapter~\ref{ch:photoNuc}. The measured cross section is in agreement with previous measurements, and favors the AB-EPS09 model. Given the measured cross section is significantly below STARlight, nuclear gluon shadowing plays an important role for $x\sim$ 10$^{-2}$. 

  \section{\JPsi{} neutron correlations~\label{sec:upcCor}}
    In this section the correlation between the rapidity of the $\mu^{+}\mu^{-}$ 
      and of the neutron is studied. The following samples are studied: 
    \begin{itemize}
      \item $\gamma + A$ collisions in which two cases are considered
      \begin{itemize}
        \item elastic coherent interaction: here photon interacts with entire
          nucleus coherently and produce \JPsi. 
          Another photon is needed to cause the breakup and neutron emission. 
          Those two photons are uncorrelated and thus we don't expect to 
            observe the correlation between the rapidity of the neutron and the
            rapidity of the \JPsi.
          In the data sample this corresponds to the low-\pt \JPsi 
            (\pt$<$0.15~GeV). 
        \item inelastic incoherent interaction: here a single high \pt photons
          interacting with nucleus produce the \JPsi and neutron. 
          The correlation between the rapidity of the neutron and the rapidity
            of the \JPsi is expected.
          In the data sample this corresponds to the high-\pt \JPsi 
            (0.15$<$\pt$<$1.05). 
      \end{itemize}
      \item $\gamma \gamma$ collisions: two photons collide and produce the 
        $\mu^{+}\mu^{-}$  and the third photon is needed to excite one of
          the nucleons and produce neutron. 
        Thus we don't expect to see the correlation between the rapidity of the
          neutron and the rapidity of the $\mu^{+}\mu^{-}$. 
        In the data sample this corresponds to the $\mu^{+}\mu^{-}$ with the
          invariant mass between 4 and 8~GeV. 
    \end{itemize}

    In order to study the correlation in rapidity between the neutron and
      dimuon direction we below four quantities and give they values in 
      Table~\ref{tab:corrneutronjpsi}.  
    \begin{itemize}
      \item $y^{-}_{\mu\mu} \wedge y_{n}^{-}$: number of $\mu^{+}\mu^{-}$ having
         $y<0$ and the neutron in ZDC$^{-}$ ($y<0$)
      \item $y^{-}_{\mu\mu} \wedge y_{n}^{+}$: number of $\mu^{+}\mu^{-}$ having
         $y<0$ and the neutron in ZDC$^{+}$ ($y>0$)
      \item $y^{+}_{\mu\mu} \wedge y_{n}^{+}$: number of $\mu^{+}\mu^{-}$ having
         $y>0$ and the neutron in ZDC$^{+}$ ($y>0$)
      \item $y^{+}_{\mu\mu} \wedge y_{n}^{-}$: number of $\mu^{+}\mu^{-}$ having
         $y>0$ and the neutron in ZDC$^{-}$ ($y<0$)
    \end{itemize}

    The ratio $R_{opp/same}$ is defined as: 
    \begin{equation}
      R_{opp/same} = \frac{y^{-}_{\mu\mu} \wedge y_{n}^{+} + y^{+}_{\mu\mu} 
        \wedge y_{n}^{-}}{y^{-}_{\mu\mu} \wedge y_{n}^{-} + y^{+}_{\mu\mu} 
        \wedge y_{n}^{+}}
    \end{equation}
    
    Ratios studied in this section are only sensitive to the difference between
      the ZDC$^{-}$ and ZDC$^{-}$.
    It is seen that the efficiency of both ZDCs in not exactly the same i.e.
      the efficiencies of ZDC$^{-}$ and ZDC$^{+}$ are respectively 
      $\varepsilon_{ZDC^{-}}$=0.98 and  $\varepsilon_{ZDC^{+}}$=0.94 and this 
      is taken in the account in the estimations. 
    The $R_{opp/same}$ radio corrected by the ZDCs efficiencies is also 
      included in Table~\ref{tab:corrneutronjpsi} and called 
      $R_{opp/same}^{\varepsilon_{ZDC}}$. 
    It is seen that the difference between corrected and uncorrected results is
      very small. 
    Other uncertainties cancel. 
    In this case cuts related to the acceptance and efficiencies corrections 
      are not necessary and thus they are released.
    
    Figure~\ref{fig:PtcorrandRation} gives \pt distributions of the \JPsi 
      when \JPsi and neutron have the opposite rapidity direction or when they 
      have the same rapidity direction for low-\pt and high-\pt \JPsi. 
    Also the Fig~\ref{fig:PtcorrandRation} gives the $R_{opp/same}$ for low-\pt
      and high-\pt \JPsi. 
    In is send from this plot that in the case of the low-\pt \JPsi this 
      $R_{opp/same}$ ratio is close to 1 and is decreasing when the \pt of
      \JPsi increases.
  
    \begin{figure*}[!Hhtb]
      \begin{center}
        \includegraphics[angle=0,width=0.45\textwidth]{cohOppAndSameNeutronDir}
        \includegraphics[angle=0,width=0.45\textwidth]{incohOppAndSameNeutronDir} \\
        \includegraphics[angle=0,width=0.45\textwidth]{ratiocohOppAndSameNeutronDir}
        \includegraphics[angle=0,width=0.45\textwidth]{ratioincohOppAndSameNeutronDir}
      \caption{ \label{fig:PtcorrandRation}  
        Transverse momentum distribution of the $J/\psi$ when  $J/\psi$ and 
          neutron have the opposite rapidity direction and the transverse 
          momentum distribution of the $J/\psi$ when  $J/\psi$ and neutron
          have the same rapidity direction for low-\pt (top left) and 
          high-\pt (top right) \JPsi. Bottom: Ratios $R_{opp/same}$ for 
          low-\pt ( left) and high-\pt ( right) \JPsi.}
      \end{center}
    \end{figure*}
    
    Compiled for \pt$<$1.05~GeV $R_{opp/same}$ ratio between the \pt 
      distribution of the \JPsi having neutron emitted in the opposite 
      direction and  the \JPsi having the neutron emitted in the same
      direction is shown on Fig.~\ref{fig:r2}. 
    \begin{figure*}[!Hhtb]
      \begin{center}
        \includegraphics[angle=0,width=0.65\textwidth]{RoppsameVsTheory}
        \caption{ \label{fig:r2} Ratio between the transverse momentum 
          distribution of the $J/\psi$ when  $J/\psi$ and neutron have 
          the opposite direction and the transverse momentum distribution 
          of the $J/\psi$ when  $J/\psi$ and neutron have the same direction.}
      \end{center}
    \end{figure*}
    The same distributions as~\ref{fig:PtcorrandRation} but now as a function 
      of rapidity of the \JPsi are presented in the Fig~\ref{fig:r3} and 
      compiled in Fig.~\ref{fig:r4}. 
    
    \begin{figure*}[!Hhtb]
      \begin{center}
        \includegraphics[angle=0,width=0.45\textwidth]{cohyOppAndSameNeutronDir}
        \includegraphics[angle=0,width=0.45\textwidth]{incohyOppAndSameNeutronDir}\\
        \includegraphics[angle=0,width=0.45\textwidth]{ratiocohyOppAndSameNeutronDir}
        \includegraphics[angle=0,width=0.45\textwidth]{ratioincohyOppAndSameNeutronDir}
        \caption{ \label{fig:r3} Rapidity distribution of the $J/\psi$ when  
          $J/\psi$ and neutron have the opposite rapidity direction and the 
          rapidity distribution of the $J/\psi$ when  $J/\psi$ and neutron 
          have the same rapidity direction for low-\pt (top left) and high-\pt 
          (top right) \JPsi. Bottom: Ratios $R_{opp/same}$ for low-\pt ( left) 
          and high-\pt ( right) \JPsi.}
      \end{center}
    \end{figure*}
    
    \begin{figure*}[!Hhtb]
      \begin{center}
        \includegraphics[angle=0,width=0.55\textwidth]{compiledratioincohandcohyOppAndSameNeutronDir}
        \caption{ \label{fig:r4} Rapidity ratios $R_{opp/same}$ for low-\pt 
          ( left) and high-\pt ( right) \JPsi.}
      \end{center}
    \end{figure*}
    
    Figure~\ref{fig:rNeutDimuCorr} shows the rapidity of the dimuon for the 
      events that are tagged by the ZDC$^{+}$ and  ZDC$^{-}$ means having 
      the neutron emitted in the $y>0$ and $y<0$. 
    
    \begin{figure*}[!Hhtb]
      \begin{center}
        \includegraphics[angle=0,width=0.42\textwidth]{ZDCDimuCorrCoh}\\
        \includegraphics[angle=0,width=0.42\textwidth]{ZDCDimuCorrIncoh}\\
        \includegraphics[angle=0,width=0.42\textwidth]{ZDCDimuCorrGammaGamma}
        \caption{ \label{fig:rNeutDimuCorr} Rapidity distribution of \JPsi in the
          case of the events having the neutron in negative and positive rapidity 
          for the low-\pt \JPsi (top), high-\pt \JPsi (middle) and dimuons from
          $\gamma \gamma$ sample (bottom). }
      \end{center}
    \end{figure*}
    
    Another, interesting correlation between the \JPsi rapidity direction and 
      the neutron rapidity can be also studied with quantities defined in 
      Eq.~\ref{eq:Ration} that are calculated in the 
      Table~\ref{tab:corrneutronjpsi}. 
    Table~\ref{tab:corrneutronjpsieffcorr} gives the same quantities as  
      Table.~\ref{tab:corrneutronjpsi} but here it is corrected for the 
      difference between the efficiency of the ZDC$^{+}$ and  ZDC$^{-}$. 
    
    \begin{equation}
      \label{eq:Ration}
      R_{(\mu\mu)^{-}}^{n^{-}/n^{+}} =  \frac{y^{-}_{\mu\mu} \wedge 
        y_{n}^{-}}{y^{-}_{\mu\mu} \wedge y_{n}^{+} }~~~~~~~~\mbox{and}~~~~~
        R_{(\mu\mu)^{+}}^{n^{-}/n^{+}} =  \frac{y^{+}_{\mu\mu} \wedge 
        y_{n}^{-}}{y^{+}_{\mu\mu} \wedge y_{n}^{+} }
    \end{equation}
    
    \begin{table}[h]
      \begin{center}
        \begin{tabular}{|c|c|c|c|c|c|c|}
          \hline
          & $y^{-}_{\mu\mu} \wedge y_{n}^{-}$ & $y^{-}_{\mu\mu} \wedge y_{n}^{+}$ 
          & $y^{+}_{\mu\mu} \wedge y_{n}^{+}$ & $y^{+}_{\mu\mu} \wedge y_{n}^{-}$ 
          & $R_{(\mu\mu)^{-}}^{n^{-}/n^{+}}$ 
          & $R_{(\mu\mu)^{+}}^{n^{-}/n^{+}} $  \\ \hline
          low-\pt \JPsi & \textcolor{blue}{78 $\pm$ 8.8} 
          & \textcolor{blue}{47 $\pm$  6.8}  & \textcolor{blue}{81 $\pm$  9}  
          & \textcolor{blue}{74 $\pm$  8.6} & \textcolor{blue}{ 1.66$\pm$0.31  } 
          & \textcolor{blue}{ 0.91$\pm$0.15  } \\ \hline
          high-\pt \JPsi & \textcolor{blue}{132 $\pm$  11.5}  
          & \textcolor{blue}{17 $\pm$  4.1}  & \textcolor{blue}{117 $\pm$  10.8}  
          & \textcolor{blue}{29 $\pm$  5.4} & \textcolor{blue}{7.76$\pm$2.0} 
          & \textcolor{blue}{0.25$\pm$0.05 } \\ \hline
          $\mu^{+}\mu^{-}$ from $\gamma \gamma$ & \textcolor{blue}{80 $\pm$8.9} 
          & \textcolor{blue}{81 $\pm$9} & \textcolor{blue}{75 $\pm$8.7} 
          & \textcolor{blue}{83 $\pm$9.1} & \textcolor{blue}{0.99$\pm$0.16 } 
          & \textcolor{blue}{1.11$\pm$0.18} \\ \hline
        \end{tabular}
        \caption{\label{tab:corrneutronjpsi} Number of dimuon pairs for 
          different directions of the neutron rapidity direction together with 
          $R_{(\mu\mu)^{-}}^{n^{-}/n^{+}}$ and 
          $R_{(\mu\mu)^{+}}^{n^{-}/n^{+}}$.}
      \end{center}
    \end{table}
    
    \begin{table}[h]
      \begin{center}
        \begin{tabular}{|c|c|c|}
          \hline
          & $R_{(\mu\mu)^{-}}^{\varepsilon_{ZDC}(n^{-}/n^{+})}$ 
          & $R_{(\mu\mu)^{+}}^{\varepsilon_{ZDC}(n^{-}/n^{+})} $  \\ \hline
          low-\pt \JPsi &  \textcolor{blue}{1.59 $\pm$ 0.29} 
          & \textcolor{blue}{0.88 $\pm$ 0.14} \\ \hline
          high-\pt \JPsi  & \textcolor{blue}{7.45 $\pm$ 1.87}  
          &  \textcolor{blue}{0.24 $\pm$ 0.05 } \\ \hline
          $\mu^{+}\mu^{-}$ from $\gamma \gamma$ 
          & \textcolor{blue}{0.95 $\pm$ 0.15} 
          & \textcolor{blue}{ 1.06 $\pm$ 0.17 } \\ \hline
        \end{tabular}
        \caption{\label{tab:corrneutronjpsieffcorr} Ratios 
          $R_{(\mu\mu)^{-}}^{\varepsilon_{ZDC}(n^{-}/n^{+})}$ and 
          $R_{(\mu\mu)^{+}}^{\varepsilon_{ZDC}(n^{-}/n^{+})} $ 
          i.e. $R_{(\mu\mu)^{-}}^{n^{-}/n^{+}}$ and 
          $R_{(\mu\mu)^{+}}^{n^{-}/n^{+}} $ corrected by the ZDC$^{+}$ and
          ZDC$^{-}$ efficiencies.}
      \end{center}
    \end{table}
    
    Integrated over rapidity, separately for $y<0$ and $y>0$ ratios from 
      Table~\ref{tab:corrneutronjpsieffcorr} are shown in the 
      Figure~\ref{fig:integRatios}.
    
    \begin{figure*}[!Hhtb]
      \begin{center}
        \includegraphics[angle=0,width=0.52\textwidth]{RCompiledYCorr}
        \caption{ \label{fig:integRatios} 
          $R_{(\mu\mu)^{-}}^{\varepsilon_{ZDC}(n^{-}/n^{+})}$ and 
          $R_{(\mu\mu)^{+}}^{\varepsilon_{ZDC}(n^{-}/n^{+})}$ integrated over 
          one side in rapidity for low- and high-\pt \JPsi and also for dimuons
          from $\gamma \gamma$ sample. }
      \end{center}
    \end{figure*}
    
    From the Tab~\ref{tab:corrneutronjpsi} and the Fig.~\ref{fig:rNeutDimuCorr} 
      it is seen as expected that there is no correlation between the \JPsi 
      rapidity and neutron rapidity in the case of the low-\pt \JPsi and 
      dimuons coming from $\gamma \gamma$ sample. 
    In the case of the high-\pt \JPsi the correlation is clearly visible. 

\iffalse
  \section{Incoherent cross section}
  The same basic procedure for measuring the coherent cross section was used to 
    calculate the incoherent cross section.  
  \section{Break up ratios}
    In Table~\ref{tab:r2} the ratio between raw yields for different break up 
      modes are shown.
    \begin{figure}[!Hhtb]
      \centering
      \includegraphics[width=.6\textwidth]{coherentBreakup}
      \caption{Ratio between J/$\psi$ yeilds XnXn and 1n0n break-up modes 
        compared the Xn0n break-up mode for J/$\psi$ with $p_{T}$ below 150 
        MeV.}
      \label{fig:coherentBreakUp}
    \end{figure}
   
    Fig.~\ref{fig:coherentBreakUp} and Fig.~\ref{fig:incoherentBreakUp} compare
      the raw break up ratios two STARlight and LTA predictions. 
    \begin{figure}[!Hhtb]
      \centering
      \includegraphics[width=.6\textwidth]{incoherentBreakup}
      \caption{Ratio between J/$\psi$ yeilds XnXn and 1n0n break-up modes 
        compared the Xn0n break-up mode for J/$\psi$ with 0.2 $< p_{T} <$ 
        1.5 GeV.}
      \label{fig:incoherentBreakUp}
    \end{figure}

    The number of the coherent and incoherent J/$\psi$ for each break-up mode are
      given in the Tab.~\ref{tab:r1}. 
    The ratios between the modes X$_{n}$X$_{n}$, 1$_{n}$0$_{n}$, 1$_{n}$1$_{n}$ and
      the mode  X$_{n}$0$_{n}$ are given in the table Tab.~\ref{tab:r2}. 
    Some of the  ratios can be obtained from  {\sc starlight} and from the Zhalov 
      and thus are given in Tab.~\ref{tab:r3}.
    \begin{table}[h]
      \begin{center}
      \begin{tabular}{|c|c|c|c|c|c|}
        \hline
         &  X$_{n}$0$_{n}$& X$_{n}$X$_{n}$ & 1$_{n}$0$_{n}$ & 1$_{n}$1$_{n}$  \\ \hline
        coherent J/$\psi$ &  242$\pm$16&94$\pm$10&58$\pm$8&8$\pm$3\\ \hline
        incoherent J/$\psi$ & 291$\pm$17&57$\pm$8&19$\pm$4&2$\pm$1\\ \hline
      \end{tabular}
      \caption{\label{tab:r1} Number of coherent J/$\psi$ integrated over $p_{T}$ and $y$ 
        with statistical uncertainty.}
      \end{center}
    \end{table}
    
    \begin{table}[h]
      \begin{center}
        \begin{tabular}{|c|c|c|c|c|}
          \hline
          & X$_{n}$X$_{n}$/X$_{n}$0$_{n}$ & 1$_{n}$0$_{n}$/X$_{n}$0$_{n}$ & 1$_{n}$1$_{n}$/X$_{n}$0$_{n}$  \\ \hline
          coherent J/$\psi$ &  0.39$\pm$0.05&0.24$\pm$0.04&0.03$\pm$0.01\\ \hline
          incoherent J/$\psi$ &  0.20$\pm$0.03&0.07$\pm$0.02&0.007$\pm$0.005 \\ \hline
        \end{tabular}
      \caption{\label{tab:r2} Number of coherent J/$\psi$ integrated over $p_{T}$ and $y$ 
        with statistical uncertainty.}
      \end{center}
    \end{table}

    In Table~\ref{tab:r3} the ratio between break up modes are shown for 
      different theories and processes.
    \begin{table}[h]
      \begin{center}
        \begin{tabular}{|c|c|c|c|c|}
          \hline
          & X$_{n}$X$_{n}$/X$_{n}$0$_{n}$ & 1$_{n}$0$_{n}$/X$_{n}$0$_{n}$ & 1$_{n}$1$_{n}$/X$_{n}$0$_{n}$  \\ \hline
          STARlight coherent &  0.37&-&0.02\\ \hline
          Zhalov coherent& 0.32&0.30&0.02\\ \hline
          STARlight incoherent &  0.37&-&0.007$\pm$0.02 \\ \hline
        \end{tabular}
        \caption{\label{tab:r3} Number of  J/$\psi$ integrated over $p_{T}$ and $y$ with 
          statistical uncertainty.}
      \end{center}
    \end{table}
\fi

  \section{\label{sec:summary}Summary}
    As physicists' understanding of the QGP has deepened over the past 30 years
      of doing experimental heavy ion physics, the questions surrounding the
      QGP have shift from the confirmation of creation of a deconfined state
      to understanding the properties of the state that is created.
    It appears that the QGP is a hot nearly viscosity free fluid of strongly 
      coupled quarks and gluons. 
    The control measurements from dAu collisions at RHIC, and pPb collisions
      at the LHC have shown signs of collective behavior such as flow, which 
      made these results difficult to interpret.
    Additional knowledge of the initial state of the colliding nuclei is needed
      in order to fully understand the QGP signal seen in PbPb and AuAu 
      collisions.
    UPC events can provide this needed knowledge. 
    This thesis contributes to the understanding of the initial state through
      the measurement of the UPC \JPsi{} photoproduction cross section. 
  
    Ultra-peripheral collisions are clean probe of the initial state. 
    In UPC \JPsi{} photoproduction, the nuclei interact through the 
      electromagnetic force precluding the possibility of creating a collective
      medium.
    The theoretical models of coherent UPC \JPsi{} photoproduction model these 
      electromagnetic interactions by combining a semi-classical calculation 
      of the photon flux with a variety of phenomenological and QCD based 
      calculations of the nuclear gluon density. 
    The Weis\"{a}cker-Williams approximation \cite{WWFermi} is used to calculate 
      the flux of photons that surround the colliding nuclei. 
    The interaction of these photons with the nucleus is either calculated 
      through a nuclear modification of the proton gluon density 
      \cite{pQCD2013.02, lta2012.03}, or by using the Glauber model approach 
      that consists in modeling the nucleus as a collection
      of nucleons and scaling the nucleon photoproduction cross sections from 
      e-p collisions \cite{vmd1999}. 
    Photoproduction cross sections from from UPC events can determine at what 
      energy scale the Glauber based method breaks down.
    For the gluon density based calculations, there is a wide discrepancy
      between the predictions, and photoproduction cross sections constrain which
      gluon density models are viable. 
  
    In this thesis, the CMS detector was used to measure the coherent UPC \JPsi{} 
      photoproduction cross section.
    The three major subsystems of CMS were used, the tracker, 
      the muon system, the calorimeter system.
    The tracker records the position charge excitations in the silicon due to 
      particle hits, which are used to reconstruct the trajectory of charged 
      particles. 
    The muon system is comprised of the three gaseous detectors, the DTs, the 
      RPC, and the CSCs, which record charge deposits as high momentum particles
      ionize the gas within the detector.
    The muon system primary purpose is for triggering on and identifying muons.
    The calorimeter system measures the energy of deposited by particle induced
      showers as a means of reconstructing neutral particles and jets.
    
    The analysis in this thesis consists of three major components, development
      of a trigger, estimation of efficiency, and measurement of signal events.
    The trigger development involved designing a trigger based on rate estimates
      from past data that ensured a sample that could be used for both measuring
      the signal and estimate the efficiency of the trigger, reconstruction, and
      event selection.
    The number of \JPsi{} candidates was measured by first applying a set 
      of event selection cuts that rejected background events such as hadronic
      collisions and beam gas collisions, then fitting the remaining events to
      templates from simulation to separate the three remaining physics processes,
      the coherent, incoherent, and photon-photon process.
    The efficiencies for each part of the trigger were measured from data. 
    The acceptance and reconstruction efficiency were estimated from MC.
    The cross section was calculated by combining the efficiency with the 
      measured luminosity and number of coherent \JPsi{}.
    The statistical uncertainties were taken from the template fit.
    The systematic uncertainties were estimated by varying the method used on 
      each component of the analysis. 
  
    The UPC \JPsi{} photoproduction cross section, $\frac{d\sigma^{J/\psi}_{co}}{dy}$,
      was found to be \textcolor{red}{368 $\pm$ 38 $\mu$b}. 
    When rescaled by a factor of 5.06 to account for the difference of break-up mode between 
      the measurement in this thesis and the ALICE result in~\cite{Abelev:2012ba,Abbas:2013oua}, the 
      result of \textcolor{red}{1.86 $\pm$ 19 mb} was found to be consistent with the 
      predictions in \cite{pQCD2013.02} of \textcolor{red}{1.8 mb}.  
    The calculation in \cite{pQCD2013.02} is also favored by the ALICE measurements. 
    Of the gluon distributions used in \cite{pQCD2013.02}, a gluon distribution with 
      moderately strong gluon shadowing, EPS09~\cite{Eskola:2009uj}, is consistent with both the results from this
      thesis and the previous ALICE results. 
    This indicates that at the scale of the mass of the \JPsi{} the nucleus gluon density is 
      significantly suppressed compared to the gluon densities of a nucleon.
    At this scale the nucleus can not be represented as a collection of nucleons as in the 
      Glauber like model described in \cite{vmd1999}.
  
    The measurement in this thesis confirms the ability to increase the knowledge of the 
      initial state through the exploration of UPC events. 
    This confirmation opens the door to additional measurements in this growing field of UPC
      research.
    A whole host of measurements will be possible with the data already recorded,
      and with that to recorded in the coming years by CMS and the other LHC 
      experiments. 

\global\long\def\bibname{References}

\bibliographystyle{bbl/utphys}

\bibliography{bbl/upcTheoryChp}

%\appendix
%\include{Appendix1/Appendix1}

\end{document}
