\chapter{Analysis}	
  \section{\label{sec:mcSim} MC Simulation}
    Monte Carlo (MC) simulations are used to understand how the detector 
      effects the measurement.
    Two main classes of Monte Carlo (MC) simulation samples were used. 
    The first class uses STARlight to generate events. 
    The Second class uses PYTHIA6 to create and decay $J/\psi$s with a given
      input $p_{T}$ and rapidity distribution.
    
  \section{\label{sec:TrigDev} Trigger Development} 
    Prior to the 2011 LHC PbPb run, UPC events had not been directly studied in 
      PbPb collisions using CMS. 
    Design of the UPC triggers required studies of the 2010 data to estimate 
      rates and insure that the bandwidth used by these trigger would be
      sufficiently low. 
    All the different physics analyses must share the limited readout rate of 
      the detector.
    For this reason, conservation of bandwidth was a major design consideration.

    To estimate the 2011 rates prior to the run, the 2010 rates were used to 
      extrapolate to the interaction rate of the 2011 run. 
    The unique UPC triggers were estimated by combining existing triggers from
      the 2010 run. 
    By calculating the ratio between the UPC trigger rates and the minimum bias
      trigger rate, the UPC trigger rates were scaled up to the 2011 
      interaction rates using the 2010 data. 
    The extrapolated rates allowed for a package triggers to be created, which 
      fit within the bandwidth requirement of CMS Heavy Ions. 
    
    The trigger package for 2011 contained ZDC based efficiency monitoring 
      triggers, muon and electron based triggers for measuring $J/\psi$, and 
      backup triggers in case there was a problem with the original muon and 
      electron triggers.
    In order to recorded the trigger efficiency monitoring data, the ZDC 
      triggers had to be prescaled to a lower rate. 
    The scaling down of the monitoring triggers were setup to insure overlap
      with the signal triggers.
    By balancing the competing objectives of rate reduction and increasing 
      the overlap between the monitoring and signal triggers, 
      the prescales for the trigger were as seen in Table .%~\ref{triggerTabel2011}.

    \subsection{\label{sec:l1Trigger} L1 Trigger}
      The goal of the L1 triggers was to record enough data to measure dimuons
        and dielectrons in UPC events.
      To achieve this, the loosest muon trigger and lowest threshold ECAL 
        triggers where paired with a trigger on energy in the ZDC and a veto on
	      energy in the BSC or HF.
      The L1 package that was constructed for the analysis of UPC $J/\psi$ 
        is presented in Table~\ref{tab:l1Triggers2011}.

      \begin{table}[h]
        \centering
        \begin{tabular}{|l|l|}
          L1 Trigger Seed  & Type \\ \hline \hline
          L1\_MuOpen\_ZdcCalo\_NotBscMinBiasThresh2\_BptxAND & Physics \\  \hline
          L1\_EG2\_ZdcCalo\_NotBscMinBiasThresh2\_BptxAND & Physics \\  \hline
          L1\_EG5\_ZdcCalo\_NotBscMinBiasThresh2\_BptxAND & Physics \\ \hline
          L1\_ZdcCaloMinus\_BptxAND & Monitor \\  \hline
          L1\_ZdcCaloMinus\_BptxAND & Monitor \\  \hline
          L1\_MuOpen\_ZdcCalo\_NotHcalHfCoincidencePm\_BptxAND & Backup \\ \hline
          L1\_EG2\_ZdcCalo\_NotHcalHfCoincidencePm\_BptxAND & Backup \\ \hline
          L1\_EG5\_ZdcCalo\_NotHcalHfCoincidencePm\_BptxAND & Backup \\ \hline \hline
        \end{tabular}
        \caption{List of 2011 L1 seeds.}
        \label{tab:l1Triggers2011}
      \end{table}

      The cumulative L1 trigger rate for all the UPC L1 trigger seeds was
        required to be 200 Hz.
      This requirement stemmed from the need to keep the tracker read-out rate
        low. 
      The trackers baseline voltage can fluctuate due to the high tracker hit 
        multiplicities in PbPb collisions.
      In order to monitor the zero suppression of the tracker, the zero 
        suppression algorithm was executed using the HLT computing farm 
	      rather than the in the tracker firmware. 
      The additional computing cycles needed to run the zero suppression 
        the limit for L1 bandwidth. 

    \subsection{HLT Trigger}
      As opposed to the L1 trigger, which reads out the tracker, the HLT has 
        access to the tracker information. 
      Reconstruction of a track in the pixel detector is used by the UPC paths.
      The use of the pixel detector only, as opposed to using the whole tracker 
        including the silicon strip detector, allows for quick track 
	reconstruction saving computing cycles.
      The requirement of at least on reconstructed pixel track for the HLT 
        triggers was designed to reject backgrounds where no particles are 
	reconstructed by the tracker. 
  \begin{table}[h]
		\centering
		\begin{tabular}{|l|l|}
		  \hline HLT Trigger  \\ \hline \hline
		  HLT\_HIUPCNeuMuPixel\_SingleTrack & Physics   \\ \hline
		  HLT\_HIUPCNeuEG2Pixel\_SingleTrack & Physics   \\ \hline
		  HLT\_HIUPCNeuEG5Pixel\_SingleTrack & Physics   \\ \hline
		  HLT\_HIMinBiasZDC\_Calo\_PlusOrMinus\_v1  & Monitor  \\ \hline
		  HLT\_HIMinBiasZDC\_PlusOrMinusPixel\_SingleTrack\_v1   & Monitor \\ \hline
		  HLT\_HIUPCNeuHcalHfMuPixel\_SingleTrack & Backup   \\ \hline
		  HLT\_HIUPCNeuHcalHfEG2Pixel\_SingleTrack & Backup   \\ \hline
		  HLT\_HIUPCNeuHcalHfEG5Pixel\_SingleTrack & Backup   \\ \hline \hline
		\end{tabular}
		\caption{List of 2011 HLT trigger.}
		\label{tab:hltTriggers2011}
	\end{table}

	The total HLT output for the UPC trigger paths was 20 Hz. 
	The limiting factor for the HLT rate was the amount of disk space 
	  available to store the data. 

  \section{\label{sec:DataSetEvSel} Data Sets and Event Selection}
    \subsection{Data Set}
      In order to investigate novel physics processes like UPC $J/\psi$ 
       production, the LHC has delivered unprecedented amounts of data.
      The data for this analysis was recorded during the 2011 LHC PbPb run. 
      During this period, 150 $\mu$$b^{-1}$ where recorded by the CMS detector,
        corresponding to over a billion PbPb collisions. 
      Of this, 143 $\mu$$b^{-1}$ are used in this analysis.
  
      Three specially selected samples were used for the present analysis 
        (see Table~\ref{tab:sampleLumiNevt}).
      These samples were recorded using subsets of the triggers found in 
        Section~\ref{sec:TrigDev}.
      The $J/\psi$ events discussed in this thesis were obtained analyzing the 
      sample labeled in Table~\ref{tab:sampleLumiNevt} as physics.
      A minimum bias sample was recorded for the sake of estimating efficiencies.
      Last, a zero bias sample was recored for investigating the ZDC and the 
        noise distributions of HF.
      By recording this hierarchy of samples, interesting events are selected 
        with a much higher purity in the physics sample, while the zero bias and 
        minimum bias samples allow for the investigation of the selection 
        criteria. 
  
      To record the physics sample containing the $J/\psi$ signal, a muon trigger
        was paired with a veto on energy in the BSC and a requirement that there 
        be energy in at least one of two sides the ZDC. 
      This trigger utilizes the unlikely chance of having overlapping noise in
        in the ZDC and muon detector.
      Because of the characteristically low momentum of UPC $J/\psi$ as compared
        to $J/\psi$ created by other physics process, the loosest muon 
        trigger was used.
      The trigger rejects muon noise by requiring that a interaction took place
        that deposits energy in the ZDC.
      Contributions from hadronic interactions are reduced by the veto on the 
        BSC.
      In this way the balance between reducing the rate and maximizing the 
        efficiency was struck, allowing for the data to be recorded without 
        producing high rates resulting in dead time for the detector.  
      
      In order to investigate the muon trigger and the other parts of the events 
        selection, a minimum bias sample was recorded using the ZDC. 
      For ZDC triggered sample, any event which had energy consistent with at 
        least one neutron in either of the two sides of the ZDC was recorded.
      This process is much more common than the UPC $J/\psi$ production.
      For this reason, the rates of this trigger are much higher than the physics
        trigger, and only a small sub set of these events are recorded.
      From this trigger the pixel track efficiency was estimated. 
  
      In addition to the minimum bias and physics sample, a zero bias sample was 
        recorded to examine the ZDC trigger and the HF noise distributions. 
      The zero bias trigger fired every time both beams passed through CMS. 
      Only 4 events out of every million triggered were recorded for this sample. 
      This sample allowed for an unbiased measurement of the ZDC trigger 
      efficiency as discussed in Section~\ref{sec:effDet}. 
      Because the zero bias trigger does not require any activity in any of the
        CMS sub detectors, the sample contains very few hadronic collisions. 
      This allowed for a measurement of the electronic noise distributions in
        the HF, which will be discussed in the next section.
  
      The integrated luminosity for each of the three samples is calculated
        by recording activity in HF. 
      The cross section for HF activity is measured from a van der Meer scan, and
        the cross section was found to be \textcolor{red}{X}.
      In this way the amount of integrated luminosity for any period running is
        related to the activity in HF. 
      \begin{table}
  	    \centering
  	    \begin{tabular}{|l|l|l|}
  	      \hline Sample & Events & $L_{int}$ \\ \hline \hline
  	      Physics & \textcolor{red}{300K} & \textcolor{red}{143.3 
  	        $\mu$$b$} \\ \hline
  	      Minimum Bias & \textcolor{red}{100K} & \textcolor{red}{X} \\ \hline
  	      Zero Bias & \textcolor{red}{5M} & \textcolor{red}{580 b} \\ \hline \hline
  	    \end{tabular}
  	    \caption{Integrated luminosities and number of events for the three
  	      samples used in this analysis.}
  	    \label{tab:sampleLumiNevt}
      \end{table}
      An additional method was used to cross check the integrated luminosity 
        obtained by the van der Meer scan technique.
      The integrated luminosity can also be measured by counting the events that
        fire the L1 minimum bias trigger together with the inelastic PbPb cross 
        section. 
  
    \subsection{Event selection}
      The analysis described in this thesis focuses on UPC $J/\psi$s decaying to 
        muons. 
      The trigger used for this analysis recored 346841 events.
      A set of off-line cuts were applied to increase the relative contribution 
        of UPC events relative to background processes. 
      The following cuts were applied. 
 
      \begin{table}
        \centering
        \begin{tabular}{|c|c|c|} \hline 
          cut & cut type & events \\ \hline
          all triggered & -- & 346841 \\ \hline
          good vertex requirement & beam background rejection & 340997 \\ \hline
          beam halo muon rejection & beam background rejection & 302777 \\ \hline
          cluster shape compatibility requirement & beam background rejection & 233590 \\ \hline
          single-sided neutron requirement & hadronic interaction rejection & 149992 \\ \hline
          two track requirement & hadronic interaction rejection & 32732 \\ \hline
          HF signal rejection & hadronic interaction rejection & 5392 \\ \hline
          muon quality requirement & fake muon rejection & 1956\\ \hline
          J/$\psi$ mass requirement & kinematic cut & 662 \\ \hline
          muon detectability cuts & kinematic cut & 541 \\ \hline
        \end{tabular}
        \caption{Effects of event selection cuts.}
        \label{tab:evSelCutNumbers}
      \end{table}

      Two sets of event selection cuts were applied to reject background events. 
      The first set rejects background from the beam.
      The second rejects events where hadronic collisions have occurred.
      
      To reject beam induced background the following cuts were applied:
      \begin{itemize}
        \item The reconstructed vertex must be within \textcolor{red}{X} cm in 
          the transverse direction and \textcolor{red}{X} cm in the 
          longitudinal direction. This cut insures that reconstructed particles 
          come from interactions between the two beams rather than event where 
          one of the two beams interact with gas particles near the interaction 
          point. 
  	    \item Beam halo muons were rejected using the timing of the muon hits.
              The beam halo cut rejects events where muons surrounding the beam 
              stream through the detector. 
  	    \item Pixel cluster shape should be compatible with the vertex. 
          This cut requires that energy deposits in the silicon tracker point 
            back to the reconstructed  primary vertex. 
      \end{itemize}
      These beam background cuts do not reject any UPC $J/\psi$ candidates. 
  
      The second set of background rejection cuts were designed to 
        reduce contamination from hadronic interactions. 
      \begin{itemize}
  	    \item No more than 2 reconstructed tracks in the event.
          The track requirement rejects events that produce many charged 
          particles.
  	    \item Maximum reconstructed hit energy in HF was required to be below 
            the threshold for electronic noise. 
          Nearly all hadronic interactions ($\sim$ 98\%) produce particles in the 
            range $3<|\eta|<5$ covered by the HF detector.
          By requiring that the energy deposits in HF resemble noise, nearly all
            elastic hadronic collisions are expected to be rejected.
  	    \item Energy in the ZDCs consistent with neutrons on only one side 
            of the interaction point
          In hadronic interactions both nuclei break-up. 
          By requiring that ZDC only reconstruct neutrons on one side of the 
            interaction point, hadronic interactions that produce neutrons on both 
            sides were rejected.
      \end{itemize}
      Each of these cuts are designed to reject topologies produced by hadronic
        interactions.
      The effect of these cuts can be seen in Table\textcolor{red}{X}.
  
      The following standard muon quality cuts are applied:
      \begin{itemize}
        \item Tracker track matched with at least one muon segment 
          (in any station) in both X and Y coordinates (< 3 $\sigma$).
        \item Cut on number of tracker layers with hits $>$ 5.
        \item Number of pixel layers $>$ 0.
        \item The $\chi^{2}$ per degrees of freedom of the track fit $<$ 3. 
        \item Loose transverse and longitudinal impact parameter cuts, with in 3 
          cm in the transverse direction and withing 30 cm in the longitudinal 
          direction with respect to the primary vertex.
      \end{itemize}
      These cuts are applied to reduce the number of fake muons.
  
  \section{\label{sec:breakUpDet} Break up determination}
    \subsection{ZDC Signal Reconstruction}
      There are 18 zdc channel. 
      There are 4 hadronic channels and 8 electromagnetic channels on each side
        of the CMS. 
      To measure neutrons in the ZDC the charge in each channel is first 
        converted to a signal. 

	\begin{figure}[h]
		\centering
		\includegraphics[width=\textwidth]{zdcPulseShape}
		\caption{ZDC pluse shape.}
		\label{fig:zdcPulseShape}
	\end{figure}

      Channel Signal definition:
      The signal in the zdc calculated two ways. 
      \begin{enumerate}
	\item Method 1
	\begin{enumerate}
	  \item signal: 4,5,6 
          \item background: 1,2
        \end{enumerate}
	\item Method 2
	\begin{enumerate}
	  \item signal: 4
	  \item background: 5
        \end{enumerate}
      \end{enumerate}

      As seen in Fig~\ref{fig:zdcPulseShape}, method 1 uses all the dominant 
        signal time slices and uses the maximum number of “clean” non signal 
        time slices to estimate the noise pedestal event by event. 
      This minimizes the effect of random noise time slice by time slice by 
        averaging over the maximum number of time slices per event. 

      Method 2 uses only the two time slice that are most likely to be above 
        zero. 
      Because all signal that is less than zero is not measured, method 1 
        only allows for a noise pedestal measurement half the time.

      The noise spectrum was measured from the pre collisions time slices. 
      Each time slice in Fig.~\ref{fig:zdcPulseShape} is 25 ns in length.
      In Fig.~\ref{fig:zdcPulseShape} higher than average signal can be seen
        in the 0th time slice. 
      This is due to events where activity was present in the ZDC for 
        two consecutive bunch crossings, and corresponds to a separation of 200
        ns.
      Time slices 1 and 2 are therefore occurred between colliding bunch 
        crossings.
      These time slices were used to estimate the noise spectrum.

      \begin{figure}[!Hhbt]
        \centering
        $ \begin{array}{cc}
          \includegraphics[width=.45\textwidth]{zdcNegEMNoiseFromZBNoCor} & 
          \includegraphics[width=.45\textwidth]{zdcPosEMNoiseFromZBNoCor} \\
          \includegraphics[width=.45\textwidth]{zdcNegHDNoiseFromZB} &
          \includegraphics[width=.45\textwidth]{zdcPosHDNoiseFromZB}
        \end{array} $
        \caption{ZDC noise spectra from ZDC$^{-}$ EM section (upper left), 
          ZDC$^{+}$ EM section (upper right), ZDC$^{-}$ HAD section (lower left), 
          and ZDC$^{+}$ HAD section (lower right).}
        \label{fig:zdcNoiseSpectra}
      \end{figure}

      Fig~\ref{fig:zdcNoiseSpectra} shows the noise spectrum for each of the 
        EM and HAD sections for the two sides of the ZDC. 
      The spectra determined which sections contain only noise. 
      
      From these two methods the ZDC+ and ZDC- energy spectra near the 
        one neutron peak are plotted in Fig.~\ref{fig:zdcSpec2v1}.
      While method 1 in blue and method 2 in red not differ much in ZDC-, 
        the clear separation of the one neutron peak signal from the noise 
        peak about zero is evident. 
      \begin{figure}[h]
        \centering
        \includegraphics[width=\textwidth]{zdcSpec2v1}
        \caption{Comparison of ZDC signal reconstruction methods.}
        \label{fig:zdcSpec2v1}
      \end{figure}

    \subsection{Determination of the one neutron thresholds}
      The ZDC thresholds used to establish the break-up mode were measured from
        zero bias data.
      By using this dataset, the spectrum does not contain a trigger bias. 
      The trigger requriment in the event is that both beams were present in 
        CMS.
      This does however include a significant electronic noise contribution due
        to events where no neutrons are emitted in the direction of the zdc.
      To sperate the signal from the electronic noise additional cuts are 
        applied in the zero bias data.

      In Fig.~\ref{fig:zdcPulseShape} the pulse shape peaks in the peaks in the
        fourth time slice, for electronic noise however, any of the ten time 
        slices are equally likely to have a peak value.
      Using this fact, signal can be preferably selected by requiring that the
        hadronic channels of the ZDC have a peak signal in the fourth time 
        slice.
      \begin{figure}[h]
        \centering
        \includegraphics[width=0.6\textwidth]{zdcMinusSingleNuNoInc}
        \caption{Effects of requiring in time signal in ZDC hadronic 
          channels.}
        \label{fig:zdcTimingCuts}
      \end{figure}
      In Fig.~\ref{fig:zdcTimingCuts} no noise subtraction is used. 
      As each additional hadronic channel is required to have a maximum signal
        in the fourth time slice, the single neutron peak emerges. 
      Using the noise subtraction method described by Method 1, the same signal
        emerges.
      \begin{figure}[h]
        \centering
        \includegraphics[width=0.6\textwidth]{zdcMinusSingleNuNoSub}
        \caption{Effect of ZDC signal timing requirements after noise 
          subtraction.}
        \label{fig:zdcTimingAfterNoiseSub}
      \end{figure}
      Fig.~\ref{fig:zdcTimingAfterNoiseSub} confirms that both noise 
        subtraction and the timing require produce the same signal.
      This gives confidence that the signal is not an artifact of either cut, 
        but the true neutron signal. 

      Fig.~\ref{fig:zdcTimingAfterNoiseSub} and Fig.~\ref{fig:zdcSpec2v1} 
        demonstrate the consistence of the using timing cuts and noise 
        subtraction to enhance the signal neutron peak. 
      Fig.~\ref{fig:zdcTimingAfterNoiseSub} confirms the legitimacy of the 
        noise subtraction method in ZDC$^{-}$ by showing the that the same signal
        emerges from the noise subtraction method as the timing method.
      Fig.~\ref{fig:zdcSpec2v1} demonstrates the corresponds between between
        noise subtraction method 1 and method 2 on in ZDC$^{-}$ where signal is 
        better separated from the electronic noise. 
      This allows for confidence that the signal seen in ZDC$^{+}$ using method 2 
        is the one neutron peak.

      The spectrum for method 1 and 2 are fit to a series of Gaussians.
      \begin{figure}[!Hh]
        \centering
        $ \begin{array}{cc}
          \includegraphics[width=0.45\textwidth]{zdcMinusZBFitTimeCut} &
          \includegraphics[width=0.45\textwidth]{zdcPlusZBFitTimeCut}
        \end{array} $
        \caption{Fit to charge spectrum from ZDC$^{-}$ (left) and ZDC$^{+}$ (right) using 
          method 1}
        \label{fig:zdcM1Fit}
      \end{figure}
      The electronic noise is fit to a Gaussian about zero.
      The one, two, and three neutron peaks are fit Gaussians that are 
        successively broader.
      The mean of each peak was initially set to multiples of the mean of the 
        one neutron peak. 
      \begin{figure}[!Hh]
        \centering
        $ \begin{array}{cc}
          \includegraphics[width=0.45\textwidth]{zdcFit45Neg} &
          \includegraphics[width=0.45\textwidth]{zdcFit45Pos}
        \end{array} $
        \caption{Fit to charge spectrum from ZDC$^{-}$ (left) and ZDC$^{+}$ (right) using 
          method 2}
        \label{fig:zdcM2Fit}
      \end{figure}
      The threshold for a neutron in the ZDC was taken from the fits in 
        Fig.~\ref{fig:zdcM1Fit} and Fig.~\ref{fig:zdcM2Fit}.
      Any signal greater 2 $\sigma$ below the mean of the one neutron peak was 
        considered signal. 

  \section{\label{sec:sigEx} Signal extraction}
    The invariant mass distribution for opposite sign dimuons is shown in 
      Figure~\ref{fig:massFit}. 
    A J/$\psi$ signal is clearly visible together with tails at higher and lower mass
      due to the photon-photon process
    A fit to the invariant mass distribution was done using a Gaussian and a Crystal Ball
      function to account for the J/psi signal and a first and second order polynomial 
      function for the photon-photon process.
    The extracted number of J/$\psi$ candidates from this fit includes all J/$\psi$s in
      the mass window that pass the analysis cuts.

    \begin{figure}[!Hhtb]
      \centering
      \includegraphics[width=.6\textwidth]{massFitSimple}
      \caption{Mass fit to J/$\psi$ using Gaussian for the 
        signal and a first order polynomial for the background}
      \label{fig:massFit}
    \end{figure}
   
    The p$_{T}$ spectrum is fit using templates from the three physics MC 
      samples. 
    These contributions display a different shape in transverse momenta. 
    For this reason, the number of coherent J/$\psi$ candidates were extracted 
      from the dimuon p$_{T}$ distribution, around the J/$\psi$ mass, 
      after fitting together MC templates for both signal and background.
    The templates for the fit come form the three STARlight MC samples. 

    \begin{figure}[!Hhbt]
      \centering
      \includegraphics[width=.6\textwidth]{ptOnly}
      \caption{ Fit to MC p$_{T}$ templates. }
      \label{fig:ptTemps}
    \end{figure}

    The shape of the photon-photon process and coherent J/$\psi$ are very 
      similar in p$_{T}$.
    These two shapes normalizations are highly anti-correlated in the fit 
      (see Fig.~\ref{fig:ptOnlyCor}).

    \begin{figure}[!Hhbt]
      \centering
      \includegraphics[width=.6\textwidth]{nCoNGammaCorPtOnly}
      \caption{68\%, 95\%, and 99\% confidence contours from the p$_{T}$ 
        template fit. }
      \label{fig:ptOnlyCor}
    \end{figure}

    To utilize the mass fits ability to distinguish the photon-photon process 
      from the coherent and incoherent process all while utilizing the p$_{T}$
      fits ability to separate the coherent and photon-photon processes from 
      the incoherent, a simultaneous fit to the mass spectrum and p$_{T}$ 
      spectrum was preformed.
    In the simultaneous fit, the parameter for the contribution from the 
      photon-photon process is shared by the two fits.
    
    \begin{figure}[!Hhbt]
      \centering
      \includegraphics[width=0.9\textwidth]{ptMassSimGaussLine}
      \caption{Simultaneous fit to the mass and p$_{T}$ spectra.}
      \label{fig:simFitMassPtGauss}
    \end{figure}

    By simultaneously fitting both spectra, the anti-correlation between the 
      photon-photon and coherent normalization parameters was reduced as well
      as the overall error on the parameters.
    The slope of the confidence contours in Fig.~\ref{fig:ptOnlyCor} and 
      Fig.~\ref{fig:simGaussCor} demonstrate the reduction in the correlation
      between the normalization parameters. 
    The slope in of the confidence contours in Fig.~\ref{fig:ptOnlyCor} 
      is noticeably closer to 0 than the apparent negative slope in 
      Fig.~\ref{fig:simGaussCor}.
    \begin{figure}[!Hhbt]
      \centering
      \includegraphics[width=0.6\textwidth]{nCoNGammaCorPtMass}
      \caption{68\%, 95\%, and 99\% confidence contours from the 
        simultaneous fit. }
      \label{fig:simGaussCor}
    \end{figure}

  \section{\label{sec:effDet} Efficiency determination}
    \subsection{Muon Efficiencies}
      The muon efficiencies are measured from MC and data.
      The MC based measurement accounts for the detector acceptance and the 
        efficiency of the muon quality discussed in 
        Section~\ref{sec:DataSetEvSel}.
      The trigger efficiencies were measured in data using the tag and probe 
        method, which is discussed below. 

       CMS has a limited acceptance for $J/\psi$s, particularly in the case of 
        $J/\psi$s with low momentum like those produced in UPC events. 
      To measure the acceptance of CMS for $J/\psi$s, reconstructed dimuon 
        candidates were considered detectable if both reconstructed daughters 
        fell into a detectability region.
      This region was defined using the coherent J/$\psi$ events obtained from 
        STARlight.
      The efficiency for reconstructing single muons $\varepsilon^{\mu}_{reco}$ 
        is defined by $\varepsilon^{\mu}_{reco} = \frac{N^{\mu}_{reco}}{N^{\mu}_{gen}}$, 
        where $N^{\mu}_{reco}$ is the number reconstructed muons after 
        after the full CMS detector simulation and that pass the standard
        muon quality cuts, and $N^{\mu}_{gen}$ is the number of generated 
        muons from STARlight.
      \begin{figure*}[!Hhtb]
        \centering
%          $ \begin{array}{cc}
          \includegraphics[width=.6\textwidth]{mcEffMaps/accMuJpCo} %&
    %      \includegraphics[width=.45\textwidth]{mcEffMaps/accMuJpInCo} \\
    %      \includegraphics[width=.45\textwidth]{mcEffMaps/accMuGamma} &
    %      \includegraphics[width=.45\textwidth]{mcEffMaps/accMuGun}
%         \end{array} $
        \caption{ Muon daughter detectability from coherent J/$\psi$, 
          incoherent J/$\psi$, photon-photon, and J/$\psi$ gun samples.}
        \label{fig:muonDaughterDet}
      \end{figure*}
      Fig.~\ref{fig:muonDaughterDet} shows the efficiency for reconstructing
        single muons from coherent J/$\psi$ events.
      To avoid the edges of the detectors acceptance, all reconstructed muons 
        that fall into a (p$_{T}$,$|\eta|$) bin that has an efficiency less 
        than 20\% were rejected thus defining the detectability region.
      The acceptance for reconstructing dimuons was calculated from MC
        using the following formula:
      \begin{equation}
        A=\frac{N_{det}(|y|,p_{T})}{N_{gen}(|y|,p_{T})},
        \label{eq:jpsiAccEq}
      \end{equation}
        where $N_{det}$ is the number of reconstructed dimuons where both 
        daughters fall in to the detectability region, and $N_{gen}$ is the
        number of generated dimuons. 
      From Eq.~\ref{eq:jpsiAccEq}, the acceptance for $J/\psi$ was calculated
        as a function of $|y|$, and p$_{T}$ (see Fig.~\ref{fig:jpsiAcceptance}).
        \begin{figure*}[!Hhtb]
          \centering
          $ \begin{array}{cc}
            \includegraphics[width=.45\textwidth]{mcEffMaps/detAccJpCoStep} &
            \includegraphics[width=.45\textwidth]{mcEffMaps/detAccJpInCoStep} \\
            \includegraphics[width=.45\textwidth]{mcEffMaps/detAccGammaStep}
          \end{array} $
          \caption{Dimuon acceptance from coherent J/$\psi$ (top left), incoherent 
            J$\psi$ (top right), and photon-photon interactions (lower).}
          \label{fig:jpsiAcceptance}
        \end{figure*}

      The tag and probe method is used to measure the trigger efficiency of 
        the muon daughters, which is a data driven approach. 
      In this method there are three categories of daughter muons. 
      Tag muons are high quality muons.
      Passing probes are reconstructed muons that match the muon trigger, 
        while failing probes do not. 
      Each dimuon will have one daughter classified as a tag and the other
        as a probe.
      From here three invariant mass histograms are studied. 
      One histogram is created from all pairs. 
      The second comes from pairs where the probe is a passing probe.  
      The last histogram comes from pairs where the probe fails to fulfill
        the trigger, \textit{i.e.} the probe is a failing probe. 
      Because this depends on the p$_{T}$ and $|\eta|$ of the probe, one set 
        of three histograms for each (p$_{T}$,$|\eta|$) bin of the probe is 
        created.

      To extract the single muon trigger efficiency $\varepsilon^{\mu}_{trig}$, 
        each set of invariant mass histograms was simultaneously fitted. 
      The signal was fitted using a Crystal Ball function, and the background 
        was fitted to an exponential.
      The Crystal Ball parameters were simultaneously fitted to all three 
        histograms.
      The exponential function was fitted to the failing and passing probe 
        histograms separately.
      Because the background shapes are in principal different for the two 
        samples, the efficiency is driven by this difference. 

      To measure the trigger efficiency a tag is required to pass all muon
        quality cuts and matched to the trigger.
      The probe is required to pass all quality cuts. 
      A passing probe is a probe that is also matched to the trigger. 
      In this way the tag leaves the probe in biased by the trigger and the 
        efficiency can be measued by fitting mass.  

      Fig.~\ref{fig:tnpFitPlot} shows the fit of the three sets of pairs. 
      \begin{figure}[!Hh]
        \centering
        \includegraphics[width=.6\textwidth]{tNp/tnpFits}
        \caption{Fits to tag and probe pairs in the J/$\psi$ mass region.}
        \label{fig:tnpFitPlot}
      \end{figure}
      This fit is done for each bin of the probes p$_{T}$ and $\eta$.
      The resulting fit is in Fig.~\ref{fig:tnpTrigMap}.
      \begin{figure}[!Hhbt]
        \centering
        \includegraphics[width=.6\textwidth]{tNp/tnpFromFit}
        \caption{Muon trigger efficiencies in p$_{T}$ and $\eta$ bins from 
          the tag and probe method.}
        \label{fig:tnpTrigMap}
      \end{figure}

      The dimuon trigger efficiency $\varepsilon_{dimuon trigger}$ was measured
        from the single muon efficiencies. 
      The efficiency of each candidate was calculated using the following
        equation:
      \begin{equation}
        \label{eq:dimuTrigEff}
        \varepsilon_{dimuon trigger}=1-(1-\varepsilon_{trigger}^{\mu_{1}})(1-\varepsilon_{trigger}^{\mu_{2}}),
      \end{equation}
      where $\varepsilon_{trigger}^{\mu_{1}}$ is the tag and probe efficiency
        of the first dimuon daughter, and $\varepsilon_{trigger}^{\mu_{2}}$ is
        the efficiency of the second muon daughter. 
      In Eq.~\ref{eq:dimuTrigEff} the probability of at least one daughter
        firing the trigger is calculated by subtracting one from the
        probability that neither daughter fires the trigger,
        thus giving the dimuon trigger efficiency. 

      The average dimuon trigger efficiency for each dimuon (p$_{T}$,$|y|$) bin
        was calculated by averageing the individual dimuon candidates in each
        bin. 
      \begin{figure}[!Hhbt]
        \centering
        \includegraphics[width=0.6\textwidth]{averageTriggerEff}
        \caption{The trigger efficiency from tag and probe averaged over candidates
          in each (p$_{T}$,$|y|$) bin.}
        \label{fig:avTrigEffCo}
      \end{figure}
      The average trigger efficiency was multiplied by the acceptance from the MC 
        to produce a total efficiency times acceptance factor. 
      \begin{figure}[!Hhtb]
        \centering
        \includegraphics[width=0.6\textwidth]{averageExA}
        \caption{The acceptance times averaged trigger efficiency from tag and 
          probe.}
        \label{fig:avAccEff}
      \end{figure}

    \subsection{ZDC trigger efficiency}
      A special trigger was prepared to monitor the ZDC trigger efficiency. 
      This trigger required either a ZDC$^{+}$ or ZDC$^{-}$ trigger, together with at 
        least one pixel track. 
      Events were accepted offline if there was no activity in the BSCs or 
        activity on a single side. 
      This sample suffers from a trigger bias. 
      For example, a sample triggered by ZDC$^{+}$ would always produce a ZDC$^{+}$ 
        trigger efficiency of one. 
      To avoid this, the special trigger sample was divided into two 
        subsamples in the following way. 
      A first sample triggered by the ZDC$^{+}$ input and second one triggered by 
        the ZDC$^{-}$. 
      The ZDC$^{+}$ trigger efficiency is measured from the ZDC$^{-}$ sample, and vice 
        versa.

      The trigger efficiency for reconstructed ZDC energies above the
        single neutron threshold were estimated (see for Sec.~\ref{sec:breakUpDet}).
      The ZDC$^{+}$ efficiency was calculated using the ZDC$^{-}$ triggered 
        sample.
      To estimate the efficiency, the number of events with energy in 
        ZDC$^{+}$ greater than the single neutron threshold, N$_{events}$, 
        were measured.
      From this set of events, the number of events that also fire the 
        ZDC$^{+}$ are measured.
      The ratio between the number of single neutron events that fire the 
        trigger and all single neutron events was taken as the estimate of 
        trigger efficiency. 
      The same procedure is applied for each side of the ZDC. 

      \begin{table}
        \centering
        \begin{tabular}{|c|c|c|c|c|}
           ZDC Side & Reco Method & N$_{events}$ & N$_{trig}$ & $\varepsilon_{ZDC}$ \\ \hline
           ZDC$^{+}$ & 1 & 72946  & 71688 & 0.982 $\pm$ 0.005 \\ \hline
           ZDC$^{+}$ & 2 & 73028  & 71706  & 0.9819  $\pm$ 0.005  \\ \hline
           ZDC$^{-}$ & 1 & 76137  & 71786  & 0.9429  $\pm$ 0.005  \\ \hline
           ZDC$^{-}$ & 2 & 76132  & 71859  & 0.9439  $\pm$ 0.005  \\ \hline
        \end{tabular}
        \caption{ZDC trigger efficiencies for ZDC reconstruction method 1 and 
          2}
        \label{tab:zdcEfficiency}
      \end{table}
  \section{\label{sec:sysCheck} Systematic checks}
    \subsection{MC vs Data compairson}
      \begin{figure}[h]
        \centering
        \includegraphics[width=0.5\textwidth]{jpsiMcComp/jpsiAbsRapCoherent}
        \caption{Comparison of the of the dimuon rapidity distributions between 
          coherent MC sample and Data.}
        \label{fig:jpsiAbsRapCoherent}
      \end{figure}
      \begin{figure}[h]
        \centering
        \includegraphics[width=0.5\textwidth]{jpsiMcComp/jpsiPhiCoherent}
        \caption{Comparison of the of the dimuon $\varphi$ distributions 
          between coherent MC sample and Data.}
        \label{fig:jpsiPhiCoherent}
      \end{figure}
      \begin{figure}[h]
        \centering
        \includegraphics[width=0.5\textwidth]{jpsiMcComp/jpsiPtCoherent}
        \caption{Comparison of the of the dimuon p$_{T}$ distributions 
          between coherent MC sample and Data.}
        \label{fig:jpsiPtCoherent}
      \end{figure}

    \subsection{Tag and probe}
      To estimate the effect the tag and probe fits, the tag and probe 
        efficiencies were additionally measured by counting probes in the 
        J/$\psi$ mass window. 
      \begin{figure}[!Hhbt]
        \centering
        $ \begin{array}{cc}
          \includegraphics[width=.45\textwidth]{tNp/tnpCounting} &
          \includegraphics[width=.45\textwidth]{tNp/tnpFromFit}
        \end{array} $ 
        \caption{Tag and probe trigger efficiencies from counting (left) 
          compared to fitting (right)}
        \label{fig:tnpCntVFit}
      \end{figure}
    \subsection{Template fit}
      \begin{figure}[!Hhtb]
        \centering
        \includegraphics[width=.6\textwidth]{ptOnly}
        \caption{Coherent, incohernt, and photon-photon process p$_{T}$ template fit to data.}
        \label{fig:ptTempFit}
      \end{figure}

      The normalization for the coherent and incoherent templates are left
        free. 
      To address this issue the template from the photon-photon process was 
        fixed to the cross section value given by STARlight and normalized to
        the integrated luminosity of the data. 
      The normalization for the photon-photon template was allowed a range of 
        20\% due to the 20\% uncertainty in the STARlight cross section.

      The systematic uncertainty for the contribution to the three components
        of the signal was estimated by varying the mass fit.
      Because the mass fit is used to fix the contribution to the p$_{T}$ 
        spectrum of the photon-photon process, the change of the mass fit
        results in different coherent contributions from the p$_{T}$ template
        fit.

      \begin{figure}[!Hhbt]
        \centering
        $ \begin{array}{ccc}
          \includegraphics[width=.3\textwidth]{cbPolyBkgEst} &
          \includegraphics[width=.3\textwidth]{gausLinBkgEst} &
          \includegraphics[width=.3\textwidth]{gausCCBkgEst} \\
          \includegraphics[width=.3\textwidth]{cbPoly} &
          \includegraphics[width=.3\textwidth]{gausLin} &
          \includegraphics[width=.3\textwidth]{gausCC}
        \end{array} $
        \caption{Various mass distribution fits and the corresponding p$_{T}$
          template fit.}
        \label{fig:massPtFitsForSyst}
      \end{figure}

      Moving from left to right in Fig~\ref{fig:massPtFitsForSyst} the 
        contribution from the photon process increases. 
      The left most fit uses a Crystal Ball function to account for the 
        radiative decay of the final state daughters of the J/$\psi$.
      The low mass exponential portion however picks up background events 
        and overestimates the J/$\psi$ contribution. 
      The right most plot fits the background to a 2nd order Cheby-Chev 
        polyinomial, which overestimates the background.
      Because the Cheby-Chev peaks just below the J/$\psi$ peak, this fit 
        overestimates the background and in turn underestimates the signal 
        contribution.
      The Gaussian fit with a linear background however does a reasonable job
        of fitting both the background and the signal. 

      From these three fits an upper and lower bound of the systematics due
        the choice of fit functions was estimated. 

    \subsection{Mass fit}
      \begin{figure*}[!Hhtb]
        \centering
        \includegraphics[width=.45\textwidth]{massFitSimple}
        \includegraphics[width=.45\textwidth]{massFitCBPoly2}
        \caption{Mass fit to J/$\psi$ using Gaussian (Left) and Crystal Ball (Right) for the 
          signal and a polynomial for the background}
        \label{fig:massFitSys}
      \end{figure*}

      Fig.~\ref{fig:massFitSys} demonstrates the small dependence the raw J/$\psi$ 
        yield has on the fitting function. 
      Both fit functions agree well, with reduced $\chi^{2}$ values below one.
      The Crystal ball fit give an upper estimate for the J/$\psi$ yield.
      The Gaussian fit gives an lower estimate. 
      The main difference comes from the lower mass tails.
      In the Crystal ball fit the lower tail is considered to be signal due to 
        shifting of the mass spectrum to lower mass due to radiation from the 
        final state muons. 
      In the Gaussian fit the lower mass tail is considered to be background and 
        the signal is sharper.

      As check on the simultaneous p$_{T}$ and mass fit, the mass fit is done
        using mass templates from STARlight.
      \begin{figure}[!Hhbt]
        \centering
        \includegraphics[width=0.6\textwidth]{ptMassSimTemp}
        \caption{Simultaneous fit to the mass and p$_{T}$ using mass templates
          for the mass fit. }
        \label{fig:simFitTemp}
      \end{figure}

    \subsection{MC acceptance}
      To estimate the effect of the GEN level p$_{T}$ distribution on the 
        acceptance the incoherent sample was used to correct the coherent 
        yield.
      \begin{figure}[!Hhbt]
        \centering
        \includegraphics[width=0.6\textwidth]{coCorInCoAcc}
        \caption{Yields corrected by the MC incoherent acceptance map.}
        \label{fig:coYieldInCoCor}
      \end{figure}

      The effect of polarization was estimated by correcting by the acceptance
        for an polarized J/$\psi$ sample.
      \begin{figure}[!Hhtb]
        \centering
        \includegraphics[width=0.6\textwidth]{coCoGaussGun}
        \caption{Yields corrected by an unpolarized J/$\psi$ sample.}
        \label{fig:coYieldGaussCor}
      \end{figure}
    \subsection{ZDC efficiency}
      To estimate systematic errors in the ZDC trigger efficiency, the trigger
        efficiency was measured from five additional samples. 
      \begin{table}
        \centering
        \begin{tabular}{|c|c|c|c|c|}
          \hline ZDC Side & Reco Method & N$_{events}$ & N$_{trig}$ & $\varepsilon_{ZDC}$ \\ \hline
           \multicolumn{5}{|c|}{(ZDC$^{+}$ or ZDC$^{-}$) and 1 pixel track} \\ \hline 
           ZDC$^{+}$ & 1 & 72946  & 71688 & 0.982 $\pm$ 0.005 \\ \hline
           ZDC$^{+}$ & 2 & 73028  & 71706  & 0.9819  $\pm$ 0.005  \\ \hline
           ZDC$^{-}$ & 1 & 76137  & 71786  & 0.9429  $\pm$ 0.005  \\ \hline
           ZDC$^{-}$ & 2 & 76132  & 71859  & 0.9439  $\pm$ 0.005  \\ \hline
           \multicolumn{5}{|c|}{(ZDC$^{+}$ or ZDC$^{-}$)} \\ \hline 
           ZDC$^{+}$ & 1 & 30986 & 30333 & 0.9789 $\pm$ 0.0079 \\ \hline
           ZDC$^{+}$ & 2 & 31029 & 30339 & 0.9778 $\pm$ 0.0079 \\ \hline
           ZDC$^{-}$ & 1 & 39178 & 30164 & 0.7699 $\pm$ 0.0059 \\ \hline
           ZDC$^{-}$ & 2 & 35703 & 30443 & 0.8527 $\pm$ 0.0067 \\ \hline
           \multicolumn{5}{|c|}{ Zero Bias} \\ \hline 
           ZDC$^{+}$ & 1 & 109967  & 101598  & 0.9239 $\pm$ 0.0040 \\ \hline
           ZDC$^{+}$ & 2 & 110230  & 101561  & 0.9214 $\pm$ 0.0040 \\ \hline
           ZDC$^{-}$ & 1 & 253241  & 86660  & 0.3422 $\pm$ 0.0013 \\ \hline
           ZDC$^{-}$ & 2 & 156336  & 87401  & 0.5591 $\pm$ 0.0024 \\ \hline
           \multicolumn{5}{|c|}{ Zero Bias with ZDC timing cuts} \\ \hline 
           ZDC$^{+}$ & 1 & 88676  & 84429  & 0.9521 $\pm$ 0.0046 \\ \hline
           ZDC$^{+}$ & 2 & 88480  & 84202  & 0.9517 $\pm$ 0.0046 \\ \hline
           ZDC$^{-}$ & 1 & 59878  & 54728  & 0.9140  $\pm$ 0.0054 \\ \hline
           ZDC$^{-}$ & 2 & 60467  & 54733  & 0.9052  $\pm$ 0.0053 \\ \hline
           \multicolumn{5}{|c|}{(ZDC$^{+}$ or ZDC$^{-}$), 1 pixel track, and L1 Muon trigger} \\ \hline 
           ZDC$^{+}$ & 1 & 65466  & 63376  & 0.9681 $\pm$ 0.0054  \\ \hline
           ZDC$^{+}$ & 2 & 65543  & 63358  & 0.9667 $\pm$ 0.0054 \\ \hline
           ZDC$^{-}$ & 1 & 71929  & 63512  & 0.8830  $\pm$ 0.0048 \\ \hline
           ZDC$^{-}$ & 2 & 72932  & 63582  & 0.8718  $\pm$ 0.0047 \\ \hline
           \multicolumn{5}{|c|}{(ZDC$^{+}$ or ZDC$^{-}$), 1 pixel track, and L1 EG trigger } \\ \hline 
           ZDC$^{+}$ & 1 & 613758  & 602123  & 0.9810 $\pm$ 0.0018 \\ \hline
           ZDC$^{+}$ & 2 & 614014  & 601863  & 0.9802 $\pm$ 0.0018 \\ \hline
           ZDC$^{-}$ & 1 & 643905  & 602671  & 0.9360  $\pm$ 0.0017 \\ \hline
           ZDC$^{-}$ & 2 & 647888  & 603089  & 0.9309  $\pm$ 0.0017 \\ \hline
        \end{tabular}
        \caption{ZDC trigger efficiencies for ZDC reconstruction method 1 and 
          2 for different trigger samples}
        \label{tab:zdcEfficiencySys}
      \end{table}
      

