\chapter{Introduction}
  \DIFaddbegin \DIFadd{High energy physics probes the smallest scales in order to discover the 
    fundamental constituents of the universe and how they interact.
  From these searches of the smallest scales, the standard model of particle
    physics has emerged. 
  The standard model contains 3 forces, the weak, the electromagnetic, and 
    strong force, and two types of mater that interact through these forces,
    quarks and leptons.
  The quarks can interact through all three forces.
  The leptons however only interact through the weak and electromagnetic force.
  The mater particles interact with each through the three forces by exchanging the
    forces carrying vector bosons. 
  Strong interactions take place by exchange of gluons, the weak by Z$^{0}$ and
    W$^{\pm}$ bosons, and the electromagnetic by photon. 
  }

  \DIFadd{Each of these three forces emerges due to symmetries in the standard model.
  For example, the wave function of the Sch}\"{o}\DIFadd{rdinger equation is comprised
    of complex numbers.
  The standard model does not depend on the phase of these complex numbers. 
  The phases for each of these numbers can be arbitrarily changed at all points
    in space and time without any changing the predictions of the theory.
  This is because the gradient of an arbitrary scaler field can be added to the
    $\vec{A}$, the vector potential, which gives rise to the electromagnetic 
    without changing the magnetic field and electric field that ultimately
    interact with the particles. 
  The invariance of the standard model to the complex phase of the wave 
    function necessitates the existence of the electromagnetic force. 
}

  \DIFadd{The standard model is made of the three such symmetries, each of which is
    described by a gauge group. 
  The U(1) is the group that accounts for the electromagnetic interaction
    and gives rise to the photon. 
  The SU(2) group produces the weak bosons, Z$^{0}$ and W$^{\pm}$.
  The strong force mediated by the gluons are consequence of the SU(3) symmetry
    of the standard model.
  Of these three groups two, SU(2) and SU(3), are non-abelian. 
  The consequence of this, is that the W$^{+}$ and W$^{-}$ can interact with 
    Z$^{0}$ and photon, the gluons can interact with each other. 
  Because the gluons can interact with each other in many more ways that the 
    limited interactions between the W,Z and the photons, the non-abelian 
    nature of the strong force is particularly pronounced. 
}

  \DIFadd{The self-interaction of the gluons produces unique qualities of the strong 
    force: confinement, and asymptotic freedom. 
  }

  \DIFadd{Microseconds after the big bang, the universe existed in a state known as
    the Quark Gluon Plasma (QGP).
  In the QGP, quarks and gluons are not in hadronic bondage, forced to 
    the confines of bound states such as protons and neutrons.
  The Large Hadron Collider (LHC) produces QGP in the lab in PbPb (lead-lead)
    collisions.
  The high energies and rates of the collisions at the LHC make it possible 
    to do detailed studies of the QGP. 
  The LHC is producing rare experimental probes such as suppressed jets and 
    heavy quarkonia at an unprecedented rate in heavy-ion collisions. 
  Physicists now have better constraints on the properties like temperature,
    viscosity, and energy density of the QGP. 
  }\DIFaddend \section{Theoretical Context}
  \section{History }
