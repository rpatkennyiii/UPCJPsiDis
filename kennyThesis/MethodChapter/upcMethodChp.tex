\chapter{Method}	
  \section{MC Simulation}
  \section{\label{sec:TrigDev} Trigger Development} 
  \section{Data Sets and Event Selection}
  \subsection{Data Set}
    In order to investigate novel physics processes like UPC $J/\Psi$ 
     production, the LHC has delivered unprecedented amounts of data.
    The data for this analysis was recorded during the 2011 LHC PbPb run. 
    During this period, 157 $\mu$$b^{-1}$ where recorded by the CMS detector,
      corresponding to over a billion PbPb collisions. 
    Of this, 143 $\mu$$b^{-1}$ are used in this analysis.

    Three specially selected samples are used.
    A list of triggers used in this analysis can be found in 
      Section~\ref{sec:TrigDev}.
    The $J/\Psi$ events come from a sample triggered by UPC triggers 
      specifically designed for this measurement.
    A minimum bias sample was recorded for the sake of estimating efficiencies.
    Last, a zero bias sample was recored for investigating the ZDC and the 
      noise distributions of HF.
    By recording this hierarchy of samples, interesting events are selected 
      with a much higher purity the physics sample, while the zero bias and 
      minimum bias samples allow for the investigation of the selection 
      criteria. 

    To record the physics sample containing the $J/\Psi$ signal, a muon trigger
      was paired with a veto on energy in the BSC and a requirement that there 
      be energy in at least one of two sides the ZDC. 
    This trigger utilizes the unlikely chance of having overlapping noise in
      in the ZDC and muon detector.
    Because of the characteristically low momentum of UPC $J/\Psi$ as compared
      to $J/\Psi$ created by other physics process, the lowest threshold muon 
      trigger was used.
    The trigger rejects muon noise by requiring that a interaction took place
      that deposits energy in the ZDC.
    Contributions from hadronic interactions are reduced by the veto on the 
      BSC.
    In this way the balance between reducing the rate and maximizing the 
      efficiency was struck, allowing for the data to be recorded without 
      producing high rates resulting in dead time for the detector.  
    
    In order to investigate the muon trigger and the other parts of the events 
      selection, a minimum bias sample was recorded using the ZDC. 
    For ZDC triggered sample, any event which had energy consistent with at 
      least one neutron in either of the two sides of the ZDC was recorded.
    This process is much more common than the UPC $J/\Psi$ production.
    For this reason, the rates of this trigger are much higher than the physics
      trigger, and only a small sub set of these events are recorded.
    From this trigger the pixel track efficiency was estimated. 

    In addition to the minimum bias and physics sample, a zero bias sample was 
      recorded to examine the ZDC trigger and the HF noise distributions. 
    This sample records every event when both beams passed through CMS. 
    Only 4 events out every million triggered were recorded for this sample. 
    This sample allowed for an unbiased measurement of the ZDC trigger 
      efficiency. 
    Because the zero bias trigger does not require any activity any of the
      CMS sub detectors, the sample contains very few hadronic collisions. 
    This allowed for a measurement of the electronic noise distributions in
      the HF. 

  \subsection{Event Selection}
    Two sets of event selection cuts are applied to reject background events. 
    The first set rejects background from the beam.
    The second reject events where hadronic collisions have occurred.
    
    Three cuts are used to eliminate beam backgrounds.
    For this purpose, the following cuts are applied:
    \begin{itemize}
	    \item the reconstructed vertex must be within X cm in the in 
		    transverse dimension and X cm in the longitudinal
	    \item muon beam halo rejection
	    \item pixel cluster shape compatibility
    \end{itemize}
    The vertex cut assures that reconstructed particles are come from 
      interactions between the two beams rather than then event where one of
      the two beams interact with gas particles near the interaction point. 
    The pixel cluster shape compatibility cut requires that the clustering
      of energy deposits in the silicon tracker point back to the reconstructed
      primary vertex. 
    This cut rejects events where remnants of the beam scrap the innermost 
      most layer of the tracker. 
    By using timing on either side of the interaction point, the beam halo cut
      rejects events where muons surrounding stream through the detector. 

    Three cuts are applied to reject hadronic interactions. 
    \begin{itemize}
	    \item no more than 2 reconstructed tracks in the event
	    \item maximum RecHit energy in HF consistent with electronic noise 
	    \item Energy in the ZDCs consistent with neutrons on only one side 
		    of the interaction point
    \end{itemize}
    Each of these cuts are designed to reject topologies that are produced when
      the two colliding nuclei overlap and interact through the strong force. 
    The track requirement rejects events that produce many charge particles.
    Nearly all hadronic interactions (~98\%) produce particles in the 
      range $3<|\eta|<5$ covered by the HF detector.
    By requiring that the energy deposits in HF resemble noise, nearly all
      elastic hadronic collisions are expected to be rejected.
    In elastic hadronic events the both nuclei break-up. 
    By requiring that ZDC only reconstruct neutrons on one side of the 
      interaction point, hadronic interactions that produce neutrons on both 
      sides are reject.

  \section{break up determination}
  \section{Signal Extraction}
  \section{Efficiency determination}
